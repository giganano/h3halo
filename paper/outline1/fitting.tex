
\documentclass[ms.tex]{subfiles}
\begin{document}

\section{The Fitting Method}
\label{sec:fitting}

\begin{itemize}

	\item Introduce a new algorithm that fits the track itself to
	the~\feh~and~\afe~abundances of individual stars as opposed to binning the
	data and fitting the distribution.
	Though we use chemical abundances as our chief observational quantity, this
	procedure is highly generic and should in principle be applicable in any
	region of parameter space where there is intrinsic variation in the density
	of data points (e.g. isochrones in stellar evolution).

	\item We make use of~\mc~\cite{Foreman-Mackey2013} to run Markov Chain
	Monte Carlo (MCMC) fits of parameters in one-zone models of chemical
	evolution.
	At each step in parameter space,~\mc~makes a call to the~\texttt{Versatile
	Integrator for Chemical Evolution}~\citep[\vice;][]{Johnson2020,
	Griffith2021, Johnson2021} to compute the predicted abundances for that
	selection of parameters.
	We then compute the likelihood function $L(d|m)$ according to the following
	procedure.

	\item For a given realization of a one-zone model with known parameters~$m$
	and one-zone model predictions~$\mu$ = (\feh,~\afe,~\logage), the
	likelihood of the data given the model is equal to the product of the
	likelihoods of each individual data point:
	\begin{subequations}\begin{align}
	L(d|m) &= \prod_i L(d_i|m)
	\\
	\implies \ln L(d|m) &= \sum_i \ln L(d_i|m).
	\end{align}\end{subequations}
	However, for a given model~$m$, there is no guaranteed way of knowing
	which point~$m_j$ along the computed~\afe-\feh~track should correspond to
	some data point~$d_i$.
	We therefore marginalize over the entire track for every data point~$d_i$
	by summing the likelihoods from all~$m_j$ model vectors:
	\begin{subequations}\begin{align}
	L(d_i|m) &= \sum_j L(d_i|m_j)
	\\
	\implies \ln L(d|m) &= \sum_i \ln \left(\sum_j L(d_i|m_j)\right)
	\end{align}\end{subequations}

	\item We relate the data point~$d_i$ and the model point~$m_j$ with the
	relation~$L(d_i|m_j) \propto e^{-\chi^2/2}$ with
	$\chi^2 = \Delta_{ij}C_i^{-1}\Delta_{ij}^T$, where
	$\Delta_{ij} = \mu_{i,\text{data}} - \mu_{j,\text{model}}$ (i.e. the
	difference between a pair of data and model vectors) and $C_i^{-1}$ is the
	inverse covariance matrix of the~$i$th data point.

	\item Chemical evolution tracks, however, have real, intrinsic variations
	in the density of points along the track.
	In one case, a high density of data points may simply reflect the fact that
	the model vector~$\mu_{j,\text{data}}$ is not far from the vector from the
	previous timestep~$\mu_{j - 1,\text{data}}$.
	In another case, a high density of data points may reflect the fact that
	the star formation rate was high when the galaxy was passing through some
	region of parameter space.
	The density of points in the data may also vary because of non-uniform
	sampling.


\end{itemize}

\end{document}

