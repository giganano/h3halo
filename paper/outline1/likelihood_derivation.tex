
\documentclass[ms.tex]{subfiles}
\begin{document}

\section{Derivation of the Likelihood Function}
\label{sec:l_derivation}

\begin{itemize}

	\item In this appendix we provide a detailed derivation of our likelihood
	function (equation~\ref{eq:likelihood}) incorporating the principles of the
	IPPP.
	We make no assumptions about the underlying model other than that it
	predicts a track in some observed space.
	In our use case, this corresponds to the~\afe-\feh~plane, though it could
	be extended to higher dimensional abundance spaces.
	This approach should also be extensible to other astrophysical models which
	predict tracks in some observed space, such as stellar streams and
	isochrones.

	\item Given some expression, whether analytic or numerical, for the model
	predicted track in the observable space~\script{M}, the likelihood of
	observing the data given some set of model parameters~$\{\theta\}$ can be
	expressed as the integrated differential likelihood along the track:
	\begin{equation}
	L(\script{D} | \{\theta\}) = \int_\script{M} dL =
	\int_\script{M} L(\script{D} | \script{M}) P(\script{M} | \{\theta\})
	d\script{M},
	\end{equation}
	where~$P(\script{M} | \{\theta\})$ describes the probability that a
	singular datum will be drawn from the model at a given point along the
	track.
	The defining characteristic of the IPPP is that~$P(\script{M} | \{\theta\})$
	follows a Poisson distribution and can be related to the intensity
	function~$\lambda$ quantifying the density of points according to
	\begin{equation}
	P(\script{M}_1, \script{M}_2, \script{M}_3, ..., \script{M}_K | \{\theta\})
	= e^{-N_\lambda}\prod_i^N \lambda(\script{M}_j | \{\theta\}),
	\end{equation}
	where~$\script{M}_j$ denotes a specific position on the track and the
	product is taken over the~$N$ points in the sample~\script{D}.
	For notational convenience below, we write this as a product rather than
	an exponent~$\lambda(\script{M} | \{\theta\})^N$.
	$N_\lambda$ quantifies the expected number of instances in the data, which
	can be expressed as the line integral of the intensity function along the
	track as
	\begin{equation}
	N_\lambda = \int_\script{M} \lambda(\script{M} | \{\theta\}) d\script{M}.
	\end{equation}
	The intensity function~$\lambda(\script{M} | \{\theta\})$ describes the
	predicted~\textit{observed} density, and should therefore incorporate any
	selection effects present in the data~\script{D}.
	It can be expressed as the product of the selection function~\script{S}
	and the~\textit{intrinsic} density~$\Lambda$ according to
	\begin{equation}
	\lambda(\script{M} | \{\theta\}) = \script{S}(\script{M}, \{\theta\})
	\Lambda(\script{M} | \{\theta\})
	\label{eq:lambda_def}
	\end{equation}

	\item By plugging this into our expression for the likelihood function,
	we obtain:
	\begin{subequations}\begin{align}
	L(\script{D} | \{\theta\}) &=
	\int_\script{M} \left(\prod_i^N L(\script{D}_i | \script{M})\right)
	\left(e^{-N_\lambda}\prod_i^N \lambda(\script{M} | \{\theta\})\right)
	d\script{M}
	\\
	&= e^{-N_\lambda} \prod_i^N \int_\script{M}
	L(\script{D}_i | \script{M}) \lambda(\script{M} | \{\theta\}) d\script{M},
	\label{eq:L_product}
	\end{align}\end{subequations}
	where we have exploited the conditional independence of the likelihood of
	observing each individual datum~$\script{D}_i$, allowing us to substitute
	$L(\script{D} | \script{M}) = \prod_i^N L(\script{D}_i | \script{M})$.
	We have also dropped the subscript~$j$ in~$\lambda(\script{M}_j |
	\{\theta\})$ because we are computing the line integral along the track
	\script{M}, so a specific location~$\script{M}_j$ is implicit.
	% We note that the expected number of observed data~$N_\lambda$ is not a
	% function of the predicted track~\script{M} but only of the model parameters
	% $\{\theta\}$.

	\item Now taking the log of the likelihood function produces the following
	expression for~$\ln L$:
	\begin{equation}
	\ln L(\script{D} | \{\theta\}) = -N_\lambda + \sum_i^N \ln
	\left(\int_\script{M} L(\script{D}_i | \script{M})
	\lambda(\script{M} | \{\theta\})d\script{M}\right).
	\label{eq:lnL_integral}
	\end{equation}

	\item As the integral within the summation would suggest, the next task is
	to determine the appropriate manner in which to assess the likelihood of
	observing the singular datum~$\script{D}_i$ given the predicted
	track~$\script{M}$.
	Because of observational uncertainties, there is no way to know~\textit{a
	priori} which point on the track~$\script{M}$ any individual datum is truly
	associated with.
	If this were the case, then~$L(\script{D}_i | \script{M})$ would be a
	delta function at a given position~$\script{M}_j$ along the track.
	It is therefore essential to keep the integral within the logarithm in the
	above equation, effectively marginalizing over the entire track to
	quantify~$\ln L(\script{D}_i | \script{M})$.
	
	\item In practice, the track may be complicated in shape, and is generally
	not known as a smooth and continuous function, instead in some piece-wise
	linear approximation computed by a numerical code.
	The line segment connecting the knots on the track~$\script{M}_j$ and
	$\script{M}_{j + 1}$ can be expressed as:
	\begin{equation}
	\Delta\script{M}_{j,j + 1} = \script{M}_j + q(\script{M}_{j + 1} -
	\script{M}_j) \quad (0 \leq q \leq 1).
	\end{equation}
	If the errors on the observed datum~$\script{D}_i$ are accurately described
	by a multivariate Gaussian, then the likelihood of observing~$\script{D}_i$
	at a point along this line segment can be expressed as:
	\begin{subequations}\begin{align}
	L(\script{D}_i | \script{M}_j, q) &=
	\label{eq:l_di_mj_q}
	\frac{1}{\sqrt{2\pi \det(C_i)}}
	\exp\left(\frac{-1}{2}\Delta_{ij} C_{i}^{-1} \Delta_{ij}^T\right)
	\\
	\Delta_{ij} &= \script{D}_i - \Delta\script{M}_{j,j + 1}(q)
	\\
	&= \script{D}_i - \script{M}_j - q(\script{M}_{j + 1} - \script{M}_j)
	\label{eq:delta_segment}
	\\
	&= \delta_{ij} - q m_j
	\end{align}\end{subequations}
	where~$C_i$ is the covariance matrix of the~$i$th datum~$\script{D}_i$
	and~$\Delta_{ij}$ is the difference between the location of~$\script{D}_i$
	and the point along the track~$\Delta\script{M}_{j,j + 1}(q)$ in the
	observed space.
	For notational convenience, we have made the substitutions
	$\delta_{ij} = \script{D}_i - \script{M}_j$ denoting the vector difference
	between the datum~$\script{D}_i$ and the model point~$\script{M}_j$ and
	$m_j$ denoting the vector difference between the model points
	$\script{M}_{j + 1}$ and~$\script{M}_j$.

	\item If the observational uncertainties on any one datum~$\script{D}_i$
	are not sufficiently described by a multivariate Gaussian, then
	equation~\refp{eq:l_di_mj_q} must be replaced with some kernel density
	esimate of the uncertainty in the observed plane evaluated at the point
	$\Delta \script{M}_{j,j + 1}(q)$.
	Provided that the track is densely sampled compared to the width of the
	kernel density estimate (see discussion below), this should not require
	any additional integration along the line segments of the track to
	marginalize over~$q$.

	\item To marginalize over the full length of the line segment, we integrate
	this likelihood from~$q = 0$ to 1, but first, we compute the square and
	isolate the terms that depend on~$q$, which results in:
	\begin{equation}
	\label{eq:chi_squared_ij}
	\Delta_{ij}C_i^{-1}\Delta_{ij}^T = 
	\delta_{ij}C_i^{-1}\delta_{ij}^T - 2q \delta_{ij} C_i^{-1} m_j^T +
	q^2 m_j C_i^{-1} m_j^T
	\end{equation}
	Visual inspection of equation~\refp{eq:delta_segment} indicates that
	whenever the spacing between track points (i.e. $\script{M}_{j}$ and
	$\script{M}_{j + 1}$) is small compared to the observational uncertainties,
	then the second and third terms of equation~\refp{eq:chi_squared_ij} are
	negligible.
	In this case, the value of~$\Delta_{ij}C_i^{-1}\Delta_{ij}^T$ reduces to
	the value obtained by simply taking the vector difference between the
	datum~$\script{D}_i$ and the track point~$\script{M}_j$.

	\item If we consider the full line segment, integrating from~$q = 0$ to 1:
	\begin{subequations}\begin{align}
	L(\script{D}_i | \script{M}_j) &= \int_0^1 L(\script{D}_i | \script{M}_j,
	q) dq
	\\
	\begin{split}
	&= \frac{1}{\sqrt{2\pi \det(C_i)}}
	\exp\left(\frac{-1}{2}\delta_{ij}C_i^{-1}\delta_{ij}^T\right)
	\\
	&\qquad \int_0^1 \exp\left(\frac{-1}{2}(aq^2 - 2bq)\right)dq
	\end{split}
	\\
	\begin{split}
	&= \frac{1}{\sqrt{2\pi \det(C_i)}}
	\exp\left(\frac{-1}{2}\delta_{ij}C_i^{-1}\delta_{ij}^T\right)
	\sqrt{\frac{\pi}{2a}}
	\\
	&\qquad \exp\left(\frac{b^2}{2a}\right)
	\left[\erf\left(\frac{a - b}{\sqrt{2a}}\right) -
	\erf\left(\frac{b}{\sqrt{2a}}\right)\right]
	\end{split}
	\\
	a &= m_j C_i^{-1} m_j^T
	\\
	b &= \delta_{ij} C_i^{-1} m_j^T
	\end{align}\end{subequations}
	For simplicity, we introduce the corrective term~$\beta_{ij}$ given by
	\begin{equation}
	\beta_{ij} = \sqrt{\frac{\pi}{2a}}
	\exp\left(\frac{b^2}{2a}\right)
	\left[
	\erf\left(\frac{a - b}{\sqrt{2a}}\right) -
	\erf\left(\frac{b}{\sqrt{2a}}\right)
	\right],
	\label{eq:corrective_beta}
	\end{equation}
	such that~$L(\script{D}_i | \script{M}_j)$ is given by:
	\begin{equation}
	L(\script{D}_i | \script{M}_j) = \frac{\beta_{ij}}{\sqrt{2\pi \det(C_i)}}
	\exp\left(\delta_{ij}C_i^{-1}\delta_{ij}^T\right).
	\end{equation}
	By marginalizing over the line segment~$\Delta\script{M}_{j,j + 1}$, we can
	now express the integral in equation~\refp{eq:lnL_integral} as a summation
	over the points at which the track is sampled~$\script{M} = \{\script{M}_1,
	\script{M}_2, \script{M}_3, ..., \script{M}_K\}$:
	\begin{equation}\begin{split}
	&\ln L(\script{D} | \{\theta\}) = -N_\lambda -
	\sum_i^N \ln\left(\sqrt{2\pi \det(C_i)}\right) +
	\\ &\qquad \sum_i^N \ln \left(
	\sum_j^K \left(
	\beta_{ij}\exp\left(\delta_{ij}C_i^{-1}\delta_{ij}^T\right)
	\lambda(\script{M}_j | \{\theta\})
	\right)
	\right).
	\end{split}\end{equation}
	As long as the track is densely sampled relative to the observational
	uncertainties, then~$\beta_{ij} \approx 1$ and this term can be safely
	neglected.
	In some cases, computing the evolutionary track~$\script{M}$ may be
	computationally expensive, making it potentially advantageous to reduce the
	number of points computed in exchange for a slightly more complicated
	likelihood calculation.

	\item The remaining term in the summation above is the predicted density of
	points~$\lambda$ and its line integral along the track~$N_\lambda$.
	As discussed above, this parameter quantifies the model predicted density
	of observed points, incorporating the intrinsic density as well as any
	selection effects present in the data.
	In a one-zone GCE model, the predicted intrinsic density~$\Lambda$ is given
	by the SFH, module the small effect of mass loss from recycled stellar
	envelopes (see discussion in, e.g.,~\citealp{Weinberg2017}).
	In our use case,~$\Lambda$ should therefore be proportional to the SFH:
	\begin{equation}
	\Lambda(\script{M}_j | \{\theta\}) \propto \dot{M}_\star(t_j | \{\theta\}),
	\end{equation}
	where~$t$ denotes time in the GCE model.

	% is the unnormalized density of
	% points predicted by the model~$\rho$.
	% As discussed above, this parameter quantifies the model predicted density
	% of observed points, incorporating the intrinsic density as well as any
	% selection effects present in the data.
	% Based on equations~\refp{eq:rho_def} and~\refp{eq:lambda_def}, it can
	% be written in terms of these two functions as
	% \begin{equation}
	% \rho(\script{M} | \{\theta\}) = \frac{1}{N_\lambda}
	% \script{S}(\script{M}, \{\theta\})
	% \Lambda(\script{M} | \{\theta\}).
	% \end{equation}
	% In a one-zone GCE model, the predicted intrinsic density~$\Lambda$ is given
	% by the SFH, modulo the small effect of mass loss from recycled stellar
	% envelopes~\citep[see discussion in, e.g., ][]{Weinberg2017}.
	% In our use case of this likelihood function,~$\Lambda$ should therefore be
	% proportional to the SFH:
	% \begin{equation}
	% \Lambda(\script{M}_j | \{\theta\}) \propto \dot{M}_\star(t_j | \{\theta\})
	% \end{equation}
	% where~$t$ denotes time in the GCE model, and~$T$ is the time interval over
	% which the GCE model is integrated.

	\item This multiplicative factor on the likelihood~$L$ can be incorporated
	by simply letting the pair-wise component of the datum~$\script{D}_i$ and
	point on the model track~$\script{M}_j$ take on a weight
	$w_j \equiv \script{S}(\script{M}_j, \{\theta\})\dot{M}_\star(t_j |
	\{\theta\})$ which is determined by the SFH of the model and the selection
	function of the survey.
	The predicted number of instances~$N_\lambda$, originally expressed in
	terms of the intensity function~$\lambda$ itself, can now be expressed as
	the sum of the weights~$w_j$.
	This gives rise to the following expression for the likelihood function:
	\begin{equation}
	\ln L(\script{D} | \{\theta\}) \propto
	\sum_i^N \ln \left(
	\sum_j^K \beta_{ij} w_j \exp\left(\delta_{ij} C_i^{-1} \delta_{ij}^T\right)
	\right) - \sum_j^K w_j,
	\label{eq:lnL_withweights}
	\end{equation}
	where we have omitted the term~$\sum_i^N \ln\left(\sqrt{2\pi \det(C_i)}
	\right)$ because it is a constant which can safely be neglected in the
	interest of optimizing the likelihood function.

	\item In many one-zone GCE models, however, the normalization of the SFH is
	irrelevant in computing the evolution of the abundances.
	Because the metallicity is given by the metal mass~\textit{relative} to the
	ISM mass, this normalization generally cancels.
	In such a case, because the SFH determines the weights~$w_j$, it is
	essential that extra steps be taken to ensure that the sum of the weights
	not impact the inferred likelihood.

	\item To this end, we consider a density~$\rho$ with some unknown overall
	normalization defined relative to the intensity function~$\lambda$
	according to:
	\begin{subequations}\begin{align}
	\lambda(\script{M} | \{\theta\}) &= N_\lambda \rho(\script{M} | \{\theta\})
	\\
	\int_\script{M} \rho(\script{M} | \{\theta\}) d\script{M} &= 1.
	\end{align}\end{subequations}
	Plugging this into equation~\refp{eq:L_product} and taking the natural
	logarithm yields the following expression for the likelihood function:
	\begin{equation}\begin{split}
	\ln L(\script{D} | \{\theta\}) &= -N_\lambda + N \ln N_\lambda +
	\\ &\qquad
	\sum_i^N \ln \left(
	\int_\script{M} L(\script{D}_i | \script{M})
	\rho(\script{M} | \{\theta\}) d\script{M}\right).
	\end{split}\end{equation}
	By applying the same procedure to computing the integral as above, we
	arrive at the following expression for the likelihood when the intensity
	function~$\lambda$ is substituted for some un-normalized version of itself:
	\begin{equation}\begin{split}
	\ln L(\script{D} | \{\theta\}) &= -N_\lambda + N \ln N_\lambda +
	\\ &\qquad
	\sum_i^N \ln \left(\sum_j^K \beta_{ij}w_j
	\exp\left(\delta_{ij}C_i^{-1}\delta_{ij}^T\right)\right),
	\end{split}\end{equation}
	where for notational convenience below, we have left the sum of the weights
	written as~$N_\lambda$.
	In the interest of optimizing the likelihood function, we take the partial
	derivative of~$\ln L$ with respect to~$N_\lambda$, and we find that it is
	equal to zero when~$N = N_\lambda$.
	Because~$\rho$ is by definition un-normalized, we can simply choose this
	normalization, after which the first two terms in the above expression for
	$\ln L$ become~$-N + N \ln N$.
	A constant for any given data set~\script{D}, we can safely neglect these
	two terms in the interest of optimizing the likelihood function.
	This yields the following expression for the likelihood function which does
	not depend on the normalization of the SFH, as required:
	\begin{subequations}\begin{align}
	\ln L(\script{D} | \{\theta\}) &\propto
	\sum_i^N \ln \left(\sum_j^K
	\beta_{ij}w_j \exp\left(\delta_{ij}C_i^{-1}\delta_{ij}^T\right)\right)
	\label{eq:lnL_noweights}
	\\
	\sum_j^K w_j &= 1
	\label{eq:lnL_weightsum1},
	\end{align}\end{subequations}
	where the second expression comes from the requirement that the line
	integral of the un-normalized density~$\rho$ along the track equal 1.

	\item In summary, when inferring best-fit models for one-zone GCE models
	in which the normalization of the SFH is irrelevant to the evolution of the
	abundances, authors should adopt equations~\refp{eq:lnL_noweights} and
	\refp{eq:lnL_weightsum1}.
	If they have instead parametrized their model in such a way that the
	normalization does indeed impact the abundance evolution, then they should
	adopt equation~\refp{eq:lnL_withweights}.
	Such models arise when evolutionary parameters depend on the surface
	density of gas or star formation, such as a non-linear scaling between the
	surface densities of gas and star formation (i.e. the Kennicutt-Schmidt
	relation,~$\dot{\Sigma}_\star \propto \Sigma_\text{gas}^k$; refs), or a
	mass-loading factor~$\eta$ which grows with the stellar mass to mimic the
	deepening of the potential well (see, e.g., Conroy et al. 2022).
	In either case, the corrective term~$\beta_{ij}$ given by
	equation~\refp{eq:corrective_beta} is approximately 1 and can be safely
	neglected when the track is densely sampled relative to the observational
	uncertainties.
	In this paper, we assume a linear relation between the star formation rate
	and the ISM mass, which makes the normalization of the SFH irrelevant; we
	therefore adopt equations~\refp{eq:lnL_noweights}
	and~\refp{eq:lnL_weightsum1}.

\end{itemize}

\end{document}

