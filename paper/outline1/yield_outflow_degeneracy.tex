
\documentclass[ms.tex]{subfiles}
\begin{document}
\renewcommand\theequation{\thesection\arabic{equation}}
\renewcommand\thefigure{\thesection\arabic{figure}}
\setcounter{equation}{0}
\setcounter{figure}{0}

\section{The Yield-Outflow Degeneracy}
\label{sec:yield_outflow_degeneracy}

\begin{figure*}
\centering
\includegraphics[scale = 0.4]{degeneracy_25k6.pdf}
\caption{
The same as Fig.~\ref{fig:fiducialmock}, but with the alpha element yield from
massive stars~\yacc~as an additional free parameter.
Motivated both by theoretical models of nucleosynthesis in massive stars and
the convenience for scaling up or down, we have adopted~$\yacc \equiv 0.01$
in this paper to set the scale of this degeneracy.
}
\label{fig:degeneracy}
\end{figure*}

\begin{itemize}

	\item As discussed in~\S~\ref{sec:onezone:yields}, there is a
	degeneracy between the absolute scale of nucleosynthetic yields and the
	strength of outflows in GCE models.
	Under the instantaneous recycling approximation, early work in GCE
	demonstrated that galaxies with ongoing accretion of metal-poor gas
	reached an equilibrium metal abundance in which the newly produced metal
	mass is balanced by losses to star formtion and, if present, outflows
	(e.g.~\citealp{Larson1972}, and more recently~\citealp{Weinberg2017}).
	These ``open-box'' models offered a simple solution to the G-dwarf problem
	which plagued the ``closed-box'' models of GCE, whereby the frequency of
	super-solar metallicity stars was severely over-predicted (see the review
	in, e.g.,~\citealp{Tinsley1980}).

	\item These results were corroborated by~\citet{Dalcanton2007}, who argued
	that metal-enriched outflows are the only mechanism which can
	significantly reduce effective yields from SNe.
	However, recent theoretical explorations of SN explosions propose that
	many massive stars collapse directly to form black holes at the end of
	their lives as opposed to exploding as CCSNe (e.g.~\citealp{Ertl2016,
	Sukhbold2016}; see also discussion in~\citealp{Griffith2021}).
	This challenges the interpretation of~\citet{Dalcanton2007}, instead
	suggesting that SN yields are perhaps~\textit{intrinsically} lower and
	consequently may not require adjusting with metal-enriched outflows to
	alter effective yields.
	Observationally, it is feasible to constrain relative but not absolute
	yields.
	For example, the two-process model explored by~\citet{Weinberg2019,
	Weinberg2021} in APOGEE~\citep{Majewski2017} and by~\citet*{Griffith2019}
	and~\citet{Griffith2022} in GALAH~\citep{DeSilva2015, Martell2017}
	quantifies the median trends in abundance ratio relative to Mg along the
	high- and low-alpha sequences to disentangle the relative contributions of
	prompt and delayed nucleosynthetic sources to various elements.
	Abundance ratios can also be derived from individual supernova remnants as
	in, e.g.,~\citet*{Holland-Ashford2020}.

	\item We quantify the strength of this degeneracy by introducing the alpha
	yield~\yacc~as an additional free parameter in our fit to our fiducial
	mock sample.
	Otherwise, we follow the exact same procedure to recover the known
	evolutionary parameters of the mock as discussed
	in~\S\S~\ref{sec:fitting} and~\ref{sec:mocks:fiducial}.
	% In the main body of this paper, we have fixed the~\yacc~at a value of 0.01
	% in order to set an overall scale
	Fig.~\ref{fig:degeneracy} illustrates the results of this procedure in the
	form of the ``corner-plot'' with the marginalized likelihood distributions
	along the diagonal and the 2-dimensional cross-sections of the likelihood
	function below the diagonal.
	As expected, there are extremely strong degeneracies in all yields with
	one another and with the outflow parameter~$\eta$.
	Additionally, there is a degeneracy between the SFE timescale~$\tau_\star$
	and the yields which arises because the position of the ``knee'' in
	the~\afe-\feh~plane can be fit with either a high-yield and slow star
	formation or a low yield and high star formation.

	\item In detail, this degeneracy arises whenever a parameter influences
	the position or shape of the evolutionary track in the~\afe-\feh~diagram
	or the centroid of the MDF.
	The infall timescale~$\tau_\text{in}$ and the total duration of star
	formation~$\tau_\text{tot}$ are unaffected by this degeneracy because they
	effect only the shape of the MDF (see discussion
	in~\S~\ref{sec:mocks:fiducial}).
	Regardless of the choice of yields and of the parameters~$\eta$ and
	$\tau_\star$, the shape of the MDF is constrained by a sufficiently large
	sample, allowing precise derivations of these timescales from our fitting
	method.
	By determining the duration of star formation in this manner, this may
	open a new pathway to deriving quenching times for now quiescent dwarf
	galaxies and stellar streams orbiting local group spirals such as the
	Milky Way and Andromeda.

\end{itemize}

\end{document}

