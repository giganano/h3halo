
\documentclass[ms.tex]{subfiles}
\begin{document}

\section{Mock Samples}
\label{sec:mocks}

\subsection{A Fiducial Mock Sample}
\label{sec:mocks:fiducial}

\begin{figure*}
\centering
\includegraphics[scale = 0.5]{fiducial_mock_afe_feh.pdf}
\includegraphics[scale = 0.42]{fiducial_mock_agedist.pdf}
\includegraphics[scale = 0.42]{fiducial_mock_amr.pdf}
\caption{
\textbf{Left}: Our fiducial mock sample in the~\afe-\feh~plane.
There are~$N = 500$ stars with abundance uncertainties
of~$\sigma(\feh) = \sigma(\afe) = 0.05$ as indicated by the errorbar.
$N = 100$ of the stars have age information available with an artificial
uncertainty of~$\sigma(\log_{10}(\text{age})) = 0.1$ as indicated by the
colorbar.
The red line denotes the evolutionary track in the gas-phase from the one-zone
model that generated the mock.
On the top and right, we show the marginalized distributions
in~\afe~and~\feh, with red lines denoting the known distribution.
\textbf{Center}: The mock (black, binned) and known (red) age distributions.
The dashed red line indicates the age distribution that is obtained by sampling
$N = 10^4$ rather than $N = 500$ stars and assuming the same age uncertainty
of~$\sigma(\log_{10}(\text{age})) = 0.1$.
\textbf{Right}: The age-\feh~(top) and age-\afe~(bottom) relation for the mock
sample, with artificial uncertainties denoted by the error bars on each panel.
The red lines denotes the known relations for the gas-phase.
}
\label{fig:fiducialmock}
\end{figure*}

\begin{table}
\caption{
The known parameters of the one-zone model from which we generate our mock
stellar samples.
}
\begin{tabularx}{\columnwidth}{l @{\extracolsep{\fill}} l r}
\hline
Parameter & Description & Value
\\
\hline
$\tau_\text{in}$ & e-folding timescale of the infall history & 2 Gyr
\\
$\eta$ & Outflow mass-loading factor
($\eta \equiv \dot{M}_\text{out} / \dot{M}_\star$) & 10
\\
$\tau_\star$ & SFE timescale ($\tau_\star \equiv M_\text{g} / \dot{M}_\star$) &
15 Gyr
\\
$\tau_\text{tot}$ & Duration of star formation & 10 Gyr
\\
\yfecc & IMF-averaged fractional net Fe yield from CCSNe & 0.0008
\\
\yfeia & IMF-averaged fractional net Fe yield from SN Ia & 0.0011
\\
\hline
\end{tabularx}
\label{tab:fiducial_mock_params}
\end{table}

\begin{itemize}

	\item We use our parametrization of one-zone GCE models described
	in~\S~\ref{sec:methods:onezone} to set up an underlying model from which
	mock samples can be drawn; we then use a fiducial mock sample to describe
	our fitting method in~\S~\ref{sec:methods:fitting} and explore variations
	in, e.g., sample size and precision.

	\item We take an exponential infall history described by
	\begin{equation}
	\dot{M}_\text{in} \propto e^{-t/\tau_\text{in}}
	\end{equation}
	with~$\tau_\text{in} = 2$ Gyr and an initial gas mass of 0.
	The overall normalization of the infall history is irrelevant because
	mass information cancels in one-zone models when you compute abundances.
	We additionally select~$\tau_\star = 15$ Gyr and~$\eta = 10$ with the
	thought that slow star formation and strong outflows would mimic the
	evolution seen in a typical field dwarf galaxy.
	We set the onset of star formation~$\tau = 13.2$ Gyr ago, allowing~$\sim$0.5
	Gyr between the Big Bang and the first stars.
	We evolve this model for 10 Gyr (i.e. the exact ages of the youngest stars
	in the mock sample are~$\tau = 3.2$ Gyr).

	\item {\color{red} YIELDS}

	\item One-zone models produce stellar populations rather than individual
	stars, so if a mock sample of individual stars is to be obtained, we must
	sample from the underlying population.
	Higher mass stellar populations have proportionally more stars than lower
	mass stellar populations, so we take the probability of sampling to be
	proportional to the mass of a population.
	In the interest of mimicing typical observational samples for local group
	dwarfs, we take~$N = 500$ stars with abundance uncertainties of
	$\sigma\afe = \sigma\feh = 0.05$.
	100 of these stars have age information with an uncertainty
	of~$\sigma\logage = 0.1$.

	\item We illustrate this sample in Fig.~\ref{fig:fiducialmock}.
	This sample shows a ``knee'' in the~\afe-\feh~diagram near~\feh~$\sim$-2.3
	and an equilibrium abundance near~\feh~$\sim$-0.3, but due to the
	declining nature of the SFH, most of the stars form in
	the~\feh~$\sim$-1 and~\afe~$\sim$+0.2 region of chemical space.

\end{itemize}

\subsection{Recovered Parameters of the Fiducial Mock}
\label{sec:mocks:fiducial_fit}

\begin{figure*}
\centering
\includegraphics[scale = 0.45]{fiducial_76k8.pdf}
\caption{
The ``corner-plot'' showing the results of applying our fitting method to our
fiducial mock sample (see discussion in~\S\S~\ref{sec:methods:fitting}
and~\ref{sec:mocks:fiducial}).
We show the marginalized likelihood distributions in each parameter along with
their best-fit values and confidence intervals along the diagonal.
Below the diagonal, we show the 2-dimensional cross-sections of the
6-dimensional likelihood function.
Blue stars mark the element of the Markov Chain with the maximum likelihood.
Red ``cross-hairs'' denote the true, known values of the parameters which were
used to generate the mock sample (see Table~\ref{tab:fiducial_mock_params}).
}
\label{fig:corner_fiducial}
\end{figure*}

\begin{itemize}

	\item We apply the method outlined in~\S~\ref{sec:methods:fitting} to the
	mock sample detailed in~\S~\ref{sec:mocks:fiducial}.
	Fig.~\ref{fig:corner_fiducial} shows the ``corner-plot'' derived from this
	procedure.
	Along the diagonal, we show the marginalized likelihood distributions in
	each parameter along with their best-fit values and confidence intervals.
	Below the diagonal, we show 2-dimensional cross-sections of the
	6-dimensional likelihood function.
	Blue stars mark the element of the Markov Chain with the highest value
	of~$\ln L$, and red ``cross-hairs'' denote the true, known values of the
	parameters from the mock sample (see Table~\ref{tab:fiducial_mock_params}).

	\item Our fitting method is able to accurately recover the known
	evolutionary parameters of the mock sample.
	Although it may appear that there are a high number of~$\gtrsim1\sigma$
	discrepancies, we demonstrate in~\S~

	\item We note a handful of degeneracies in the likelihood distributions of
	the recovered parameters.
	{\color{red}
	Because some of these degeneracies arise from similar physics, this
	discussion can be shortened significantly by motivating items from that
	perspective rather than going through panel-by-panel.
	}

	\begin{itemize}

		\item[\textbf{1.}] There is an inverse relationship between the
		derived Fe yields from CCSN and SN Ia.
		This is expected, because for the fixed~$\alpha$ yield of~\yacc~= 0.01,
		the total yield of Fe is constrained by the~\feh~MDF, and for a high
		(low) value of~\yfecc, a low (high) value of~\yfeia~is required to
		make up the difference between the model and the data.

		\item[\textbf{2.}] There is an inverse relationship between the prompt
		component of the Fe yield~\yfecc~and the SFE timescale~$\tau_\star$.
		This arises because~$\tau_\star$ is the parameter which plays the
		dominant role in establishing the position of the ``knee'' in
		the~\afe-\feh~diagram by influencing the rate of enrichment at early
		times~\citep[see discussion in][]{Weinberg2017}.
		For high (low) values of~\yfecc, a low (high) value of~$\tau_\star$ is
		required in order to place the knee at a position consistent with the
		data.

		\item[\textbf{3.}] There is a direct relationship between the infall
		timescale~$\tau_\text{in}$ and the total duration of star formation
		$\tau_\text{tot}$.
		This arises because both affect the high-metallicity tail of the~\feh
		distribution similarly.
		With shorter~$\tau_\text{in}$, the gas supply and the SFH drop off more
		sharply.
		This increases the~\feh~equilibrium abundance by allowing SNe Ia to
		enrich a ``gas-starved'' ISM~\citep[see discussion in][]{Weinberg2017},
		thereby increasing the fraction of high~\feh~stars.
		To make up this potential difference with the data, the model can shut
		off star formation earlier, cutting off the high-metallicity end of
		the~\feh~MDF, giving rise to this direct relationship.

		\item[\textbf{4.}] There is an inverse relationship between the infall
		timescale and the mass loading factor~$\eta$.
		This arises because both affect the equilibrium abundance in the same
		manner.
		As discussed above, shorter values of~$\tau_\text{in}$ increase the
		equilibrium Fe abundance slightly by allowing SNe Ia to enrich a
		``gas-starved'' ISM.
		To make up for this potential discrepancy with the data, one can
		strengthen outflows by increasing~$\eta$, decreasing the equilibrium
		abundance.

		\item[\textbf{5.}] There is a degeneracy between the total duration of
		star formation~$\tau_\text{tot}$ and the SFE timescale~$\tau_\star$.
		This arises because both affect the shape of the~\feh~MDF.
		Fe approaches equilibrium on either the SFE timescale~$\tau_\star$ or
		the SN Ia DTD characteristic timescale - whichever is longer, which for
		dwarf galaxies with inefficient star formation, is~$\tau_\star$.
		As a result, higher~$\tau_\star$ models approach equilibrium faster.
		To avoid overpredicting the frequency of near-equilibrium stars
		compared to the data, one can simply cut off star formation earlier,
		thereby increasing the frequency of lower metallicity stars.

		\item[\textbf{6.}] {\color{red}
		The~$\eta$~\yfeia~degeneracy?
		It goes in the opposite direction that one would intuitively think:
		higher~\yfeia~models prefer lower~$\eta$.
		}

	\end{itemize}

	\item There are additional, weaker degeneracies which can be understood
	by confounding variables.
	For example, the degeneracy between~$\tau_\star$ and~\yfeia~arises because
	both are degenerate with~\yfecc.

	\item We remind the reader that our fits achieve such a precision by
	selecting a scale on which the~$\alpha$ element yield~\yacc~is defined to
	be exactly 0.01 (see discussion in~\S~\ref{sec:methods:onezone:yields}).
	Aside from the infall timescale~$\tau_\text{in}$ and the total duration of
	star formation~$\tau_\text{tot}$, each parameter in this fit is affected by
	the yield-outflow degeneracy.
	We quantify this in detail in Appendix X.

\end{itemize}

\subsection{Variations in Sample Size, Measurement Precision, and the
Availability of Age Information}
\label{sec:mocks:variations}

\subfile{mocksamples.tablebody.tex}

% {\renewcommand{\arraystretch}{1.8}
% \begin{table*}
% \caption{
% Known (top row) and recovered best-fit values of the evolutionary parameters
% used to generated mock data samples (see discussion in~\S~\ref{sec:mocks}).
% We conduct fits for variations of our fiducial mock data in sample size (top
% block), measurement uncertainty in~\feh~and~\afe~abundances (top-middle block),
% measurement uncertainty in~$\log_{10}(\text{age})$ (bottom-middle block), and
% the fraction of the sample with available age measurements (bottom block).
% We provide illustrations of the accuracy and precision of these fits in
% Figs.~\ref{fig:dp_sigma} and~\ref{fig:precision}, respectively.
% }
% \begin{tabularx}{\textwidth}{l @{\extracolsep{\fill}} c c c c c c}
% \hline
% Mock Sample & $\tau_\text{in}$ & $\eta$ & $\tau_\star$ & $\tau_\text{tot}$ &
% \yfecc & \yfeia
% \\
% \hline
% \hline
% \null & 2 Gyr & 10 & 15 Gyr & 10 Gyr & \scinote{8}{-4} & \scinote{1.1}{-3}
% \\
% \hline
% \hline
% $N = 20$ & $2.55^{+0.75}_{-0.45}$ Gyr & $8.39^{+1.11}_{-1.30}$ &
% $14.35^{+5.56}_{-3.32}$ Gyr & $10.60^{+1.65}_{-1.09}$ Gyr &
% $\scinote{7.9^{+1.2}_{-1.9}}{-4}$ & $\scinote{1.36^{+0.32}_{-0.23}}{-3}$
% \\
% $N = 50$ & $2.13^{+0.42}_{-0.36}$ Gyr & $10.39^{+0.80}_{-0.76}$ &
% $13.75^{+2.79}_{-2.38}$ Gyr & $11.25^{+1.37}_{-1.76}$ Gyr &
% $\scinote{(8.3 \pm 0.6)}{-4}$ & $\scinote{0.96 \pm 0.14}{-3}$
% \\
% $N = 100$ & $2.06^{+0.27}_{-0.26}$ Gyr & $9.88^{+0.64}_{-0.62}$ &
% $15.06^{+2.00}_{-1.79}$ Gyr & $11.52^{+1.06}_{-1.30}$ Gyr &
% $\scinote{(8.1 \pm 0.4)}{-4}$ & $\scinote{1.08 \pm 0.09}{-3}$
% \\
% $N = 200$ & $2.10^{+0.18}_{-0.17}$ Gyr & $10.11^{+0.45}_{-0.43}$ &
% $14.61^{+1.34}_{-1.18}$ Gyr & $10.60^{+1.07}_{-0.86}$ Gyr &
% $\scinote{(7.7 \pm 0.3)}{-4}$ & $\scinote{1.14 \pm 0.07}{-3}$
% \\
% $\bm{N = 500}$ & $\bm{1.84 \pm 0.11}$ \textbf{Gyr} &
% $\bm{9.98^{+0.30}_{-0.29}}$ & $\bm{14.01^{+0.86}_{-0.84}} \textbf{Gyr}$ &
% $\bm{9.41^{+0.63}_{-0.56}}$ \textbf{Gyr} & $\bm{\scinote{(8.3 \pm 0.2)}{-4}}$ &
% $\bm{\scinote{(1.05 \pm 0.05)}{-3}}$
% \\
% $N = 1000$ & $2.05^{+0.09}_{-0.08}$ Gyr & $9.72 \pm 0.20$ &
% $14.62^{+0.57}_{-0.56}$ Gyr & $9.83^{+0.38}_{-0.39}$ Gyr &
% $\scinote{(8.1 \pm 0.1)}{-4}$ & $\scinote{(1.14 \pm 0.03)}{-3}$
% \\
% $N = 2000$ & $2.00 \pm 0.05$ Gyr & $10.26 \pm 0.15$ &
% $15.82^{+0.44}_{-0.42}$ Gyr & $10.30^{+0.25}_{-0.32}$ Gyr &
% $\scinote{(8.0 \pm 0.1)}{-4}$ & $\scinote{(1.09 \pm 0.02)}{-3}$
% \\
% \hline
% \hline
% $\sigma_\text{[X/Y]} = 0.01$ & $1.89 \pm 0.10$ Gyr & $10.25 \pm 0.28$ &
% $15.06^{+0.52}_{-0.47}$ Gyr & $9.70^{+0.51}_{-0.59}$ Gyr &
% $\scinote{(8.0 \pm 0.1)}{-4}$ & $\scinote{(1.08 \pm 0.02)}{-3}$
% \\
% $\sigma_{\text{[X/Y]}} = 0.02$ & $1.92^{+0.10}_{-0.09}$ Gyr & $10.10 \pm 0.25$ &
% $14.71^{+0.56}_{-0.55}$ Gyr & $9.79^{+0.45}_{-0.40}$ Gyr &
% $\scinote{(8.1 \pm 0.1)}{-4}$ & $\scinote{1.08^{+0.02}_{-0.03}}{-3}$
% \\
% $\bm{\sigma_{\textbf{[X/Y]}} = 0.05}$ & $\bm{1.84 \pm 0.11}$ \textbf{Gyr} &
% $\bm{9.98^{+0.30}_{-0.29}}$ & $\bm{14.01^{+0.86}_{-0.84}}$ \textbf{Gyr} &
% $\bm{9.41^{+0.63}_{-0.56}}$ \textbf{Gyr} & $\bm{\scinote{(8.3 \pm 0.2)}{-4}}$ &
% $\bm{\scinote{(1.05 \pm 0.05)}{-3}}$
% \\
% $\sigma_\text{[X/Y]} = 0.1$ & $2.00^{+0.13}_{-0.12}$ Gyr &
% $9.88^{+0.31}_{-0.33}$ & $13.39 \pm 1.02$ Gyr & $11.10^{+1.00}_{-0.84}$ Gyr &
% $\scinote{8.5^{+0.4}_{-0.3}}{-4}$ & $\scinote{(1.01 \pm 0.07)}{-3}$
% \\
% $\sigma_\text{[X/Y]} = 0.2$ & $2.22 \pm 0.21$ Gyr & $9.83^{+0.58}_{-0.67}$ &
% $18.21^{+2.19}_{-2.02}$ Gyr & $10.32^{+1.05}_{-0.67}$ Gyr &
% $\scinote{(8.7 \pm 0.7)}{-4}$ & $\scinote{(1.05 \pm 0.14)}{-3}$
% \\
% $\sigma_\text{[X/Y]} = 0.5$ & $2.73^{+0.82}_{-0.60}$ Gyr &
% $10.05^{+1.22}_{-1.26}$ & $12.52^{+3.75}_{-3.35}$ Gyr &
% $9.00^{+1.26}_{-0.95}$ Gyr & $\scinote{7.5^{+1.8}_{-1.6}}{-4}$ &
% $\scinote{(1.12 \pm 0.31)}{-3}$
% \\
% \hline
% \hline
% $\sigma_{\log(\text{age})} = 0.02$ & $2.08^{+0.09}_{-0.08}$ Gyr &
% $9.84^{+0.24}_{-0.26}$ & $14.69^{+0.50}_{-0.46}$ Gyr &
% $10.41^{+0.47}_{-0.41}$ Gyr & $\scinote{(8.1 \pm 0.2)}{-4}$ &
% $\scinote{1.11^{+0.05}_{-0.04}}{-3}$
% \\
% $\sigma_{\log(\text{age})} = 0.05$ & $1.96 \pm 0.11$ Gyr &
% $9.88^{+0.32}_{-0.30}$ & $15.70^{+0.71}_{-0.68}$ Gyr &
% $9.95^{+0.63}_{-0.53}$ Gyr & $\scinote{(8.0 \pm 0.2)}{-4}$ &
% $\scinote{1.11^{+0.05}_{-0.04}}{-3}$
% \\
% $\bm{\sigma_{\log(\textbf{age})} = 0.1}$ & $\bm{1.84 \pm 0.11}$ \textbf{Gyr} &
% $\bm{9.98^{+0.30}_{-0.29}}$ & $\bm{14.01^{+0.86}_{-0.84}}$ \textbf{Gyr} &
% $\bm{9.41^{+0.63}_{-0.56}}$ \textbf{Gyr} & $\bm{\scinote{(8.3 \pm 0.2)}{-4}}$ &
% $\bm{\scinote{(1.05 \pm 0.05)}{-3}}$
% \\
% $\sigma_{\log(\text{age})} = 0.2$ & $2.20^{+0.18}_{-0.17}$ Gyr &
% $9.83^{+0.28}_{-0.27}$ & $15.19 \pm 1.11$ Gyr & $10.76^{+0.85}_{-0.93}$ Gyr &
% $\scinote{(8.0 \pm 0.2)}{-4}$ & $\scinote{1.11^{+0.05}_{-0.04}}{-3}$
% \\
% $\sigma_{\log(\text{age})} = 0.5$ & $2.25^{+0.20}_{-0.25}$ Gyr &
% $9.86^{+0.28}_{-0.30}$ & $16.24^{+1.24}_{-1.62}$ Gyr &
% $11.38^{+1.00}_{-1.34}$ Gyr & $\scinote{(8.0 \pm 0.2)}{-4}$ &
% $\scinote{(1.10 \pm 0.05)}{-3}$
% \\
% $\sigma_{\log(\text{age})} = 1.0$ & $1.69^{+0.35}_{-0.32}$ Gyr &
% $9.53 \pm 0.29$ & $12.38^{+2.27}_{-2.08}$ Gyr & $8.66^{+1.86}_{-1.74}$ Gyr &
% $\scinote{(8.3 \pm 0.3)}{-4}$ & $\scinote{(1.15 \pm 0.06)}{-3}$
% \\
% \hline
% \hline
% $f_\text{age} = 0.0$ & $1.65^{+0.55}_{-0.37}$ Gyr & $9.39^{+0.30}_{-0.29}$ &
% $11.80^{+3.36}_{-2.44}$ Gyr & $7.35^{+2.62}_{-1.74}$ Gyr &
% $\scinote{(8.3 \pm 0.4)}{-4}$ & $\scinote{1.19^{+0.08}_{-0.07}}{-3}$
% \\
% $f_\text{age} = 0.1$ & $1.75^{+0.16}_{-0.17}$ Gyr & $10.06^{+0.29}_{-0.28}$ &
% $13.65^{+1.22}_{-1.12}$ Gyr & $8.84 \pm 0.87$ Gyr &
% $\scinote{(8.4 \pm 0.2)}{-4}$ & $\scinote{(1.06 \pm 0.05)}{-3}$
% \\
% $\bm{f_\textbf{age} = 0.2}$ & $\bm{1.84 \pm 0.11}$ \textbf{Gyr} &
% $\bm{9.98^{+0.30}_{-0.29}}$ & $\bm{14.01^{+0.86}_{-0.84}}$ \textbf{Gyr} &
% $\bm{9.41^{+0.63}_{-0.56}}$ \textbf{Gyr} & $\bm{\scinote{(8.3 \pm 0.2)}{-4}}$ &
% $\bm{\scinote{(1.05 \pm 0.05)}{-3}}$
% \\
% $f_\text{age} = 0.3$ & $1.94^{+0.11}_{-0.10}$ Gyr & $9.80^{+0.27}_{-0.28}$ &
% $14.26^{+0.74}_{-0.67}$ Gyr & $9.89^{+0.54}_{-0.48}$ Gyr &
% $\scinote{(8.0 \pm 0.2)}{-4}$ & $\scinote{(1.10 \pm 0.04)}{-3}$
% \\
% $f_\text{age} = 0.4$ & $1.91^{+0.09}_{-0.10}$ Gyr & $10.07^{+0.32}_{-0.30}$ &
% $16.79^{+0.81}_{-0.83}$ Gyr & $10.34^{+0.61}_{-0.50}$ Gyr &
% $\scinote{(7.8 \pm 0.2)}{-4}$ & $\scinote{(1.12 \pm 0.05)}{-3}$
% \\
% $f_\text{age} = 0.5$ & $2.00 \pm 0.10$ Gyr & $10.16^{+0.30}_{-0.29}$ &
% $15.46^{+0.70}_{-0.69}$ Gyr & $9.83^{+0.48}_{-0.40}$ Gyr &
% $\scinote{(7.8 \pm 0.2)}{-4}$ & $\scinote{1.12^{+0.05}_{-0.04}}{-3}$
% \\
% $f_\text{age} = 0.6$ & $2.18 \pm 0.09$ Gyr & $9.65^{+0.27}_{-0.25}$ &
% $14.25^{+0.67}_{-0.64}$ Gyr & $10.49^{+0.44}_{-0.37}$ Gyr &
% $\scinote{(7.8 \pm 0.2)}{-4}$ & $\scinote{(1.15 \pm 0.04)}{-3}$
% \\
% $f_\text{age} = 0.7$ & $1.99 \pm 0.08$ Gyr & $9.81^{+0.28}_{-0.27}$ &
% $14.92^{+0.68}_{-0.62}$ Gyr & $10.25^{+0.46}_{-0.37}$ Gyr &
% $\scinote{(8.1 \pm 0.2)}{-4}$ & $\scinote{(1.08 \pm 0.04)}{-3}$
% \\
% $f_\text{age} = 0.8$ & $2.06 \pm 0.09$ Gyr & $9.53^{+0.29}_{-0.26}$ &
% $15.18^{+0.63}_{-0.59}$ Gyr & $9.76^{+0.36}_{-0.33}$ Gyr &
% $\scinote{(7.9 \pm 0.2)}{-4}$ & $\scinote{(1.15 \pm 0.05)}{-3}$
% \\
% $f_\text{age} = 0.9$ & $1.93 \pm 0.08$ Gyr & $10.41 \pm 0.31$ &
% $16.23^{+0.73}_{-0.70}$ Gyr & $10.03^{+0.39}_{-0.33}$ Gyr &
% $\scinote{(7.7 \pm 0.2)}{-4}$ & $\scinote{(1.14 \pm 0.04)}{-3}$
% \\
% $f_\text{age} = 1.0$ & $2.13 \pm 0.09$ Gyr & $9.44^{+0.28}_{-0.27}$ &
% $15.67^{+0.64}_{-0.60}$ Gyr & $10.21^{+0.35}_{-0.31}$ Gyr &
% $\scinote{(8.0 \pm 0.2)}{-4}$ & $\scinote{(1.15 \pm 0.05)}{-3}$
% \\
% \hline
% \hline
% \end{tabularx}
% \label{tab:recovered_values}
% \end{table*}
% }

\begin{figure*}
\centering
\includegraphics[scale = 0.45]{dp_sigma_samplesize.pdf}
\includegraphics[scale = 0.45]{dp_sigma_precision.pdf}
\includegraphics[scale = 0.45]{dp_sigma_agefrac.pdf}
\caption{
The mean deviation between the re-derived mock sample parameters~$\{\theta\} =
\{\tau_\text{in},~\eta,~\tau_\star,~\tau_\text{tot},~\yfecc,~\yfeia\}$ and
their known values from the mock sample in units of the uncertainty on the
best-fit values as a function of the sample size of the mock (left),
measurement precision in abundances~\feh~and~\afe (middle, black),
measurement precision in~$\log_{10}(\text{age})$ (middle, red),
and the fraction of the sample with available age information (right).
Error bars denote the error on the mean deviation in the six parameters.
Blue dotted lines mark~$\langle\Delta\theta/\sigma\rangle = 1$, the
expected value of the mean deviation for a Gaussian random process.
}
\label{fig:dp_sigma}
\end{figure*}

\begin{figure*}
\centering
\includegraphics[scale = 0.45]{precision_samplesize.pdf}
\includegraphics[scale = 0.45]{precision_uncertainty.pdf}
\includegraphics[scale = 0.45]{precision_agefrac.pdf}
\caption{
The mean precision of our re-derived mock sample parameters~$\{\theta\} =
\{\tau_\text{in},~\eta,~\tau_\star,~\tau_\text{tot},~\yfecc,~\yfeia\}$ as a
function of the sample size of the mock (left), measurements precision in
abundances~\feh~and~\afe (middle, black), measurement precision
in~$\log_{10}(\text{age})$ (middle, red), and the fraction of the sample with
age information (right).
Error bars denote the error on the mean precision of the six parameters.
}
\label{fig:precision}
\end{figure*}

\begin{itemize}

	\item Next we explore variations in details surrounding the mock sample
	itself, such as sample size, measurement precision, and the availability of
	age information.

	\item In Fig.~\ref{fig:dp_sigma}, we show the mean deviation between each
	re-derived parameter~$\theta$ and its known value from the mock sample in
	units of the precision on~$\theta$.
	In the left panel, we show this quantity as a function of sample size.
	We consider samples of~$N = 20$, 50, 100, 200, 500 (fiducial), 1000, and
	2000.
	For all sample sizes, the re-derived parameters
	$\{\theta\} = \{\theta_1, \theta_2, \theta_3, ..., \theta_J\}$ are
	derived within~$\sim1\sigma$ or slightly better of their known values.
	This is consistent with a Gaussian random process, because even with
	infinite data, the most likely deviation from the true value is exactly
	1$\sigma$.
	This suggests that our method for deriving the evolutionary parameters of
	dwarf galaxies accurately described by one-zone GCE models should be
	accurate even when only~$N \approx 20$ stars are available.
	We do not explore mock samples smaller than~$N = 20$ because we fit our
	mock samples with~$J = 6$ free parameters, and below~$N = 20$, the number
	of degrees of freedom drops rapidly.

\end{itemize}

\end{document}



























