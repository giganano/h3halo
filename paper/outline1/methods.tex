
\documentclass[ms.tex]{subfiles}
\begin{document}

% \section{Methods}
% \label{sec:methods}

% \begin{itemize}

% 	\item We are interested in applying one-zone GCE models to dwarf galaxies
% 	and determining best-fit parameters.
% 	We begin by providing background on one-zone models, and then we select
% 	a parametrization from which we draw a fiducial mock stellar sample.
% 	We then use these data to introduce our fitting method.

% \end{itemize}

\section{One-Zone Models of Galactic Chemical Evolution}
\label{sec:onezone}

\begin{itemize}

	\item GCE bridges the gap between nuclear physics and astrophysics.
	By combining galactic processes such as star formation, one can estimate
	the production rates of various nuclear species by stars and derive their
	abundances in the interstellar medium (ISM).
	This field has its roots in the early works assessing the astrophysical
	sites and conditions which give rise to nuclear reactions, a review of
	which can be found in the famous ``B2FH'' paper~\citep{Burbidge1957}.
	Here we provide a brief overview of the GCE formalismof one-zone models,
	historical reviews of which can be found in works such as
	\citet{Tinsley1980},~\citet{Pagel2009} and~\citet{Matteucci2012,
	Matteucci2021}.
	We then select a parametrization from which we draw a fiducial mock stellar
	sample, and we use these synthetic data to introduce our fitting method
	in~\S~\ref{sec:fitting}.

	\item The fundamental assumption of one-zone models is that newly produced
	metals mix instantaneously throughout the star forming gas reservoir.
	In real galaxies, this assumption is valid as long as the mixing time-scale
	is short compared to the depletion time-scale (i.e. the average time a
	fluid element remains in the ISM before getting incorporated into new
	stars or ejected in an outflow, if present).
	Based on the observations of~\citet{Leroy2008},~\citet*{Weinberg2017}
	calculate that characteristic depletion times can range from~$\sim$500 Myr
	up to~$\sim$10 Gyr for conditions in typical star forming disc galaxies.
	In the dwarf galaxy regime, the length scales are short, star formation
	is slow~\citep{Hudson2015}, and the ISM velocities are turbulent
	\citep{Dutta2009, Stilp2013, Schleicher2016}.
	With this combination, instantaneous mixing should be a good approximation.
	We are unaware of any studies in the literature which address this
	observationally, though in principle it should be feasible on a
	galaxy-by-galaxy basis by assessing the sources of scatter in chemical
	space (e.g. the~\afe-\feh~plane).
	If instantaneous mixing is a valid approximation, then intrinsic scatter
	about some evolutionary track should be negligible and the observed scatter
	should be dominated by measurement uncertainty.
	This assumption eliminates the need for spatial information, reducing GCE
	to a system of coupled integro-differential equations which can be solved
	numerically; we discuss them below.

\end{itemize}

\subsection{Inflows, Outflows, Star Formation, and Recycling}
\label{sec:onezone:gas}

\begin{itemize}

	\item At a given moment in time, gas is added to the interstellar medium
	(ISM) via inflows and recycled stellar envelopes and is taken out of the
	ISM via outflows and new stars.
	This gives rise to the following differential equation describing the
	evolution of the gas-supply:
	\begin{equation}
	\label{eq:mdot_gas}
	\dot{M}_\text{g} = \dot{M}_\text{in} - \dot{M}_\star - \dot{M}_\text{out}
	+ \dot{M}_\text{r},
	\end{equation}
	where~$\dot{M}_\text{in}$ is the infall rate,~$\dot{M}_\star$ is the star
	formation rate (SFR),~$\dot{M}_\text{out}$ is the outflow rate,
	and~$\dot{M}_\text{r}$ is the return of stellar envelopes from previous
	generations of stars.

	\begin{itemize}
		\item We relate the SFR to the gas supply by introducing the ``star
		formation efficiency (SFE) timescale'':
		\begin{equation}
		\tau_\star \equiv \frac{M_\text{g}}{\dot{M}_\star},
		\end{equation}
		This quantity is often referred to as the ``depletion time'' in the
		observational literature~\citep[e.g.][]{Tacconi2018}.
		This nomenclature, taken from~\citet{Weinberg2017}, is based on its
		inverse~$\tau_\star^{-1}$ often being referred to as the SFE itself
		because it describes the~\textit{fractional} rate at which some ISM
		fluid element is forming stars.

		\item There are various prescriptions for outflows in the literature.
		Some authors~\citep[e.g.][]{Andrews2017, Weinberg2017} assume a linear
		proportionality between the two:
		\begin{equation}
		\label{eq:eta}
		\dot{M}_\text{out} \equiv \eta\dot{M}_\star.
		\end{equation}
		Recently,~\citet{delosReyes2022} constrained the evolution of the
		Sculptor dwarf spheroidal galaxy with a linear proportionality between
		the SFR and the SN rate~$\dot{N}_\text{II} + \dot{N}_\text{Ia}$.
		\citet*{Kobayashi2020} developed a model in which outflow-driving winds
		develop in the early phases of the Milky Way's evolution, but die out
		on some timescale as the Galaxy grows.
		For modelling the Milky Way, some authors neglect outflows, arguing
		that they do not signicantly alter the chemical evolution of the
		disc~\citep[e.g.][]{Spitoni2019, Spitoni2021}.
		In our mock sample and in our fits to the GSE and the Sagitarrius dSph,
		we assume the linear proportionality given by equation~\refp{eq:eta}.
		Our fitting routine, however, is easily extended to the parametrization
		of~\citet{delosReyes2022}, and if outflows are to be neglected, one can
		simply take~$\eta = 0$ in their fit.

		\item The recycling rate~$\dot{M}_\text{r}$, in general, depends on the
		stellar IMF~\citep[e.g.][]{Salpeter1955, Miller1979, Kroupa2001,
		Chabrier2003}, the initial-final remnant mass relation
		\citep[e.g.][]{Kalirai2008}, and mass-lifetime relation
		(e.g.~\citealp{Larson1974, Maeder1989};~\citealp*{Hurley2000}).
		A single stellar population returns some fraction of its initial
		mass~$r$ back to the ISM according to:
		\begin{equation}
		\label{eq:crf}
		r(\tau) = \ddfrac{
			\int_{m_\text{to}(\tau)}^u (m - m_\text{rem})\frac{dN}{dm} dm
		}{
			\int_l^u m \frac{dN}{dm} dm
		}
		\end{equation}
		where~$l$ and~$u$ are the lower and upper mass limits of star formation,
		respectively,~$m_\text{to}(\tau)$ is the turnoff mass of a stellar
		population of age~$\tau$,~$m_\text{rem}$ is the mass of a remnant left
		behind by a star of initial mass~$m$, and~$dN/dm$ is the adopted IMF.
		Under this prescription, the recycling rate from~\textit{many} stellar
		populations, taking into account the full SFH, is given by:
		\begin{equation}
		\label{eq:mdot_recycled}
		\dot{M}_\text{r} = \int_0^T \dot{M}_\star(t) \dot{r}(T - t) dt
		\end{equation}
		where~$T$ is the time in the model.
		Due to the steep nature of the mass-lifetime relation, the recycling
		rate is dominated by young stellar populations.
		\citet{Weinberg2017} demonstrate that it is sufficiently accurate in
		one-zone models to assume that some fraction~$r_\text{inst}$ of a
		stellar population's initial mass is returned to the ISM immediately
		(see their Fig. 7; they recommend~$r_\text{inst} = 0.4$ for a
		\citealt{Kroupa2001} IMF, and~$r_\text{inst} = 0.2$ for a
		\citealt{Salpeter1955} IMF).
		Although it is simpler to assume~$\dot{M}_\text{r} =
		r_\text{inst}\dot{M}_\star$, numerical integration of equations
		\refp{eq:crf} and~\refp{eq:mdot_recycled} is easy, and~\vice~already
		does it, so we stick with that.

		\item \citet{Weinberg2017} demonstrate that~$\tau_\star$ and~$\eta$
		determine the first-order details of the gas-phase evolutionary track
		in the~\afe-\feh~plane (see their Fig. 2).
		With low~$\tau_\star$ (i.e. high SFE), nucleosynthesis is fast because
		star formation is fast, and a higher metallicity can be obtained before
		the onset of SN Ia than in lower SFE models.
		For this reason,~$\tau_\star$ plays the dominant role in shaping the
		position of the knee in the~\afe-\feh~plane.
		As the galaxy evolves, it approaches a chemical equilibrium in which
		newly produced metals are balanced by the loss of metals to outflows
		and new stars.
		Controlling the strength of the sink term of outflows,~$\eta$ plays
		the dominant role in shaping the late-time equilibrium abundance of the
		model, which high outflow models (i.e. high~$\eta$) predicting lower
		equilibrium abundances than their weak outflow counterparts.
		For observed data, the shape of the track itself directly constrains
		these parameters.
		The detailed form of the SFH has minimal impact on the shape of the
		tracks; rather, that information is encoded in the density of points
		along the evolutionary track and in the stellar metallicity
		distribution functions (MDFs).

	\end{itemize}

\end{itemize}

\subsection{Core Collapse Supernovae}
\label{sec:onezone:ccsne}

\begin{itemize}

	\item Massive stars, their winds, and their supernovae enrich the ISM on
	short timescales due to their short lifetimes.
	As long as the relevant timescales for galaxy evolution are significantly
	longer than the lifetimes of massive stars, it is adequate to approximate
	this nucleosynthetic material as ejected instantaneously following a single
	stellar population's formation.
	This implies a linear relationship between the CCSN enrichment rate and
	the SFR:
	\begin{equation}
	\label{eq:mdot_cc}
	\dot{M}_\text{x}^\text{CC} = y_\text{x}^\text{CC}\dot{M}_\star
	\end{equation}
	where~$y_\text{x}^\text{CC}$ is the IMF-averaged fractional net yield from
	massive stars.
	In parameterizing this term with an IMF-averaged yield, we ipmlicitly assume
	that the stellar mass is sufficiently high such that stochastic sampling of
	the IMF is an insignificant effect.

\end{itemize}

\subsection{Type Ia Supernovae}
\label{sec:onezone:sneia}

\begin{itemize}

	\item Unlike CCSN enrichment, SN Ia enrichment can occur on delay
	timescales of a Gyr or more.
	In general, the enrichment rate can be expressed as an integral over the
	SFH weighted by the delay-time distribution (DTD):
	\begin{equation}
	\dot{M}_\text{x}^\text{Ia} = y_\text{x}^\text{Ia} \ddfrac{
		\int_0^{T - t_\text{D}} \dot{M}_\star(t) R_\text{Ia}(T - t) dt
	}{
		\int_0^\infty R_\text{Ia}(t) dt
	},
	\end{equation}
	where~$R_\text{Ia}(t)$ is the DTD itself.
	By comparing the cosmic SFH~\citep[e.g.][]{Madau2014} with the cosmic SN
	Ia rate, the cosmic SN Ia DTD appears consistent with a uniform~$t^{-1}$
	power-law (e.g.~\citealp{Maoz2012a};~\citealp*{Maoz2012b, Graur2013}).
	Following~\citet{Weinberg2017}, we take a~$t^{-1.1}$ power-law DTD with a
	minimum delay-time of~$t_\text{D} = 150$ Myr.

	\item In general, the mass of some element x in the ISM is also affected by
	outflows, recycling, star formation, and infall.
	The enrichment rate can be calculated by simply adding up all of the source
	terms and subtracting the sink terms:
	\begin{equation}
	\label{eq:enrichment_eq}
	\dot{M}_\text{x} = \dot{M}_\text{x}^\text{CC} + \dot{M}_\text{x}^\text{Ia}
	- Z_\text{x}\dot{M}_\star - Z_\text{x}\dot{M}_\text{out} +
	\dot{M}_\text{x,r},
	\end{equation}
	where the rate of return of the element x from recycled stellar envelopes
	can be computed by weighting the integral in equation~\refp{eq:mdot_recycled}
	by~$Z_\text{x}(t)$.
	If there is metal-rich infall, this equation picks up the additional term
	$Z_\text{x,in}\dot{M}_\text{in}$ quantifying that, although here we assume
	that infall is pristine.

\end{itemize}

\subsection{Nucleosynthetic Yields}
\label{sec:onezone:yields}

\begin{itemize}

	\item In general, nucleosynthetic yields are degenerate with the outflow
	mass loading factor~$\eta$.
	We quantify this is in more detail in Appendix X, simply noting there that
	the two are simply the dominant source and sink terms, and as such,
	high-yield high-outflow models generally have a low-yield low-outflow
	counterpart that predicts a similar chemical evolution.
	In order to break this degeneracy, only one number setting the absolute
	scale is required.
	Here, we simply set the alpha element yield to~\yacc~= 0.01.
	This value is somewhat informed by nucleosynthesis theory in that
	massive star evolutionary models (e.g.~\citealp{Sukhbold2016,
	Limongi2018};~\citealp*{Nomoto2013}) typically predict
	$y_\text{O}^\text{CC} = 0.005 - 0.015$ (see discussion in, e.g.,
	\citealp{Weinberg2017, Johnson2020}), but is otherwise intended to be a
	round number from which our best-fit values affected by this degeneracy can
	simply be scaled up or down.

	\item We let our Fe yields~\yfecc~and~\yfeia~be free parameters.
	With this approach, we implicitly fit the height of the [$\alpha$/Fe]
	plateau as well as the Fe yield ratio~\yfecc/\yfeia.

	\item In general, nucleosynthetic material is also expelled by asymptotic
	giant branch (AGB) stars~\citep[e.g.][]{Cristallo2011, Cristallo2015,
	Ventura2013, Karakas2016, Karakas2018}.
	Here we are interested primarily in~$\alpha$ and Fe-peak elements, elements
	whose AGB star yields are negligible compared to their SN yields
	\citep[e.g.][]{Johnson2019}.
	We therefore omit discussion of AGB star nucleosynthesis here, but we note
	that our fitting method described in~\S~\ref{sec:fitting} is easily
	extensible to include an AGB star enrichment channel.
	Mathematical details of how this is implemented in~\vice~can be found in
	\citet{Johnson2020},~\citet{Johnson2022}, and in the~\vice~science
	documentation.\footnote{
		\url{https://vice-astro.readthedocs.io/en/latest/science_documentation/index.html}
	}

\end{itemize}


\section{The Fitting Method}
\label{sec:fitting}

\begin{itemize}

	\item Here we provide an overview of our method for fitting one-zone GCE
	models to data.
	Our method uses the observed abundances and ages (where available) of an
	ensemble of stars and, with no binning of the data, accurately constructs
	the~\textit{likelihood function}~$L(\script{D} | \{\theta\})$ describing
	the probability of observing the data~\script{D} given a set of model
	parameters~$\{\theta\}$.
	This is related to the~\textit{posterior probability}~$L(\{\theta\} |
	\script{D})$ according to Bayes' Theorem:
	\begin{equation}
	L(\{\theta\} | \script{D}) = \frac{
		L(\script{D} | \{\theta\}) L(\{\theta\})
	}{
		L(\script{D}),
	}
	\end{equation}
	where~$L(\{\theta\})$ is the likelihood of the parameters themselves
	(known as the~\textit{prior}) and~$L(\script{D})$ is the likelihood of the
	data (known as the~\textit{evidence}).
	Although it is more desirable to measure the posterior probability,
	in practice only the likelihood function can be robustly determined
	because the prior is not directly quantifiable; it requires information
	independent of the data on how likely the chosen parameters~$\{\theta\}$
	are to be accurate.
	With no prior information on what the parameters should be, the best
	practice is to assume a ``flat'' or ``uniform'' prior in which
	$L(\{\theta\})$ is a constant, and therefore
	$L(\{\theta\} | \script{D}) \approx L(\script{D} | \{\theta\})$; we retain
	this convention here.

	\item Our method treats the sampling of stars from a one-zone GCE model
	as an~\textit{inhomogeneous poisson point process} (IPPP).
	In Appendix~\ref{sec:l_derivation}, we present a detailed derivation of
	our likelihood function in which we apply the principles of an IPPP and
	statistical likelihood to one-zone GCE models.
	Owing to its detailed nature, we provide only qualitative discussion of
	its form here, reserving more in-depth justification for Appendix
	\ref{sec:l_derivation}.
	Though our use case in this paper is in the context of GCE, this method
	is highly generic and should be extensible to other astrophysical models
	which produce evolutionary tracks in some observed space.
	Such models arise also in the context of stellar streams and
	color-magnitude diagrams (i.e. stellar isochrones).

	\item In practice, one-zone GCE models do not predict a smooth, functional
	form for the evolutionary track through abundance space.
	Instead, it is most often quantified in some piece-wise linear form
	predicted by some numerical code.
	For a sample~\script{D} with~$N$ individual stars and a predicted track
	\script{M} sampled at~$K$ points in the observed space, the likelihood
	function is given by
	\begin{equation}
	\ln L(\script{D} | \{\theta\}) = \sum_i^N \ln \left(
	\sum_j^K w_j \exp \left(
	\frac{-1}{2} \delta_{ij} C_i^{-1} \delta_{ij}^T
	\right)
	\right),
	\label{eq:likelihood}
	\end{equation}
	where~$\delta_{ij} = \script{D}_i - \script{M}_j$ is the vector difference
	between the~$i$th datum and the~$j$th point on the predicted track,
	$C_i^{-1}$ is the inverse covariance matrix of the~$i$th datum, and~$w_j$
	is a weight to be attached to~$\script{M}_j$ which scales with the
	SFR and the selection function of the survey, normalized such that
	$\sum_j w_j = 1$ (see discussion below).

	\item This functional form arises from marginalizing the likelihood of
	observing each datum over the entire evolutionary track~\script{M}
	according to:
	\begin{equation}
	\ln L(\script{D} | \{\theta\}) = \sum_i^K \ln \left(
	\sum_j^K L\left(\script{D}_i | \script{M}_j
	\right)\right),
	\end{equation}
	where the likelihood of observing the~$i$th datum given the~$j$th point on
	the evolutionary track is given by a weighted~$e^{-\chi^2/2}$
	expression.
	Mathematically, the requirement for this marginalization arises naturally
	from the application of statistical likelihood and the IPPP to an
	evolutionary track (see Appendix~\ref{sec:l_derivation}).
	Qualitatively, it can be understood from the notion that there is no way
	of knowing which point on the evolutionary track the datum~$\script{D}_i$
	is truly associated with, and the only way to properly take this into
	account is to consider all pair-wise combinations of~\script{D}
	and~\script{M}.

	\item The requirement for a weighted as opposed to
	unweighted~$e^{-\chi^2/2}$ likelihood expression also arises naturally out
	of the application of statistical likelihood and the IPPP to an
	evolutionary track.
	While the requirement for marginalization arises because there is no way of
	knowing which point on the evolutionary track a datum~$\script{D}_i$ arose
	from, the requirement for weights arises because it is proportionally more
	likely to be associated with points on the track at which either the SFR is
	high or the survey selection function is deeper.
	For a survey selection function~\script{S} and SFR~$\dot{M}_\star$, the
	weights should scale as their product:
	\begin{equation}
	w_j \propto \script{S}(\script{M}_j, \{\theta\}) \dot{M}_\star(t_j |
	\{\theta\}),
	\label{eq:weights}
	\end{equation}
	where~$t_j$ denotes the time in the GCE model under the selection of
	parameters~$\{\theta\}$.
	Because we have parameterized our GCE models in a manner such that the
	normalization of the SFH is inconsequential to the abundance evolution,
	the weights must be normalized such that they add up to 1.
	For parametrizations in which the normalization does impact the abundance
	evolution, the modification to equation~\refp{eq:likelihood} is simple
	(see discussion below).

	\item The validity of equation~\refp{eq:likelihood} is contingent on the
	following assumptions.
	\begin{itemize}
		\item \textit{The track is infinitely thin.}
		In the absence of measurement errors, all of the data would fall
		perfectly on a line in the observed space.
		As discussed at the beginning of~\S~\ref{sec:onezone}, the
		fundamental assumption of one-zone GCE models is instantaneous
		diffusion and, consequently, chemical homogeneity.
		They sacrifice this spatial information in exchange for a drastic
		reduction in computational expense.
		By construction, they predict a single exact abundance of all nuclear
		species in the star formation reservoir at any given time.
		If the model in question instead predicts a track with some finite
		width, then computing the likelihood function is a fundamentally
		different problem.

		\item \textit{Each observation is independent.}
		When this condition is met, the total likelihood of observing the
		data~\script{D} can be expressed as the product of the likelihood of
		observing each individual datum:
		\begin{subequations}\begin{align}
		L(\script{D} | \{\theta\}) &= \prod_i^N L(\script{D}_i | \script{M})
		\\
		\implies \ln L(\script{D} | \{\theta\}) &= \sum_i^N \ln
		L(\script{D}_i | \{\theta\}).
		\end{align}\end{subequations}
		This condition plays an important role in giving rise to the
		functional form of equation~\refp{eq:likelihood}.

		\item \textit{The observational uncertainties are described by a
		multivariate Gaussian.}
		If this condition fails, the expression for~$\chi^2 =
		\delta_{ij}C_i^{-1}\delta_{ij}^T$ is no longer accurate.
		In this case, the expression~$L(\script{D}_i | \script{M}_j) \propto
		w_j e^{-\chi^2/2}$ should be replaced with some kernel density
		estimate of the uncertainty at the point~$\script{M}_j$, but retaining
		the weight~$w_j$.

		\item \textit{The track is densely sampled.}
		That is, the spacing between the points on track~\script{M} is small
		compared to the observational uncertainties in the data.
		This assumption can be relaxed at the expense of including an
		additional correction factor~$\beta_{ij}$ given by equation
		\refp{eq:corrective_beta} which integrates the likelihood between each
		pair of adjacent points~$\script{M}_j$ and~$\script{M}_{j + 1}$ along
		the track.
		If computing the evolutionary track is sufficiently expensive,
		relaxing the number of points and including this correction factor may
		be the more computationally efficient option.

		\item \textit{The normalization of the SFH is inconsequential.}
		That is, only the time-dependence of the SFH impacts the abundance
		evolution predicted by the GCE model.
		We have parametrized our models in a manner such that the normalization
		of the SFH is inconsequential to the evolution in the abundances (for
		details, see discussion in~\S~\ref{sec:onezone}).
		Because the SFH and the selection function of the survey determine the
		proper weights to attach to the likelihood of each pair-wise
		combination between the data and the track (see
		equation~\ref{eq:weights}), it is essential that the normalization of
		the weights not impact the inferred likelihood.
		In this case, we find in our derivation of the likelihood function that
		the proper manner in which to handle the weights is to normalize them
		such that they add up to 1 (see Appendix~\ref{sec:l_derivation}).
		If the GCE model is instead parametrized such that the normalization of
		the SFH~\textit{does} impact the abundance evolution, then the weights
		must remain un-normalized, and their sum must be subtracted from
		equation~\refp{eq:likelihood}.
		The requirement that their sum be subtracted from the inferred
		likelihood can be qualitatively understood as a penalty for models
		which predict data in regions of the observed space where there is
		none.
		It is a term which encourages parsimony, rewarding parameter choices
		which explain the data in as few predicted instances as possible.
		This penalty is still included in models which normalize the weights;
		in these cases, tracks which extend too far in the observed space
		have a higher~\textit{fractional} weight from data at large~$\chi^2$,
		lowering the total likelihood.
	\end{itemize}

	\item 
	We demonstrate the accuracy of equation~\refp{eq:likelihood}
	in~\S~\ref{sec:mocks} below by means of tests against mock data samples.
	Although our likelihood function does not include a direct fit to
	the stellar distributions in age and abundances, weighting the inferred
	likelihood by the SFR in the model indeed incorporates this information on
	how many stars should form at which ages and abundances.
	With this~\textit{implicit} fit, our method accurately characterizes the
	age and abundance distributions of our mock samples even though they are
	not explicitly included in the likelihood calculation.

	\item There are a variety of ways in which one could measure the likelihood
	function~$L(\script{D} | \{\theta\})$ given by equation
	\refp{eq:likelihood}, and in the present paper we use the Markov Chain
	Monte Carlo (MCMC) method.
	Despite being more computationally expensive than, e.g., maximum a posterori
	(MAP) estimation, MCMC offers a more generic solution than other options by
	sampling tails and multiple modes of the likelihood distribution that could
	otherwise be missed by assuming Gaussianity.
	Our method should nonetheless be extensible to additional data sets
	described by GCE models with different parametrizations as well as
	different methods of optimizing the likelihood function, such as MAP
	estimates.

	\item We make use of the~\mc~\python~package~\citep{Foreman-Mackey2013} to
	construct our markov chains, additionally using the~\vice~GCE software
	\citep{Johnson2020} to compute the predicted abundances for a given choice
	of parameters~$\{\theta\}$ provided by~\mc.
	% The log of the likelihood function~$\ln L(\script{D} | \{\theta\})$, as
	% required by~\mc~to take steps in parameter space, is then given by
	% equation~\refp{eq:likelihood}.

\end{itemize}

\end{document}
