
\documentclass[ms.tex]{subfiles}
\begin{document}

% \section{Methods}
% \label{sec:methods}

% \begin{itemize}

% 	\item We are interested in applying one-zone GCE models to dwarf galaxies
% 	and determining best-fit parameters.
% 	We begin by providing background on one-zone models, and then we select
% 	a parametrization from which we draw a fiducial mock stellar sample.
% 	We then use these data to introduce our fitting method.

% \end{itemize}

\section{Galactic Chemical Evolution}
\label{sec:onezone}

\begin{itemize}

	\item The fundamental assumption of one-zone models is that newly produced
	metals mix instantaneously throughout the star forming gas reservoir.
	In detail, this assumption is valid as long as the mixing time-scale
	is short compared to the depletion time-scale (i.e. the average time a
	fluid element remains in the ISM before getting incorporated into new
	stars or ejected in an outflow, if present).
	Based on the observations of~\citet{Leroy2008},~\citet*{Weinberg2017}
	calculate that characteristic depletion times can range from~$\sim$500 Myr
	up to~$\sim$10 Gyr for conditions in typical star forming disc galaxies.
	In the dwarf galaxy regime, the length scales are short, star formation
	is slow~\citep[e.g.][]{Hudson2015}, and the ISM velocities are turbulent
	\citep{Dutta2009, Stilp2013, Schleicher2016}.
	With this combination, instantaneous mixing should be a good approximation,
	though we are unaware of any studies which address this observationally.
	If this approximation is valid, then there should exist some evolutionary
	track in chemical space (e.g. the~\afe-\feh~plane) about which the
	intrinsic scatter is negligible compared to the measurement uncertainty.
	This empirical test should be feasible on a galaxy-by-galaxy basis.

	\item With the goal of assessing the information content of one-zone GCE
	models applied to dwarf galaxies, we emphasize that the accuracy of the
	methods we outline in this paper are contingent on the validity of the
	instantaneous mixing approximation.
	This assumption reduces GCE to a system of coupled integro-differential
	equations which we solve using the publicly available
	\textsc{Versatile Integrator for Chemical Evolution} (\vice\footnote{
		\url{https://pypi.org/project/vice}
	};~\citealp{Johnson2020}).
	We provide an overview of these equations below and refer to
	\citet{Johnson2020} and the~\vice~science documentation\footnote{
		\url{https://vice-astro.readthedocs.io/en/latest/science_documentation}
	} for futher details.

	% \item In larger systems such as the Milky Way or Andromeda, the
	% instantaneous mixing approximation breaks down.
	% Observationally, the presence of gas-phase abundance gradients in
	% star-forming disk galaxies~\citep*[see, e.g., recent reviews
	% by][]{Kewley2019, Maiolino2019, Sanchez2020} indicates that metal diffusion
	% in the radial direction is inefficient.
	% Furthermore, processes such as radial gas flows~\citep{Lacey1985,
	% Bilitewski2012, Vincenzo2020, Sharda2021} and the radial migration of stars
	% (\citealp{Sellwood2002, Roskar2008a, Roskar2008b, Loebman2011, Minchev2011};
	% \citealp*{Bird2012};~\citealp{Bird2013};~\citealp*{Grand2012a, Grand2012b,
	% Kubryk2013};~\citealp{Okalidis2022}) are expected to significantly
	% impact chemical enrichment and the observed abundance distributions in a
	% given galactic region.
	% This has prompted the development of a number of so-called ``multi-zone''
	% models (\citealp*{Minchev2013, Minchev2014};~\citealp{Minchev2017,
	% Johnson2021, Chen2022}) where the core motivation is to add some of the
	% spatial information lost by one-zone models back into the framework by
	% allowing the exchange of gas and stars between zones.

	% \item With the goal of assessing the information content of stellar
	% abundances in dwarf galaxies where the instantaneous mixing approximation
	% is more likely to be valid, we do not explore multi-zone models here.
	% The increased computational expense of multi-zone models also makes them
	% less conducive to best-fit parameter determination, which is of central
	% interest to this paper.
	% When applied to any given galaxy, we emphasize that the accuracy of the
	% methods we outline in this paper are contingent on the validity of the
	% instantaneous mixing approximation.
	% This assumption reduces GCE to a system of coupled integro-differential
	% equations which we discuss below.
	% Throughout this paper, we solve these equations using the publicly
	% available~\textsc{Versatile Integrator for Chemical Evolution}
	% (\vice\footnote{
	% 	\url{https://pypi.org/project/vice}
	% };~\citealp{Johnson2020}).

% \end{itemize}

% \subsection{Inflows, Outflows, Star Formation, and Recycling}
% \label{sec:onezone:gas}

% \begin{itemize}

	\item At a given moment in time, gas is added to the interstellar medium
	(ISM) via inflows and recycled stellar envelopes and is removed from the
	ISM by star formation and outflows, if present.
	This gives rise to the following differential equation describing the
	evolution of the gas-supply:
	\begin{equation}
	\label{eq:mdot_gas}
	\dot{M}_\text{g} = \dot{M}_\text{in} - \dot{M}_\star - \dot{M}_\text{out}
	+ \dot{M}_\text{r},
	\end{equation}
	where~$\dot{M}_\text{in}$ is the infall rate,~$\dot{M}_\star$ is the star
	formation rate (SFR),~$\dot{M}_\text{out}$ is the outflow rate,
	and~$\dot{M}_\text{r}$ describes the return of stellar envelopes from
	previous generations of stars.

	\item Here we retain the characterization of outflows implemented in the
	publicly available GCE codes~\textsc{OMEGA}~\citep{Cote2017},
	\textsc{flexCE}~\citep{Andrews2017}, and~\vice~\citep{Johnson2020}, in
	which a ``mass loading factor''~$\eta$ describes a linear relationship
	between the outflow rate itself and the SFR:
	\begin{equation}
	\eta \equiv \frac{\dot{M}_\text{out}}{\dot{M}_\star}.
	\label{eq:mass_loading}
	\end{equation}
	This parametrization is appropriate for models in which massive stars are
	the dominant source of energy for outflow-driving winds.
	Empirically, the strength of outflows (i.e. the value of~$\eta$) is
	strongly degenerate with the absolute scale of nucleosynthetic yields.
	This ``yield-outflow degeneracy'' arises because the yields themselves are
	the primary source term in describing the enrichment rates in galaxies
	while outflows are the primary sink term.
	We quantify the strength of this degeneracy in more detail in Appendix
	\ref{sec:yield_outflow_degeneracy}.
	We discuss our adopted nucleosynthetic yields in~\S~\ref{sec:onezone:yields}
	below, which are intended to address this degeneracy at least in part.

	\item The SFR and the mass of the ISM are related by the timescale
	$\tau_\star$, defined as the ratio of the two:
	\begin{equation}
	\tau_\star \equiv \frac{M_\text{g}}{\dot{M}_\star}.
	\label{eq:tau_star_def}
	\end{equation}
	The inverse of this quantity~$\tau_\star^{-1}$ is typically referred to as
	the ``star formation efficiency''~\citep[e.g.][]{Schaefer2020} because it
	quantifies the~\textit{fractional} rate at which some ISM fluid element
	is forming stars.
	Some authors, however, refer to this parameter as the ``depletion time''
	\citep[e.g.][]{Tacconi2018}, because it also describes the e-folding
	decay timescale of the ISM mass due to star formation if no additional gas
	is added.
	Following~\citet{Weinberg2017}, we hereafter refer to~$\tau_\star$ as the
	SFE timescale, because depletion timescales in GCE models can shorten
	considerably due to outflows (specifically,
	$\tau_\text{dep} = \tau_\star / (1 + \eta - r)$ where~$r$ is a corrective
	term for recycled stellar envelopes; see discussion below).

	% \item The recycling\footnote{
	% 	Here, recycling refers only to ejected stellar envelopes returning
	% 	baryons to the ISM.
	% 	We do not implement the ``instantaneous recycling approximation'' as
	% 	originally formulated, which assumes that stellar populations eject
	% 	their entire nucleosynthetic yields instantaneously
	% 	\citep[see, e.g., the review in][]{Tinsley1980}.
	% } rate~$\dot{M}_\text{r}$ can be expressed as an integral over the SFH
	% according to
	% \begin{equation}
	% \dot{M}_\text{r} = \int_0^T \dot{M}_\star(t)\dot{r}(T - t)dt,
	% \label{eq:mdot_recycled}
	% \end{equation}
	% where~$T$ is the current time in a GCE model and~$r(\tau)$ is the
	% ``cumulative return fraction,'' which describes the fraction of a single
	% stellar population's mass that has been returned to the ISM via ejected
	% stellar envelopes at an age of~$\tau$.
	% In detail,~$r$ is a complicated function which depends on the stellar
	% IMF~\citep[e.g.][]{Salpeter1955, Miller1979, Kroupa2001, Chabrier2003},
	% the initial-final remnant mass relation~\citep[e.g.][]{Kalirai2008},
	% and the mass-lifetime relation\footnote{
	% 	We assume a~\citet{Kroupa2001} IMF and the~\citet{Larson1974}
	% 	mass-lifetime relation throughout this paper.
	% 	These parameters, however, do not significantly impact our conclusions
	% 	because to first-order the enrichment history of our models is
	% 	set by the parameters~$\tau_\star$ and~$\eta$.
	% 	Our fitting method (see discussion in~\S~\ref{sec:fitting}) is
	% 	however easily extensible to models which relax these assumptions.
	% } (e.g.~\citealp{Larson1974, Maeder1989};~\citealp*{Hurley2000}).
	% We provide only qualitative discussion here; further details can be found
	% in~\citet{Weinberg2017} and in the VICE science documentation.\footnote{
	% 	\url{https://vice-astro.readthedocs.io/en/latest/science_documentation}
	% }
	% The recycling rate is initially high but declines rapidly as stellar
	% populations age due to the steep nature of the mass-lifetime relation.
	% \citet{Weinberg2017} demonstrate that it is therefore sufficiently accurate
	% in one-zone models to assume that some fraction~$r_\text{inst}$ of a
	% stellar population's initial mass is returned to the ISM instantaneously
	% (see their Fig. 7; they recommend~$r_\text{inst} = 0.4$ for a
	% \citealt{Kroupa2001} IMF, and~$r_\text{inst} = 0.2$ for
	% a~\citealt{Salpeter1955} IMF).
	% Although it is simpler to assume~$\dot{M}_\text{r} =
	% r_\text{inst}\dot{M}_\star$, this numerical integration is computationally
	% cheap and is already implemented in~\vice.

	\item The recycling rate is a complicated function which depends on the
	stellar IMF~\citep[e.g.][]{Salpeter1955, Miller1979, Kroupa2001,
	Chabrier2003}, the initial-final remnant mass relation
	\citep[e.g.][]{Kalirai2008}, and the mass-lifetime relation\footnote{
		We assume a~\citet{Kroupa2001} IMF and the~\citet{Larson1974}
		mass-lifetime relation throughout this paper.
		These choices, however, do not significantly impact our conclusions,
		and our fitting method is easily extensible to modeils which relax
		these assumptions.
	} (e.g.~\citealp{Larson1974, Maeder1989};~\citealp*{Hurley2000}).
	However, the detailed recycling rate has only a second-order effect on
	the gas-phase evolutionary track in the~\afe-\feh~plane.
	The first-order details are instead determined by the SFE timescale
	$\tau_\star$ and the mass-loading factor~$\eta$~\citep{Weinberg2017}.
	With low~$\tau_\star$ (i.e. high SFE), nucleosynthesis is fast because
	star formation is fast, and a higher metallicity can be obtained before
	the onset of SN Ia than in lower SFE models.
	For this reason,~$\tau_\star$ plays the dominant role in shaping the
	position of the knee in the~\afe-\feh~plane.
	As the galaxy evolves, it approaches a chemical equilibrium in which
	newly produced metals are balanced by the loss of metals to outflows
	and new stars.
	Controlling the strength of the sink term of outflows,~$\eta$ plays
	the dominant role in shaping the late-time equilibrium abundance of the
	model, with high outflow models (i.e. high~$\eta$) predicting lower
	equilibrium abundances than their weak outflow counterparts.
	For observed data, the shape of the track itself directly constrains
	these parameters (see discussion in~\S~\ref{sec:mocks:fiducial} below).
	The detailed form of the SFH has minimal impact on the shape of the
	tracks, provided that there are no sudden events such as a burst of
	star formation~\citep{Weinberg2017, Johnson2020}.
	Instead, that information is encoded in the stellar metallicity
	distribution functions (MDFs; i.e. the density of stars along the
	evolutionary track).
	% Instead, that information is encoded in the density of stars along the
	% evolutionary track and in the stellar metallicity distribution functions
	% (MDFs).

	% \end{itemize}

% \end{itemize}

% \subsection{Enrichment}
% \label{sec:onezone:enrichment}

% \subsection{Core Collapse Supernovae}
% \label{sec:onezone:ccsne}

% \begin{itemize}

	% \item In the present paper, we focus on th eenrichment of the so-called
	% ``alpha'' (e.g. O, Ne, Mg, Si) and ``iron-peak'' (e.g. Cr, Fe, Ni, Zn)
	% elements, with the distribution of stars in the~\afe-\feh~plane being our
	% primary observational diagnostic to distinguish between GCE models.
	% Alpha elements are so-named because they are produced by alpha capture
	% reactions in massive stars, and for the lighter ones like O and Mg, this
	% is the only dominant enrichment source (see, e.g., the review in
	% \citealp{Johnson2019}).
	% Iron-peak elements are also produced in massive stars, but also owe a
	% portion of their abundance to white dwarf SNe occurring on longer delay
	% times.
	% Heavier nuclei (specifically Sr and up) are also produced by neutron
	% capture processes.
	% Although we do not treat these elements here, our fitting method (see
	% discussion in~\S~\ref{sec:fitting}) is easily extensible to GCE models
	% which do, provided that the data contain such measurements.

	\item In the present paper, we focus on the enrichment of the so-called
	``alpha'' (e.g. O, Ne, Mg) and ``iron-peak'' elements (e.g. Cr, Fe,
	Ni, Zn), with the distribution of stars in the~\afe-\feh~plane being our
	primary observational diagnostic to distinguish between GCE models.
	Massive stars and their SNe are the dominant enrichment source of alpha
	elements in the universe, while iron-peak elements are produced in
	significant amounts by massive stars and SNe Ia~\citep[e.g.][]{Johnson2019}.
	In detail, alpha and iron-peak elements are also produced by slow neutron
	capture nucleosynthesis, an enrichment pathway responsible for much of the
	abundances of yet heavier nuclei (specifically Sr and up).
	Because the neutron capture yields of alpha and iron-peak elements are
	negligible compared to their SN yields, we do not discuss this process
	further.
	Our fitting method is nonetheless easily extensible to GCE models which do,
	provided that the data contain such measurements.

	\item Due to the steep nature of the stellar mass-lifetime relation
	\citep[e.g.][]{Larson1974, Maeder1989, Hurley2000}, massive stars, their
	winds, and their supernovae enrich the ISM on~$\sim$few Myr timescales.
	As long as this is shorter than the relevant timescales for a given
	galaxy's evolution and the present-day stellar mass is sufficiently high
	such that stochastic sampling of the IMF does not significantly impact the
	yields, then it is adequate to approximate this nucleosynthetic material as
	some average yield ejected instantaneously following a single stellar
	population's formation.
	This implies a linear relationship between the CCSN enrichment rate and
	the SFR:
	\begin{equation}
	\label{eq:mdot_cc}
	\dot{M}_\text{x}^\text{CC} = y_\text{x}^\text{CC}\dot{M}_\star
	\end{equation}
	where~$y_\text{x}^\text{CC}$ is the IMF-averaged fractional net yield from
	massive stars.
	That is, for a fiducial value of~$y_\text{x}^\text{CC} = 0.01$, 100~\msun~of
	star formation would produce 1~\msun~of~\textit{newly produced} mass of
	element x (we implement the return of previously produced metals separately;
	see~\citealt{Johnson2020} or the~\vice~science documentation for details).

% \subsection{Type Ia Supernovae}
% \label{sec:onezone:sneia}

	\item Unlike CCSNe, SNe Ia occur on a significantly extended delay time
	distribution (DTD).
	The details of the DTD are a topic of active inquiry
	\citep[e.g.][]{Greggio2005, Strolger2020, Freundlich2021}, and at least a
	portion of the uncertainty can be traced to uncertainties in both galactic
	and cosmic SFHs.
	Comparisons of the cosmic SFH~\citep[e.g.][]{Hopkins2006, Madau2014,
	Davies2016, Madau2017, Driver2018} with
	volumetric SN Ia rates as a function of redshift indicate that the cosmic
	DTD is broadly consistent with a uniform~$\tau^{-1}$ power-law (e.g.
	\citealp*{Maoz2012a, Maoz2012b, Graur2013};~\citealp{Graur2014}).
	Following~\citet{Weinberg2017}, we take a~$\tau^{-1.1}$ power-law DTD with a
	minimum delay-time of~$t_\text{D} = 150$ Myr, though in principle this
	delay-time could be as short as~$t_\text{D} \approx 40$ Myr due to the
	lifetimes of the most massive white dwarf progenitors.
	For any selected DTD~$R_\text{Ia}(\tau)$, the SN Ia enrichment rate can be
	expressed an integral over the SFH weighted by the DTD:
	\begin{equation}
	\dot{M}_\text{x}^\text{Ia} = y_\text{x}^\text{Ia} \ddfrac{
		\int_0^{T - t_\text{D}} \dot{M}_\star(t) R_\text{Ia}(T - t) dt
	}{
		\int_0^\infty R_\text{Ia}(t) dt
	}.
	\end{equation}

	\item In general, the mass of some element x in the ISM is also affected by
	outflows, recycling, star formation, and infall.
	The enrichment rate can be calculated by simply adding up all of the source
	terms and subtracting the sink terms:
	\begin{equation}
	\label{eq:enrichment_eq}
	\dot{M}_\text{x} = \dot{M}_\text{x}^\text{CC} + \dot{M}_\text{x}^\text{Ia}
	- Z_\text{x}\dot{M}_\star - Z_\text{x}\dot{M}_\text{out} +
	\dot{M}_\text{x,r}.
	\end{equation}
	If there is metal-rich infall, this equation picks up the additional term
	$Z_\text{x,in}\dot{M}_\text{in}$ quantifying that, although here we assume
	that infall is pristine.
	% If additional enrichment sources such as slow neutron capture in asymptotic
	% giant branch stars~\citep[e.g.][]{Cristallo2011, Cristallo2015, Ventura2013,
	% Ventura2014, Ventura2018, Ventura2020, Karakas2016, Karakas2018} are to be
	% included, then they contribute additional source terms to
	% equation~\refp{eq:enrichment_eq} as well.
	% Since we focus this paper on alpha and iron-peak elements whose yields are
	% dominated by SNe~\citep[e.g.][]{Johnson2019}, we do not discuss these
	% processes here.

% \end{itemize}

% \subsection{Nucleosynthetic Yields}
% \label{sec:onezone:yields}

% \begin{itemize}

	\item Empirically, the absolute scale of nucleosynthetic yields is
	degenerate with the strength of outflows (i.e. the value of the
	mass-loading factor~$\eta$).
	This degeneracy is remarkably strong, arising because yields and outflows
	are the dominant source and sink terms in equation~\refp{eq:enrichment_eq}
	above.
	As a consequence, high-yield and high-outflow outflow models generally have
	a low-yield and low-outflow counterpart that predicts a similar enrichment
	history.
	Some authors~\citep[e.g.][]{Minchev2013, Minchev2014, Minchev2017,
	Spitoni2020, Spitoni2021} even assume that outflows do not sweep up ambient
	ISM at all (i.e.~$\eta = 0$) , while others~\citep[e.g.][]{Andrews2017,
	Weinberg2017, Cote2017, Trueman2022} instead argue that this is an
	important ingredient in GCE models.
	In order to break this degeneracy, only a single number setting the
	absolute scale is required.
	Throughout this paper (with the exception of Appendix
	\ref{sec:yield_outflow_degeneracy} where the lack of an absolute scale is
	directly relevant), we set the alpha element yield from massive stars to
	be exactly~$\yacc = 0.01$ and let our Fe yields be free parameters.
	This value is somewhat informed by nucleosynthesis theory in that massive
	star evolutionary models (e.g.~\citealp*{Nomoto2013};~\citealp{Sukhbold2016,
	Limongi2018}) typically predict~$y_\text{O}^\text{CC} = 0.005 - 0.015$ (see
	discussion in, e.g.,~\citealp{Weinberg2017} and~\citealp{Johnson2020}).
	The primary motivation behind this choice, however, is to select a round
	number from which our best-fit values affected by this degeneracy can
	simply be scaled up or down to accommodate alternate parameter choices.
	In models which assume that ambient ISM is not ejected in
	outflows,~\yacc~can be included as a free parameter and the overall scale
	is instead set by the assumption that~$\eta = 0$.
	However, the arguments supporting this assumption have generally been made
	in the context of the Milky Way, while ejecting ambient ISM in an outflow
	should in principle be easier in a dwarf galaxy where the gravity well
	is intrinsically shallower, potentially implying~$\eta > 0$.
	We reserve further discussion of this uncertainty in GCE models and the
	various prescriptions of outflows in GCE models in the literature for
	Appendix~\ref{sec:yield_outflow_degeneracy}, where we additionally quantify
	its considerable strength in more detail.

\end{itemize}


\section{The Fitting Method}
\label{sec:fitting}

\begin{itemize}

	% \item Here we provide an overview of our method for fitting one-zone GCE
	% models to data.
	\item Our fitting method uses the abundances and ages (where available) of
	an ensemble of stars and, with no binning of the data, accurately
	constructs the~\textit{likelihood function}~$L(\script{D} | \{\theta\})$
	describing the probability of observing the data~\script{D} given a set of
	model parameters~$\{\theta\}$.
	This is related to the~\textit{posterior probability}~$L(\{\theta\} |
	\script{D})$ according to Bayes' Theorem:
	\begin{equation}
	L(\{\theta\} | \script{D}) = \frac{
		L(\script{D} | \{\theta\}) L(\{\theta\})
	}{
		L(\script{D})
	},
	\end{equation}
	where~$L(\{\theta\})$ is the likelihood of the parameters themselves
	(known as the~\textit{prior}) and~$L(\script{D})$ is the likelihood of the
	data (known as the~\textit{evidence}).
	Although it is more desirable to measure the posterior probability,
	in practice only the likelihood function can be robustly determined
	because the prior is not directly quantifiable; it requires quantitative
	information independent of the data on the accuracy of a chosen set of
	parameters.
	With no additional information on what the parameters should be, the best
	practice is to assume a ``flat'' or ``uniform'' prior in which
	$L(\{\theta\})$ is a constant, and therefore
	$L(\{\theta\} | \script{D}) \approx L(\script{D} | \{\theta\})$; we retain
	this convention here.

	\item As mentioned in~\S~\ref{sec:intro}, the sampling of stars from an
	underlying distribution in abundance space proceeds according to an IPPP
	\citep[e.g.][]{Press2007}.
	Due to its detailed nature, we reserve a full derivation of our likelihood
	function for Appendix~\ref{sec:l_derivation} and provide qualitative
	discussion of its form here.
	Though our use case in the present paper is in the context of one-zone GCE
	models, our derivation assumes only that the chief prediction of the model
	is a track of some arbitrary form in the observed space.
	It is therefore highly generic and should be easily extensible to other
	astrophysical models which predict tracks of some form (e.g. stellar
	streams in kinematic space and stellar isochrones on CMDs).

	\item In practice, the evolutionary track predicted by a one-zone GCE model
	is generally not known in some analytic functional form (unless some
	approximations are made as in, e.g.,~\citealp{Weinberg2017}).
	Instead, it is most often quantified in a piece-wise linear form predicted
	by some numerical code (in our case,~\vice).
	For a sample~$\script{D} = \{\script{D}_1, \script{D}_2, \script{D}_3,
	..., \script{D}_N\}$ containing~$N$ abundance and (where available) age
	measurements of individual stars and a track~$\script{M} = \{\script{M}_1,
	\script{M}_2, \script{M}_3, ..., \script{M}_K\}$ sampled at~$K$ points in
	the observed space, the likelihood function is given by
	\begin{equation}
	\ln L(\script{D} | \{\theta\}) = \sum_i^N \ln \left(
	\sum_j^K w_j \exp \left(
	\frac{-1}{2} \Delta_{ij} C_i^{-1} \Delta_{ij}^T
	\right)
	\right),
	\label{eq:likelihood}
	\end{equation}
	where~$\Delta_{ij} = \script{D}_i - \script{M}_j$ is the vector difference
	between the~$i$th datum and the~$j$th point on the predicted track,
	$C_i^{-1}$ is the inverse covariance matrix of the~$i$th datum, and~$w_j$
	is a weight to be attached to~$\script{M}_j$.
	This functional form is appropriate for GCE models in which the
	normalization of the SFH is inconsequential to the evolution of the
	abundances; in the opposing case where the normalization does impact the
	predicted abundances evolution, one additional term subtracting the sum of
	the weights is required (see discussion below).

	\item Equation~\refp{eq:likelihood} arises from marginalizing the
	likelihood of observing each datum over the entire evolutionary track and
	has the more general form of
	\begin{subequations}\begin{align}
	\ln L(\script{D} | \{\theta\}) &= \sum_i^N \ln \left(
	\int_\script{M} L(\script{D}_i | \script{M}) d\script{M}
	\right)
	\label{eq:likelihood_general_integral}
	\\
	&\approx \sum_i^N \ln \left(
	\sum_j^K L\left(\script{D}_i | \script{M}_j
	\right)\right).
	\label{eq:likelihood_general}
	\end{align}\end{subequations}
	Equation~\refp{eq:likelihood_general} follows from equation
	\refp{eq:likelihood_general_integral} when the track is densely sampled by
	the numerical integrator (see discussion below), and equation
	\refp{eq:likelihood} follows thereafter when the likelihood of observing
	the~$i$th datum~$\script{D}_i$ is given by a weighted~$e^{-\chi^2/2}$
	expression.
	Mathematically, the requirement for this marginalization arises naturally
	from the application of statistical likelihood and an IPPP to an
	evolutionary track (see Appendix~\ref{sec:l_derivation}).
	Qualitatively, it arises due to observational uncertainties -- there is no
	way of knowing which point on the evolutionary track the datum
	$\script{D}_i$ is truly associated with, and the only way to properly take
	this into account is to consider all pair-wise combinations of~\script{D}
	and~\script{M}.

	\item The mathematical requirement for a weighted as opposed to
	unweighted~$e^{-\chi^2/2}$ likelihood expression also arises naturally in
	our derivation.
	While the requirement for marginalization arises because of observational
	uncertainties, the requirement for weights arises because the likelihood of
	observing the datum~$\script{D}_i$ is proportionally higher at points in
	time when the SFR is high or the survey selection function is deeper.
	For a selection function~\script{S} and SFR~$\dot{M}_\star$, the weights
	should scale as their product:
	\begin{equation}
	w_j \propto \script{S}(\script{M}_j | \{\theta\})
	\dot{M}_\star(\script{M}_j | \{\theta\}).
	\label{eq:weights}
	\end{equation}
	In this paper, we have parametrized our GCE models such that the
	normalization of the SFH is inconsequential to the model predictions, and
	it is therefore essential that the corresponding normalization of the
	weights not impact the inferred likelihood.
	We therefore use~\textit{fractional} weights such that~$\sum_j w_j = 1$,
	and our derivation in Appendix~\ref{sec:l_derivation} demonstrates the
	validity of this decision.
	The unmodified weights should be used when fitting GCE models in which the
	normalization of the SFH~\textit{does} impact the predicted abundance
	evolution and consequently should impact the inferred likelihood.
	For these cases, the necessary modification to
	equation~\refp{eq:likelihood} is simple (see discussion below).
	% Because we have parameterized our GCE models in a manner such that the
	% normalization of the SFH is inconsequential to the abundance evolution,
	% the weights must be normalized such that they add up to 1.
	% For parametrizations in which the normalization does impact the abundance
	% evolution, the modification to equation~\refp{eq:likelihood} is simple
	% (see discussion below).

	\item The marginalization over the track and the weighted likelihood are
	of the utmost importance to the accuracy of the fit.
	In our tests against mock samples (see~\S~\ref{sec:mocks} below), we are
	unable to recover the known evolutionary parameters of the mock with
	discrepancies at the many-$\sigma$ level if either is neglected.
	% For this reason, we caution against the validity of evolutionary parameters
	% which are inferred from simple likelihood estimates, such as matching each
	% datum with the nearest point on the track.
	Equation~\refp{eq:likelihood} can however change slightly in form if one of
	a handful of conditions are met.
	We discuss them individually below, noting the necessary changes in form,
	and we refer to Appendix~\ref{sec:l_derivation} for detailed justification.
	\begin{itemize}
		\item \textit{The track is infinitely thin.}
		In the absence of measurement errors, all of the data would fall
		perfectly on a line in the observed space.
		As discussed at the beginning of~\S~\ref{sec:onezone}, the
		fundamental assumption of one-zone GCE models is instantaneous mixing
		of the various nuclear species throughout the star forming reservoir.
		Consequently, the ISM is chemically homogeneous and the models predict
		a single exact abundance for each element or isotope at any given time.
		If the model in question instead predicts a track of some finite
		width, then the likelihood function will have a fundamentally different
		form.

		\item \textit{Each observation is independent.}
		When this condition is met, the total likelihood of observing the
		data~\script{D} can be expressed as the product of the likelihood of
		observing each individual datum:
		\begin{subequations}\begin{align}
		L(\script{D} | \{\theta\}) &= \prod_i^N L(\script{D}_i | \script{M})
		\\
		\implies \ln L(\script{D} | \{\theta\}) &= \sum_i^N \ln
		L(\script{D}_i | \{\theta\}).
		\end{align}\end{subequations}
		This condition plays an important role in giving rise to the
		functional form of equation~\refp{eq:likelihood}, and if violated, the
		likelihood function will also have a fundamentally different form.

		\item \textit{The observational uncertainties are described by a
		multivariate Gaussian.}
		If this condition fails, the weighted~$\chi^2 = \Delta_{ij} C_i^{-1}
		\Delta_{ij}^T$ expression is no longer an accurate parametrization of
		$L(\script{D}_i | \script{M}_j)$ and it should be replaced with the
		more general form of equation~\refp{eq:likelihood}.
		A more general approach would be to replace~$e^{-\chi^2/2}$ with
		some kernel density estimate of the uncertainty at the point
		$\script{M}_j$ while retaining the weight~$w_j$, but this is only
		necessary for the subset of~\script{D} whose uncertainties are not
		adequately described by a multivariate Gaussian.

		\item \textit{The track is densely sampled.}
		That is, the spacing between the points on track~\script{M} is small
		compared to the observational uncertainties in the data.
		This assumption can be relaxed at the expense of including an
		additional correction factor~$\beta_{ij}$ given by equation
		\refp{eq:corrective_beta} which integrates the likelihood between each
		pair of adjacent points~$\script{M}_j$ and~$\script{M}_{j + 1}$ along
		the track.
		If computing the evolutionary track is sufficiently expensive,
		relaxing the number of points and including this correction factor may
		be the more computationally efficient option.

		\item \textit{The normalization of the SFH is inconsequential.}
		Only the time-dependence of the SFH impacts the abundance evolution
		predicted by the GCE model.
		As mentioned above, the model-predicted SFH and the selection function
		of the survey determine the weights~$w_j$ to attach to each point
		$\script{M}_j$ along the track, and if the normalization of the SFH
		does not impact the abundance evolution, then it must not impact the
		inferred likelihood.
		In our detailed derivation of equation~\refp{eq:likelihood}, we find
		that the proper manner in which to assign the weights is to normalize
		them such that they add up to 1 (see Appendix~\ref{sec:l_derivation}).
		Some GCE models, however, are parametrized such that the normalization
		of the SFH~\textit{does} impact the abundance evolution.
		One such example would be if the SFE
		timescale~$\tau_\star$ (see equation~\ref{eq:tau_star_def} and
		discussion in~\S~\ref{sec:onezone:gas}) depends on the gas supply
		$M_\text{g}$ in order to implement some version of a non-linear
		Kennicutt-Schmidt relation\footnote{
			$\dot{\Sigma}_\star \propto \Sigma_\text{g}^N \implies \tau_\star
			\propto \Sigma_\text{g}^{1 - N}$ where~$N \neq 1$.
			\citet{kennicutt1998} measured~$N = 1.4 \pm 0.15$ from the global
			gas densities and SFRs in star-forming spiral galaxies, although
			there have been recent advancements in this field suggesting more
			sophisticated forms than a single power-law (see discussion
			in~\S~2.6 of~\citealt{Johnson2021} and references therein).
		} where the normalization of the SFH and size of the galaxy are taken
		into account.
		In these cases, the likelihood function is given by equation
		\refp{eq:lnL_withweights} where the weights remain un-normalized and
		their sum must be subtracted from the corresponding version of
		equation~\refp{eq:likelihood}.
		This requirement to subtract their sum can be qualitatively understood
		as a penalty for models which predict data in regions of the observed
		space where there are none.
		It is a term which encourages parsimony, rewarding parameter choices
		which explain the data in as few predicted instances as possible.
		This penalty is still included in models which normalize the weights,
		with the tracks that extend too far in abundance space instead having a
		higher~\textit{fractional} weight from data at large~$\chi^2$, lowering
		the total likelihood (see discussion near the end of Appendix
		\ref{sec:l_derivation}).
		% That is, only the time-dependence of the SFH impacts the abundance
		% evolution predicted by the GCE model.
		% We have parametrized our models in a manner such that the normalization
		% of the SFH is inconsequential to the evolution in the abundances (for
		% details, see discussion in~\S~\ref{sec:onezone}).
		% Because the SFH and the selection function of the survey determine the
		% proper weights to attach to the likelihood of each pair-wise
		% combination between the data and the track (see
		% equation~\ref{eq:weights}), it is essential that the normalization of
		% the weights not impact the inferred likelihood.
		% In this case, we find in our derivation of the likelihood function that
		% the proper manner in which to handle the weights is to normalize them
		% such that they add up to 1 (see Appendix~\ref{sec:l_derivation}).
		% If the GCE model is instead parametrized such that the normalization of
		% the SFH~\textit{does} impact the abundance evolution, then the weights
		% must remain un-normalized, and their sum must be subtracted from
		% equation~\refp{eq:likelihood}.
		% The requirement that their sum be subtracted from the inferred
		% likelihood can be qualitatively understood as a penalty for models
		% which predict data in regions of the observed space where there is
		% none.
		% It is a term which encourages parsimony, rewarding parameter choices
		% which explain the data in as few predicted instances as possible.
		% This penalty is still included in models which normalize the weights;
		% in these cases, tracks which extend too far in the observed space
		% have a higher~\textit{fractional} weight from data at large~$\chi^2$,
		% lowering the total likelihood.
	\end{itemize}

	\item 
	We demonstrate the accuracy of equation~\refp{eq:likelihood}
	in~\S~\ref{sec:mocks} below by means of tests against mock data samples.
	Although our likelihood function does not include a direct fit to
	the stellar distributions in age and abundances, weighting the inferred
	likelihood by the SFR in the model indeed incorporates this information on
	how many stars should form at which ages and abundances.
	This results in an~\textit{implicit} fit to the age and abundance
	distributions, even though this information is not directly included in the
	likelihood calculation.

	\item There are a variety of ways in which one could measure the likelihood
	function~$L(\script{D} | \{\theta\})$ given by equation
	\refp{eq:likelihood}.
	In the present paper we employ the Markov chain Monte Carlo (MCMC) method,
	making use of the~\mc~\python~package~\citep{Foreman-Mackey2013} to
	construct our Markov chains.
	Despite being more computationally expensive than other methods (e.g.
	maximum a posteriori estimation), MCMC offers a more generic solution by
	sampling tails and multiple modes of the likelihood distribution which
	could otherwise be missed or inaccurately characterized by the assumption
	of Gaussianity.
	Our method should nonetheless be extensible to additional data sets
	described by GCE models with different parametrizations as well as
	different methods of optimizing the likelihood function, such as maximum
	a posteriori estimates.

	% \item We make use of the~\mc~\python~package~\citep{Foreman-Mackey2013} to
	% construct our markov chains, additionally using the~\vice~GCE software
	% \citep{Johnson2020} to compute the predicted abundances for a given choice
	% of parameters~$\{\theta\}$ provided by~\mc.
	% The log of the likelihood function~$\ln L(\script{D} | \{\theta\})$, as
	% required by~\mc~to take steps in parameter space, is then given by
	% equation~\refp{eq:likelihood}.

\end{itemize}

\end{document}
