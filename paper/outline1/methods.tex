
\documentclass[ms.tex]{subfiles}
\begin{document}

% \section{Methods}
% \label{sec:methods}

% \begin{itemize}

% 	\item We are interested in applying one-zone GCE models to dwarf galaxies
% 	and determining best-fit parameters.
% 	We begin by providing background on one-zone models, and then we select
% 	a parametrization from which we draw a fiducial mock stellar sample.
% 	We then use these data to introduce our fitting method.

% \end{itemize}

\section{Galactic Chemical Evolution}
\label{sec:onezone}

\begin{itemize}	

	\item The fundamental assumption of one-zone models is that newly produced
	metals mix instantaneously throughout the star forming gas reservoir.
	In detail, this assumption is valid as long as the mixing time-scale
	is short compared to the depletion time-scale (i.e. the average time a
	fluid element remains in the ISM before getting incorporated into new
	stars or ejected in an outflow, if present).
	Based on the observations of~\citet{Leroy2008},~\citet*{Weinberg2017}
	calculate that characteristic depletion times can range from~$\sim$500 Myr
	up to~$\sim$10 Gyr for conditions in typical star forming disc galaxies.
	In the dwarf galaxy regime, the length scales are short, star formation
	is slow~\citep{Hudson2015}, and the ISM velocities are turbulent
	\citep{Dutta2009, Stilp2013, Schleicher2016}.
	With this combination, instantaneous mixing should be a good approximation.
	We are unaware of any studies in the literature which address this
	observationally, though in principle it should be feasible on a
	galaxy-by-galaxy basis by assessing the sources of scatter in chemical
	space (e.g. the~\afe-\feh~plane).
	If instantaneous mixing is a valid approximation, then the intrinsic
	scatter about some evolutionary track should be negligible, and the
	observed scatter should be dominated by measurement uncertainty.

	\item In larger systems such as the Milky Way or Andromeda, the
	instantaneous mixing approximation breaks down.
	Observationally, the presence of gas-phase abundance gradients in
	star-forming disk galaxies~\citep*[see, e.g., recent reviews
	by][]{Kewley2019, Maiolino2019, Sanchez2020} indicates that metal diffusion
	in the radial direction is inefficient.
	Furthermore, processes such as radial gas flows~\citep{Lacey1985,
	Bilitewski2012, Vincenzo2020, Sharda2021} and the radial migration of stars
	(\citealp{Sellwood2002, Roskar2008a, Roskar2008b, Loebman2011, Minchev2011};
	\citealp*{Bird2012};~\citealp{Bird2013};~\citealp*{Grand2012a, Grand2012b,
	Kubryk2013};~\citealp{Okalidis2022}) are expected to significantly
	impact chemical enrichment and the observed abundance distributions in a
	given galactic region.
	This has prompted the development of a number of so-called ``multi-zone''
	models (\citealp*{Minchev2013, Minchev2014};~\citealp{Minchev2017,
	Johnson2021, Chen2022}) where the core motivation is to add some of the
	spatial information lost by one-zone models back into the framework by
	allowing the exchange of gas and stars between zones.

	\item With the goal of assessing the information content of stellar
	abundances in dwarf galaxies where the instantaneous mixing approximation
	is more likely to be valid, we do not explore multi-zone models here.
	The increased computational expense of multi-zone models also makes them
	less conducive to best-fit parameter determination, which is of central
	interest to this paper.
	When applied to any given galaxy, we emphasize that the accuracy of the
	methods we outline in this paper are contingent on the validity of the
	instantaneous mixing approximation.
	This assumption reduces GCE to a system of coupled integro-differential
	equations which we discuss below.
	Throughout this paper, we solve these equations using the publicly
	available~\textsc{Versatile Integrator for Chemical Evolution}
	(\vice;~\citealp{Johnson2020}).\footnote{
		\url{https://pypi.org/project/vice}
	}

\end{itemize}

\subsection{Inflows, Outflows, Star Formation, and Recycling}
\label{sec:onezone:gas}

\begin{itemize}

	\item At a given moment in time, gas is added to the interstellar medium
	(ISM) via inflows and recycled stellar envelopes and is removed from the
	ISM by star formation and outflows, if present.
	This gives rise to the following differential equation describing the
	evolution of the gas-supply:
	\begin{equation}
	\label{eq:mdot_gas}
	\dot{M}_\text{g} = \dot{M}_\text{in} - \dot{M}_\star - \dot{M}_\text{out}
	+ \dot{M}_\text{r},
	\end{equation}
	where~$\dot{M}_\text{in}$ is the infall rate,~$\dot{M}_\star$ is the star
	formation rate (SFR),~$\dot{M}_\text{out}$ is the outflow rate,
	and~$\dot{M}_\text{r}$ describes the return of stellar envelopes from
	previous generations of stars.

	\item Here we retain the characterization of outflows implemented in the
	publicly available GCE codes~\textsc{OMEGA}~\citep{Cote2017},
	\textsc{flexCE}~\citep{Andrews2017}, and~\vice~\citep{Johnson2020}, in
	which a ``mass loading factor''~$\eta$ describes a linear relationship
	between the outflow rate itself and the SFR:
	\begin{equation}
	\eta \equiv \frac{\dot{M}_\text{out}}{\dot{M}_\star}.
	\label{eq:mass_loading}
	\end{equation}
	This parametrization is appropriate for models in which massive stars are
	the dominant source of energy for outflow-driving winds.
	Empirically, the strength of outflows (i.e. the value of~$\eta$) is
	strongly degenerate with the absolute scale of nucleosynthetic yields.
	This ``yield-outflow degeneracy'' arises because the yields themselves are
	the primary source term in describing the enrichment rates in galaxies
	while outflows are the primary sink term.
	We quantify the strength of this degeneracy in more detail in Appendix
	\ref{sec:yield_outflow_degeneracy}.
	We discuss our adopted nucleosynthetic yields in~\S~\ref{sec:onezone:yields}
	below, which are intended to address this degeneracy at least in part.
	% This parametrization, appropriate for models in which massive stars are
	% the dominant source of energy for outflow-driving winds, is only one of
	% many prescriptions for outflows in the literature.
	% \begin{itemize}
	% 	\item Recently,~\citet{delosReyes2022} modelled the evolution of the
	% 	Sculptor dwarf spheroidal by letting the outflow rate instead be
	% 	linearly proportional to the SN
	% 	rate~$\dot{N}_\text{II} + \dot{N}_\text{Ia}$.
	% 	\citet*{Kobayashi2020} constructed a model for the Milky Way in which
	% 	outflow-driving winds develop in the early phases of evolution, but die
	% 	out as the Galaxy grows.
	% 	Based on theoretical models suggesting that the re-accretion timescales
	% 	of metals ejected from the Milky Way disk are short ($\sim$100 Myr;
	% 	\citealp{Melioli2008, Melioli2009, Spitoni2008, Spitoni2009}), some
	% 	authors even neglect outflows when modelling the Milky Way (e.g.
	% 	\citealp{Spitoni2019, Spitoni2021}, and the recently released publicly
	% 	available GCE code~\textsc{GalCEM}; Gjergo et al. 2022, in prep).
	% 	This argument is, however, at odds with measurements of the deuterium
	% 	abundance (\citealp{Linsky2006};~\citealp*{Prodanovic2010}) and the
	% 	$^3$He/$^4$He ratio~\citep{Balser2018} in the local ISM, both of which
	% 	are near their primordial values.
	% 	This indicates that much of the gas in the Galaxy has not been
	% 	processed by stars, further suggesting that nuclear products are
	% 	readily removed from the star forming reservoir by outflows and are
	% 	replaced by ongoing accretion of unprocessed
	% 	baryons~\citep{Weinberg2017b, Cooke2022}.
	% 	Although it is possible that much of the deuterium could be depleted
	% 	onto dust grains (\citealp{Romano2006};~\citealp*{Steigman2007}), the
	% 	same cannot be said of~$^3$He or~$^4$He.

	% 	\item Suffice it to say that the community has settled on neither the
	% 	proper parametrization nor the importance of outflows in GCE models.
	% 	Empirically, the strength of outflows (i.e. the value of~$\eta$) is
	% 	strongly degenerate with the absolute scale of effective
	% 	nucleosynthetic yields.
	% 	This ``yield-outflow degeneracy'' arises because the yields themselves
	% 	are the primary ``source term'' in describing the enrichment rates in
	% 	galaxies while outflows, when present, are the primary ``sink term.''
	% 	We quantify the strength of this degeneracy in more detail in Appendix
	% 	\ref{sec:yield_outflow_degeneracy}.

	% \end{itemize}

	\item The SFR and the mass of the ISM are related by the timescale
	$\tau_\star$, defined as the ratio of the two:
	\begin{equation}
	\tau_\star \equiv \frac{M_\text{g}}{\dot{M}_\star}.
	\end{equation}
	The inverse of this quantity~$\tau_\star^{-1}$ is typically referred to as
	the ``star formation efficiency''~\citep[e.g.][]{Schaefer2020} because it
	quantifies the~\textit{fractional} rate at which some ISM fluid element
	is forming stars.
	Some authors, however, refer to this parameter as the ``depletion time''
	\citep[e.g.][]{Tacconi2018}, because it also describes the e-folding
	decay timescale of the ISM mass due to star formation if no additional gas
	is added.
	Following~\citet{Weinberg2017}, we hereafter refer to~$\tau_\star$ as the
	SFE timescale, because depletion timescales in GCE models can shorten
	considerably due to outflows (specifically,
	$\tau_\text{dep} = \tau_\star / (1 + \eta - r)$ where~$r$ is a corrective
	term for recycled stellar envelopes; see discussion below).

	% \begin{itemize}
		% \item We relate the SFR to the gas supply by introducing the ``star
		% formation efficiency (SFE) timescale'':
		% \begin{equation}
		% \tau_\star \equiv \frac{M_\text{g}}{\dot{M}_\star},
		% \end{equation}
		% This quantity is often referred to as the ``depletion time'' in the
		% observational literature~\citep[e.g.][]{Tacconi2018}.
		% This nomenclature, taken from~\citet{Weinberg2017}, is based on its
		% inverse~$\tau_\star^{-1}$ often being referred to as the SFE itself
		% because it describes the~\textit{fractional} rate at which some ISM
		% fluid element is forming stars.

		% \item There are various prescriptions for outflows in the literature.
		% Some authors~\citep[e.g.][]{Andrews2017, Weinberg2017} assume a linear
		% proportionality between the two:
		% \begin{equation}
		% \label{eq:eta}
		% \dot{M}_\text{out} \equiv \eta\dot{M}_\star.
		% \end{equation}
		% Recently,~\citet{delosReyes2022} constrained the evolution of the
		% Sculptor dwarf spheroidal galaxy with a linear proportionality between
		% the SFR and the SN rate~$\dot{N}_\text{II} + \dot{N}_\text{Ia}$.
		% \citet*{Kobayashi2020} developed a model in which outflow-driving winds
		% develop in the early phases of the Milky Way's evolution, but die out
		% on some timescale as the Galaxy grows.
		% For modelling the Milky Way, some authors neglect outflows, arguing
		% that they do not signicantly alter the chemical evolution of the
		% disc~\citep[e.g.][]{Spitoni2019, Spitoni2021}.
		% In our mock sample and in our fits to the GSE and the Sagitarrius dSph,
		% we assume the linear proportionality given by equation~\refp{eq:eta}.
		% Our fitting routine, however, is easily extended to the parametrization
		% of~\citet{delosReyes2022}, and if outflows are to be neglected, one can
		% simply take~$\eta = 0$ in their fit.

	\item The recycling\footnote{
		Here, recycling refers only to ejected stellar envelopes returning
		baryons to the ISM.
		We do not implement the ``instantaneous recycling approximation'' as
		originally formulated, which assumes that stellar populations eject
		their entire nucleosynthetic yields instantaneously
		\citep[see, e.g., the review in][]{Tinsley1980}.
	} rate~$\dot{M}_\text{r}$ can be expressed as an integral over the SFH
	according to
	\begin{equation}
	\dot{M}_\text{r} = \int_0^T \dot{M}_\star(t)\dot{r}(T - t)dt,
	\label{eq:mdot_recycled}
	\end{equation}
	where~$T$ is the current time in a GCE model and~$r(\tau)$ is the
	``cumulative return fraction,'' which describes the fraction of a single
	stellar population's mass that has been returned to the ISM via ejected
	stellar envelopes at an age of~$\tau$.
	In detail,~$r$ is a complicated function which depends on the stellar
	IMF~\citep[e.g.][]{Salpeter1955, Miller1979, Kroupa2001, Chabrier2003},
	the initial-final remnant mass relation~\citep[e.g.][]{Kalirai2008},
	and the mass-lifetime relation\footnote{
		We assume a~\citet{Kroupa2001} IMF and the~\citet{Larson1974}
		mass-lifetime relation throughout this paper.
		These parameters, however, do not significantly impact our conclusions
		because to first-order the enrichment history of our models is
		set by the parameters~$\tau_\star$ and~$\eta$.
		Our fitting method (see discussion in~\S~\ref{sec:fitting}) is
		however easily extensible to models which relax these assumptions.
	} (e.g.~\citealp{Larson1974, Maeder1989};~\citealp*{Hurley2000}).
	We provide only qualitative discussion here; further details can be found
	in~\citet{Weinberg2017} and in the VICE science documentation.\footnote{
		\url{https://vice-astro.readthedocs.io/en/latest/science_documentation}
	}
	The recycling rate is initially high but declines rapidly as stellar
	populations age due to the steep nature of the mass-lifetime relation.
	\citet{Weinberg2017} demonstrate that it is therefore sufficiently accurate
	in one-zone models to assume that some fraction~$r_\text{inst}$ of a
	stellar population's initial mass is returned to the ISM instantaneously
	(see their Fig. 7; they recommend~$r_\text{inst} = 0.4$ for a
	\citealt{Kroupa2001} IMF, and~$r_\text{inst} = 0.2$ for
	a~\citealt{Salpeter1955} IMF).
	Although it is simpler to assume~$\dot{M}_\text{r} =
	r_\text{inst}\dot{M}_\star$, this numerical integration is computationally
	cheap and is already implemented in~\vice.

	% \item The recycling rate~$\dot{M}_\text{r}$, in general, depends on the
	% stellar IMF~\citep[e.g.][]{Salpeter1955, Miller1979, Kroupa2001,
	% Chabrier2003}, the initial-final remnant mass relation
	% \citep[e.g.][]{Kalirai2008}, and mass-lifetime relation
	% (e.g.~\citealp{Larson1974, Maeder1989};~\citealp*{Hurley2000}).
	% A single stellar population returns some fraction of its initial
	% mass~$r$ back to the ISM according to:
	% \begin{equation}
	% \label{eq:crf}
	% r(\tau) = \ddfrac{
	% 	\int_{m_\text{to}(\tau)}^u (m - m_\text{rem})\frac{dN}{dm} dm
	% }{
	% 	\int_l^u m \frac{dN}{dm} dm
	% }
	% \end{equation}
	% where~$l$ and~$u$ are the lower and upper mass limits of star formation,
	% respectively,~$m_\text{to}(\tau)$ is the turnoff mass of a stellar
	% population of age~$\tau$,~$m_\text{rem}$ is the mass of a remnant left
	% behind by a star of initial mass~$m$, and~$dN/dm$ is the adopted IMF.
	% Under this prescription, the recycling rate from~\textit{many} stellar
	% populations, taking into account the full SFH, is given by:
	% \begin{equation}
	% \label{eq:mdot_recycled}
	% \dot{M}_\text{r} = \int_0^T \dot{M}_\star(t) \dot{r}(T - t) dt
	% \end{equation}
	% where~$T$ is the time in the model.
	% Due to the steep nature of the mass-lifetime relation, the recycling
	% rate is dominated by young stellar populations.
	% \citet{Weinberg2017} demonstrate that it is sufficiently accurate in
	% one-zone models to assume that some fraction~$r_\text{inst}$ of a
	% stellar population's initial mass is returned to the ISM immediately
	% (see their Fig. 7; they recommend~$r_\text{inst} = 0.4$ for a
	% \citealt{Kroupa2001} IMF, and~$r_\text{inst} = 0.2$ for a
	% \citealt{Salpeter1955} IMF).
	% Although it is simpler to assume~$\dot{M}_\text{r} =
	% r_\text{inst}\dot{M}_\star$, numerical integration of equations
	% \refp{eq:crf} and~\refp{eq:mdot_recycled} is easy, and~\vice~already
	% does it, so we stick with that.

	\item The first-order details of the gas-phase evolutionary track in
	the~\afe-\feh~plane are determined by the SFE timescales~$\tau_\star$ and
	the mass loading factor~$\eta$~\citep{Weinberg2017}.
	With low~$\tau_\star$ (i.e. high SFE), nucleosynthesis is fast because
	star formation is fast, and a higher metallicity can be obtained before
	the onset of SN Ia than in lower SFE models.
	For this reason,~$\tau_\star$ plays the dominant role in shaping the
	position of the knee in the~\afe-\feh~plane.
	As the galaxy evolves, it approaches a chemical equilibrium in which
	newly produced metals are balanced by the loss of metals to outflows
	and new stars.
	Controlling the strength of the sink term of outflows,~$\eta$ plays
	the dominant role in shaping the late-time equilibrium abundance of the
	model, with high outflow models (i.e. high~$\eta$) predicting lower
	equilibrium abundances than their weak outflow counterparts.
	For observed data, the shape of the track itself directly constrains
	these parameters (see discussion in~\S~\ref{sec:mocks:fiducial} below).
	The detailed form of the SFH has minimal impact on the shape of the
	tracks, provided that there are no sudden events such as a burst of
	star formation~\citep{Weinberg2017, Johnson2020}.
	Instead, that information is encoded in the density of stars along the
	evolutionary track and in the stellar metallicity distribution functions
	(MDFs).

	% \end{itemize}

\end{itemize}

\subsection{Enrichment}
\label{sec:onezone:enrichment}

% \subsection{Core Collapse Supernovae}
% \label{sec:onezone:ccsne}

\begin{itemize}

	\item In the present paper, we focus on th eenrichment of the so-called
	``alpha'' (e.g. O, Ne, Mg, Si) and ``iron-peak'' (e.g. Cr, Fe, Ni, Zn)
	elements, with the distribution of stars in the~\afe-\feh~plane being our
	primary observational diagnostic to distinguish between GCE models.
	Alpha elements are so-named because they are produced by alpha capture
	reactions in massive stars, and for the lighter ones like O and Mg, this
	is the only dominant enrichment source (see, e.g., the review in
	\citealp{Johnson2019}).
	Iron-peak elements are also produced in massive stars, but also owe a
	portion of their abundance to white dwarf SNe occurring on longer delay
	times.
	Heavier nuclei (specifically Sr and up) are also produced by neutron
	capture processes.
	Although we do not treat these elements here, our fitting method (see
	discussion in~\S~\ref{sec:fitting}) is easily extensible to GCE models
	which do, provided that the data contain such measurements.

	\item Due to the steep nature of the stellar mass-lifetime relation
	\citep[e.g.][]{Larson1974, Maeder1989, Hurley2000}, massive stars, their
	winds, and their supernovae enrich the ISM on~$\sim$few Myr timescales.
	As long as this is shorter than the relevant timescales for a given
	galaxy's evolution, and the present-day stellar mass is sufficiently high
	such that stochastic sampling of the IMF does not significantly impact the
	yields, then it is adequate to approximate this nucleosynthetic material as
	being ejected instantaneously following a single stellar population's
	formation.
	This implies a linear relationship between the CCSN enrichment rate and
	the SFR:
	\begin{equation}
	\label{eq:mdot_cc}
	\dot{M}_\text{x}^\text{CC} = y_\text{x}^\text{CC}\dot{M}_\star
	\end{equation}
	where~$y_\text{x}^\text{CC}$ is the IMF-averaged fractional net yield from
	massive stars.
	That is, for a fiducial value of~$y_\text{x}^\text{CC} = 0.01$, 100~\msun~of
	star formation would produce 1~\msun~of~\textit{newly produced} mass of
	element x (we implement the return of previously produced metals separately;
	see discussion below).

% \subsection{Type Ia Supernovae}
% \label{sec:onezone:sneia}

	\item Unlike CCSN enrichment, SN Ia enrichment occurs on a significantly
	extended delay time distribution (DTD).
	The details of the DTD are a topic of active inquiry
	\citep[e.g.][]{Greggio2005, Strolger2020, Freundlich2021}, and at least a
	portion of the uncertainty can be traced to uncertainties in both galactic
	and cosmic star formation histories.
	Comparisons of the cosmic SFH~\citep[e.g.][]{Madau2014, Madau2017} with
	volumetric SN Ia rates as a function of redshift, the cosmic DTD appears
	broadly consistent with a uniform~$\tau^{-1}$ power-law (e.g.
	\citealp*{Maoz2012a, Maoz2012b, Graur2013};~\citealp{Graur2014}).
	Following~\citet{Weinberg2017}, we take a~$\tau^{-1.1}$ power-law DTD with a
	minimum delay-time of~$t_\text{D} = 150$ Myr, though in principle this
	delay-time could be as short as~$t_\text{D} \approx 40$ Myr due to the
	lifetimes of the most massive white dwarf progenitors.
	For any selected DTD~$R_\text{Ia}(\tau)$, the SN Ia enrichment rate can be
	expressed an integral over the SFH weighted by the DTD:
	\begin{equation}
	\dot{M}_\text{x}^\text{Ia} = y_\text{x}^\text{Ia} \ddfrac{
		\int_0^{T - t_\text{D}} \dot{M}_\star(t) R_\text{Ia}(T - t) dt
	}{
		\int_0^\infty R_\text{Ia}(t) dt
	}.
	\end{equation}

	\item In general, the mass of some element x in the ISM is also affected by
	outflows, recycling, star formation, and infall.
	The enrichment rate can be calculated by simply adding up all of the source
	terms and subtracting the sink terms:
	\begin{equation}
	\label{eq:enrichment_eq}
	\dot{M}_\text{x} = \dot{M}_\text{x}^\text{CC} + \dot{M}_\text{x}^\text{Ia}
	- Z_\text{x}\dot{M}_\star - Z_\text{x}\dot{M}_\text{out} +
	\dot{M}_\text{x,r},
	\end{equation}
	where the rate of return of the element x from recycled stellar envelopes
	can be computed by weighting the integral in equation~\refp{eq:mdot_recycled}
	by~$Z_\text{x}(t)$.
	If there is metal-rich infall, this equation picks up the additional term
	$Z_\text{x,in}\dot{M}_\text{in}$ quantifying that, although here we assume
	that infall is pristine.
	If additional enrichment sources such as slow neutron capture in asymptotic
	giant branch stars~\citep[e.g.][]{Cristallo2011, Cristallo2015, Ventura2013,
	Ventura2014, Ventura2018, Ventura2020, Karakas2016, Karakas2018} are to be
	included, then they contribute additional source terms to
	equation~\refp{eq:enrichment_eq} as well.
	Since we focus this paper on alpha and iron-peak elements whose yields are
	dominated by SNe~\citep[e.g.][]{Johnson2019}, we do not discus these
	processes here.

\end{itemize}

\subsection{Nucleosynthetic Yields}
\label{sec:onezone:yields}

\begin{itemize}

	\item Empirically, nucleosynthetic yields are degenerate with the strength
	of outflows (i.e. the value of the mass loading factor~$\eta$).
	This degeneracy is quite strong, and we quantify its strength in more
	detail in Appendix~\ref{sec:yield_outflow_degeneracy}.
	This arises because yields and outflows are the dominant source and sink
	terms in equation~\refp{eq:enrichment_eq} above.
	Consequently, high-yield and high-outflow outflow models have a low-yield
	and low-outflow counterpart that predicts a similar enrichment history.
	Some authors~\citep[e.g.][]{Minchev2013, Minchev2014, Minchev2017,
	Spitoni2020, Spitoni2021} even assume that outflows do not sweep up ambient
	ISM (i.e.~$\eta = 0$) , while others~\citep[e.g.][]{Andrews2017,
	Weinberg2017, Cote2017, Trueman2022} instead argue that this is an
	important ingredient in GCE models.

	\item In order to break this degeneracy, only a single number setting the
	absolute scale is required.
	Throughout this paper (with the exception of
	Appendix~\ref{sec:yield_outflow_degeneracy} where the lack of an absolute
	scale is directly relevant), we set the alpha element yield from massive
	stars to be exactly~$\yacc = 0.01$ and otherwise let our Fe yields be
	free parameters.
	This value is somewhat informed by nucleosynthesis theory in that massive
	star evolutionary models (e.g.~\citealp*{Nomoto2013};~\citealp{Sukhbold2016,
	Limongi2018}) typically predict~$y_\text{O}^\text{CC} = 0.005 - 0.015$
	(see discussion in, e.g.,~\citealp{Weinberg2017, Johnson2020}).
	The primary motivation behind this choice, however, is to select a round
	number from which our best-fit values affected by this degeneracy can
	simply be scaled up or down to accommodate alternate parameter choices.
	If swept up ambient ISM is to be neglected in the outflow, then~\yacc~can
	be included as a free parameter and the overall scale is instead set by
	the assumption that~$\eta = 0$.
	While massive star explodability and the black hole landscape
	\citep[e.g.][]{Pejcha2015, Ertl2016, Sukhbold2016} can produce factor
	of~$\sim2 - 3$ fluctuations in the value of~\yacc~\citep{Griffith2021},
	values lower by an order of magnitude or more can be achieved if a
	significant fraction of SN ejecta are immediately lost to a hot outflow.
	This is a necessary addition to models which assume~$\eta = 0$, otherwise
	unphysically high metal abundances will arise. 
	There is some observational support for this scenario in that galactic
	outflows have been seen to be significantly more metal-rich than the ISM of
	the host galaxy (\citealp*{Chisholm2018};~\citealp{Cameron2021}).
	However, the outflow metallicities are not as high as the SN ejecta
	themselves and cold-phase material is observed in galaxy-scale outflows as
	well~\citep[e.g.][]{Lopez2020, Veilleux2020}.

	% \item In general, nucleosynthetic yields are degenerate with the outflow
	% mass loading factor~$\eta$.
	% We quantify this is in more detail in Appendix X, simply noting there that
	% the two are simply the dominant source and sink terms, and as such,
	% high-yield high-outflow models generally have a low-yield low-outflow
	% counterpart that predicts a similar chemical evolution.
	% In order to break this degeneracy, only one number setting the absolute
	% scale is required.
	% Here, we simply set the alpha element yield to~\yacc~= 0.01.
	% This value is somewhat informed by nucleosynthesis theory in that
	% massive star evolutionary models (e.g.~\citealp{Sukhbold2016,
	% Limongi2018};~\citealp*{Nomoto2013}) typically predict
	% $y_\text{O}^\text{CC} = 0.005 - 0.015$ (see discussion in, e.g.,
	% \citealp{Weinberg2017, Johnson2020}), but is otherwise intended to be a
	% round number from which our best-fit values affected by this degeneracy can
	% simply be scaled up or down.

	% \item We let our Fe yields~\yfecc~and~\yfeia~be free parameters.
	% With this approach, we implicitly fit the height of the [$\alpha$/Fe]
	% plateau as well as the Fe yield ratio~\yfecc/\yfeia.

	% \item In general, nucleosynthetic material is also expelled by asymptotic
	% giant branch (AGB) stars~\citep[e.g.][]{Cristallo2011, Cristallo2015,
	% Ventura2013, Ventura2014, Ventura2018, Ventura2020, Karakas2016,
	% Karakas2018}.
	% Here we are interested primarily in~$\alpha$ and Fe-peak elements, elements
	% whose AGB star yields are negligible compared to their SN yields
	% \citep[e.g.][]{Johnson2019}.
	% We therefore omit discussion of AGB star nucleosynthesis here, but we note
	% that our fitting method described in~\S~\ref{sec:fitting} is easily
	% extensible to include an AGB star enrichment channel.
	% Mathematical details of how this is implemented in~\vice~can be found in
	% \citet{Johnson2020},~\citet{Johnson2022}, and in the~\vice~science
	% documentation.\footnote{
	% 	\url{https://vice-astro.readthedocs.io/en/latest/science_documentation/index.html}
	% }

\end{itemize}


\section{The Fitting Method}
\label{sec:fitting}

\begin{itemize}

	% \item Here we provide an overview of our method for fitting one-zone GCE
	% models to data.
	\item Our fitting method uses the abundances and ages (where available) of
	an ensemble of stars and, with no binning of the data, accurately
	constructs the~\textit{likelihood function}~$L(\script{D} | \{\theta\})$
	describing the probability of observing the data~\script{D} given a set of
	model parameters~$\{\theta\}$.
	This is related to the~\textit{posterior probability}~$L(\{\theta\} |
	\script{D})$ according to Bayes' Theorem:
	\begin{equation}
	L(\{\theta\} | \script{D}) = \frac{
		L(\script{D} | \{\theta\}) L(\{\theta\})
	}{
		L(\script{D})
	},
	\end{equation}
	where~$L(\{\theta\})$ is the likelihood of the parameters themselves
	(known as the~\textit{prior}) and~$L(\script{D})$ is the likelihood of the
	data (known as the~\textit{evidence}).
	Although it is more desirable to measure the posterior probability,
	in practice only the likelihood function can be robustly determined
	because the prior is not directly quantifiable; it requires quantitative
	information independent of the data on the accuracy of a chosen set of
	parameters.
	With no additional information on what the parameters should be, the best
	practice is to assume a ``flat'' or ``uniform'' prior in which
	$L(\{\theta\})$ is a constant, and therefore
	$L(\{\theta\} | \script{D}) \approx L(\script{D} | \{\theta\})$; we retain
	this convention here.

	\item As mentioned in~\S~\ref{sec:intro}, the sampling of stars from an
	underlying distribution in abundance space proceeds according to an IPPP
	\citep[e.g.][]{Press2007}.
	In Appendix~\ref{sec:l_derivation}, we present a detailed derivation of
	our likelihood function in which we apply the principles of an IPPP and
	statistical likelihood to one-zone GCE models.
	Owing to its detailed nature, we provide only qualitative discussion of
	its form here, reserving more in-depth justification for Appendix
	\ref{sec:l_derivation}.
	Though our use case in this paper is in the context of GCE, our derivation
	assumes only that the model predicts a ``track'' in some observed space.
	It is therefore highly generic and should be extensible to other
	astrophysical models whose chief prediction is a track of some form (e.g.
	stellar streams in kinematic space and stellar isochrones on CMDs).

	\item In practice, the evolutionary track predicted by a one-zone GCE model
	is most often not expressed in some analytic functional form.
	Instead, it is most often quantified in some piece-wise linear form
	predicted by a numerical code (in our case,~\vice).
	For a sample~\script{D} with~$N$ individual stars and a predicted track
	\script{M} sampled at~$K$ points in the observed space, the likelihood
	function is given by
	\begin{equation}
	\ln L(\script{D} | \{\theta\}) = \sum_i^N \ln \left(
	\sum_j^K w_j \exp \left(
	\frac{-1}{2} \Delta_{ij} C_i^{-1} \Delta_{ij}^T
	\right)
	\right),
	\label{eq:likelihood}
	\end{equation}
	where~$\Delta_{ij} = \script{D}_i - \script{M}_j$ is the vector difference
	between the~$i$th datum and the~$j$th point on the predicted track,
	$C_i^{-1}$ is the inverse covariance matrix of the~$i$th datum, and~$w_j$
	is a weight to be attached to~$\script{M}_j$.
	The weights scale with both the SFR and the selection function of the
	survey at the point~$\script{M}_j$ along the track, and they should be
	normalized such that~$\sum_j w_j = 1$ (though unnormalized weights should
	be used if the normalization of the SFH impacts the model predictions; see
	discussion below).

	\item This functional form arises from marginalizing the likelihood of
	observing each datum over the entire evolutionary track~\script{M}
	according to:
	\begin{equation}
	\ln L(\script{D} | \{\theta\}) = \sum_i^N \ln \left(
	\sum_j^K L\left(\script{D}_i | \script{M}_j
	\right)\right),
	\end{equation}
	where the likelihood of observing the~$i$th datum given the~$j$th point on
	the evolutionary track is given by a weighted~$e^{-\chi^2/2}$
	expression.
	Mathematically, the requirement for this marginalization arises naturally
	from the application of statistical likelihood and an IPPP to an
	evolutionary track (see Appendix~\ref{sec:l_derivation}).
	Qualitatively, it arises due to observational uncertainties.
	There is now way of knowing which point on the evolutionary track the datum
	$\script{D}_i$ is truly associated with, and the only way to properly take
	this into account is to consider all pair-wise combinations of~\script{D}
	and~\script{M}.

	\item The requirement for a weighted as opposed to
	unweighted~$e^{-\chi^2/2}$ likelihood expression also arises naturally out
	of the application of statistical likelihood and the IPPP to an
	evolutionary track.
	Qualitatively, this arises because 
	While the requirement for marginalization arises because there is no way of
	knowing which point on the evolutionary track a datum~$\script{D}_i$ arose
	from, the requirement for weights arises because it is proportionally more
	likely to be associated with points on the track at which either the SFR is
	high or the survey selection function is deeper.
	For a survey selection function~\script{S} and SFR~$\dot{M}_\star$, the
	weights should scale as their product:
	\begin{equation}
	w_j \propto \script{S}(\script{M}_j, \{\theta\}) \dot{M}_\star(t_j |
	\{\theta\}),
	\label{eq:weights}
	\end{equation}
	where~$t_j$ denotes the time in the GCE model under the selection of
	parameters~$\{\theta\}$.
	Because we have parameterized our GCE models in a manner such that the
	normalization of the SFH is inconsequential to the abundance evolution,
	the weights must be normalized such that they add up to 1.
	For parametrizations in which the normalization does impact the abundance
	evolution, the modification to equation~\refp{eq:likelihood} is simple
	(see discussion below).

	\item The validity of equation~\refp{eq:likelihood} is contingent on the
	following assumptions.
	\begin{itemize}
		\item \textit{The track is infinitely thin.}
		In the absence of measurement errors, all of the data would fall
		perfectly on a line in the observed space.
		As discussed at the beginning of~\S~\ref{sec:onezone}, the
		fundamental assumption of one-zone GCE models is instantaneous
		diffusion and, consequently, chemical homogeneity.
		They sacrifice this spatial information in exchange for a drastic
		reduction in computational expense.
		By construction, they predict a single exact abundance of all nuclear
		species in the star formation reservoir at any given time.
		If the model in question instead predicts a track with some finite
		width, then computing the likelihood function is a fundamentally
		different problem.

		\item \textit{Each observation is independent.}
		When this condition is met, the total likelihood of observing the
		data~\script{D} can be expressed as the product of the likelihood of
		observing each individual datum:
		\begin{subequations}\begin{align}
		L(\script{D} | \{\theta\}) &= \prod_i^N L(\script{D}_i | \script{M})
		\\
		\implies \ln L(\script{D} | \{\theta\}) &= \sum_i^N \ln
		L(\script{D}_i | \{\theta\}).
		\end{align}\end{subequations}
		This condition plays an important role in giving rise to the
		functional form of equation~\refp{eq:likelihood}.

		\item \textit{The observational uncertainties are described by a
		multivariate Gaussian.}
		If this condition fails, the expression for~$\chi^2 =
		\delta_{ij}C_i^{-1}\delta_{ij}^T$ is no longer accurate.
		In this case, the expression~$L(\script{D}_i | \script{M}_j) \propto
		w_j e^{-\chi^2/2}$ should be replaced with some kernel density
		estimate of the uncertainty at the point~$\script{M}_j$, but retaining
		the weight~$w_j$.

		\item \textit{The track is densely sampled.}
		That is, the spacing between the points on track~\script{M} is small
		compared to the observational uncertainties in the data.
		This assumption can be relaxed at the expense of including an
		additional correction factor~$\beta_{ij}$ given by equation
		\refp{eq:corrective_beta} which integrates the likelihood between each
		pair of adjacent points~$\script{M}_j$ and~$\script{M}_{j + 1}$ along
		the track.
		If computing the evolutionary track is sufficiently expensive,
		relaxing the number of points and including this correction factor may
		be the more computationally efficient option.

		\item \textit{The normalization of the SFH is inconsequential.}
		That is, only the time-dependence of the SFH impacts the abundance
		evolution predicted by the GCE model.
		We have parametrized our models in a manner such that the normalization
		of the SFH is inconsequential to the evolution in the abundances (for
		details, see discussion in~\S~\ref{sec:onezone}).
		Because the SFH and the selection function of the survey determine the
		proper weights to attach to the likelihood of each pair-wise
		combination between the data and the track (see
		equation~\ref{eq:weights}), it is essential that the normalization of
		the weights not impact the inferred likelihood.
		In this case, we find in our derivation of the likelihood function that
		the proper manner in which to handle the weights is to normalize them
		such that they add up to 1 (see Appendix~\ref{sec:l_derivation}).
		If the GCE model is instead parametrized such that the normalization of
		the SFH~\textit{does} impact the abundance evolution, then the weights
		must remain un-normalized, and their sum must be subtracted from
		equation~\refp{eq:likelihood}.
		The requirement that their sum be subtracted from the inferred
		likelihood can be qualitatively understood as a penalty for models
		which predict data in regions of the observed space where there is
		none.
		It is a term which encourages parsimony, rewarding parameter choices
		which explain the data in as few predicted instances as possible.
		This penalty is still included in models which normalize the weights;
		in these cases, tracks which extend too far in the observed space
		have a higher~\textit{fractional} weight from data at large~$\chi^2$,
		lowering the total likelihood.
	\end{itemize}

	\item 
	We demonstrate the accuracy of equation~\refp{eq:likelihood}
	in~\S~\ref{sec:mocks} below by means of tests against mock data samples.
	Although our likelihood function does not include a direct fit to
	the stellar distributions in age and abundances, weighting the inferred
	likelihood by the SFR in the model indeed incorporates this information on
	how many stars should form at which ages and abundances.
	With this~\textit{implicit} fit, our method accurately characterizes the
	age and abundance distributions of our mock samples even though they are
	not explicitly included in the likelihood calculation.

	\item There are a variety of ways in which one could measure the likelihood
	function~$L(\script{D} | \{\theta\})$ given by equation
	\refp{eq:likelihood}, and in the present paper we use the Markov Chain
	Monte Carlo (MCMC) method.
	Despite being more computationally expensive than, e.g., maximum a posterori
	(MAP) estimation, MCMC offers a more generic solution than other options by
	sampling tails and multiple modes of the likelihood distribution that could
	otherwise be missed by assuming Gaussianity.
	Our method should nonetheless be extensible to additional data sets
	described by GCE models with different parametrizations as well as
	different methods of optimizing the likelihood function, such as MAP
	estimates.

	\item We make use of the~\mc~\python~package~\citep{Foreman-Mackey2013} to
	construct our markov chains, additionally using the~\vice~GCE software
	\citep{Johnson2020} to compute the predicted abundances for a given choice
	of parameters~$\{\theta\}$ provided by~\mc.
	% The log of the likelihood function~$\ln L(\script{D} | \{\theta\})$, as
	% required by~\mc~to take steps in parameter space, is then given by
	% equation~\refp{eq:likelihood}.

\end{itemize}

\end{document}
