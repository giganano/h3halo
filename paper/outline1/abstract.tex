
\documentclass[ms.tex]{subfiles}
\begin{document}

\begin{abstract}
We develop a Bayesian method for fitting one-zone models of galactic chemical
evolution to observed stellar abundances and ages.
The two defining characteristics of this method are: (i) the likelihood of a
given model-data pair of points is weighted by the star formation rate in the
model, and (ii) for every data point, the likelihood is marginalized over the
entire evolutionary history of the model.
We demonstrate the accuracy of this method by means of mock data samples.
These tests indicate that measurement precision and sample size are of similar
importance in establishing the precision of the best-fit parameters, while
stellar age information is of minimal importance.
Indeed, even in the absence of age information, we find that this method
produces accurate fits for (i) the timescales associated with the infall and
star formation histories, and (ii) the total duration of star formation (i.e.
quenching time) in a galaxy.
We apply this new method to the Gaia-Sausage-Enceladus and the Sagitarrius
dwarf Spheroidal, and we discuss how the best-fit parameters compare with
existing measurements in the literature and the implications thereof.
\end{abstract}

\end{document}
