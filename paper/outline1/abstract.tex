
\documentclass[ms.tex]{subfiles}
\begin{document}

\begin{abstract}
We develop a Bayesian method for fitting one-zone models of galactic chemical
evolution to observed stellar abundances and ages, where available.
This method requires no binning of the data and is based on the application of
an inhomogeneous poisson point process (IPPP) to an arbitrary evolutionary
track in some observed space.
It should therefore be extensible to other astrophysical models which also
predict tracks (e.g. stellar streams and isochrones).
The application of an IPPP to one-zone models results in the derivation of
a single exact form of the likelihood function for determining best-fit
parameters.
We demonstrate by means of mock data samples that this likelihood function
yields best-fit parameters where the fit precision scales with sample size
approximately as~$N^{-0.5}$ and that it remains accurate with as few
as~$\sim$20 stars in the sample.
We find that evolutionary timescales, including the total duration of star
formation, can be derived even in the absence of age information via its impact
on the shape of the metallicity distribution.
Age measurements nonetheless improve the precision of the inferred timescales,
but only for samples where~$\lesssim$30\% of the stars have them.
Above this threshold, increasing the portion of the sample with age information
has a marginal impact on the fit precision.
We apply our method to the Gaia-Sausage Enceladus and the Wukong stellar
streams in the Milky Way halo using data from the H3 survey, finding good
agreement with existing constraints in the literature.
\end{abstract}

\end{document}
