
\documentclass[ms.tex]{subfiles}
\begin{document}

\begin{abstract}
We develop a Bayesian method for fitting one-zone models of galactic chemical
evolution to observed stellar abundances and ages.
We demonstrate the accuracy of this method by means of mock data samples, for
which we are able to recover the known values of evolutionary parameters with
accuracy and precision consistent with random processes.
Our tests indicate the measrement precision and sample size are of comparable
importance in establishing the precision of best-fit parameters, while stellar
age information plays a significantly weaker role.
We find that known parameters of a mock sample's evolutionary history,
including the infall timescale and the duration of star formation, are
accurately recovered even in the absence of age information.
We apply this method to the Gaia-Sausage-Enceladus and the Sagitarrius dwarf
Spheroidal using data observed with the H3 survey.
Our characterization of the Gaia-Sausage-Enceladus achieves
$\chi^2_\text{dof} = 1.2$, and the duration of star formation that we derive
from chemistry alone is consistent with directage constraints.
This method is based on the treatment of stellar abundances in as a one-zone
model of galactic chemical evolution sampled according to an inhomogeneous
poisson point process.
It requires no binning of the data, and should be extensible to other models
which predict evolutionary tracks in some observed space (e.g. stellar
isochrones and stellar streams).
\end{abstract}

\end{document}
