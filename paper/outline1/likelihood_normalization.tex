
\documentclass[ms.tex]{subfiles}
\begin{document}

\section{Derivation of the Likelihood Function}
\label{sec:weightnorm}

\begin{itemize}

	\item In this section we provide a detailed derivation of our likelihood
	function incorporating the principles of the IPPP.
	Given some expression for the model predicted track in the observable
	space~\script{M}, the likelihood of observing the data given some set of
	model parameters~$\{\theta\}$ can be expressed as the integrated
	differential likelihood along the track:
	\begin{equation}
	L(\script{D} | \{\theta\}) = \int_\script{M} dL =
	\int_\script{M} L(\script{D} | \script{M}) p(\script{M} | \{\theta\})
	d\script{M},
	\end{equation}
	where~$p(\script{M} | \{\theta\})$ is the probability that a singular
	datum will be drawn from the model at a given point along the track.

	\item According to the IPPP,~$p(\script{M} | \{\theta\})$ can be expressed
	in terms of the predicted observed density~$\lambda(\script{M} |
	\{\theta\})$:
	\begin{equation}
	p(\script{M}_j | \{\theta\}) = e^{-N_\lambda}
	\prod_i^N \lambda(\script{M}_j | \{\theta\})
	\end{equation}
	where the product is taken over the~$N$ points in the sample~\script{D},
	$\script{M}_j$ denotes a specific point along the model track~\script{M},
	and
	\begin{equation}
	N_\lambda = \int_\script{M} \lambda(\script{M} | \{\theta\}) d\script{M}
	\end{equation}
	is the expected number of instances taken from the line integral of the
	density over the track.
	Note that~$\lambda(\script{M} | \{\theta\})$ is the predicted
	\textit{observed} density, and should therefore incorporate any selection
	effects present in the data.
	It can be expressed as the product of the selection function~$\script{S}$
	and the~\textit{intrinsic} density~$\Lambda$ according to
	\begin{equation}
	\label{eq:lambda_def}
	\lambda(\script{M}_j | \{\theta\}) =
	S(\script{M}_j, \{\theta\})
	\Lambda(\script{M}_j | \{\theta\}).
	\end{equation}

	\item By plugging this into our expression for the likelihood function,
	we obtain:
	\begin{subequations}\begin{align}
	L(\script{D} | \{\theta\}) &=
	\int_\script{M} \left(\prod_i^N L(\script{D}_i | \script{M})\right)
	\left(e^{-N_\lambda}\prod_i^N \lambda(\script{M} | \{\theta\})\right)
	d\script{M}
	\\
	&= e^{-N_\lambda} \prod_i^N \int_\script{M}
	L(\script{D}_i | \script{M}) \lambda(\script{M} | \{\theta\}) d\script{M},
	\end{align}\end{subequations}
	where we have exploited the conditional independence of the likelihood of
	observing each individual datum~$\script{D}_i$, allowing us to substitute
	$L(\script{D} | \script{M}) = \prod_i^N L(\script{D}_i | \script{M})$.
	We note that the expected number of observed data~$N_\lambda$ is not a
	function of the predicted track~\script{M} but only of the model parameters
	$\{\theta\}$.

	\item In many one-zone GCE models, the normalization of the SFH is
	irrelevant when only considering the predicted abundances.
	Because abundances are given by the metal mass~\textit{relative} to the ISM
	mass, this normalization cancels.
	Indeed, if we amplify the SFH of our models by any factor, the model
	predicts the same evolution in abundances.
	It is therefore essential that the total predicted number of instances
	$N_\lambda$ not impact our estimate of the maximum likelihood parameters
	$\{\theta\}$.
	We therefore consider a density~$\rho$ with some unknown overall
	normalization defined according to:
	\begin{subequations}\begin{align}
	\label{eq:rho_def}
	\lambda(\script{M} | \{\theta\}) &= N_\lambda \rho(\script{M} | \{\theta\})
	\\
	\label{eq:rho_integrated}
	\int_\script{M} \rho(\script{M} | \{\theta\}) d\script{M} &= 1
	\end{align}\end{subequations}
	
	\item Plugging this into our likelihood function yields:
	\begin{subequations}\begin{align}
	L(\script{D} | \{\theta\}, N_\lambda) &= e^{-N_\lambda} \prod_i^N
	\int_\script{M} L(\script{D}_i | \script{M}) N_\lambda \rho(\script{M} |
	\{\theta\}) d\script{M}
	\\
	&= e^{-N_\lambda} N_\lambda^N \prod_i^N \int_\script{M}
	L(\script{D}_i | \script{M}) \rho(\script{M} | \{\theta\}) d\script{M}.
	\end{align}\end{subequations}
	Now taking the log of the likelihood function produces the following
	expression for~$\ln L$:
	\begin{equation}\begin{split}
	\ln L(\script{D} | \{\theta\}, N_\lambda) &= -N_\lambda + N \ln N_\lambda +
	\\ &\qquad \sum_i^N \ln \left( \int_M L(\script{D}_i | \script{M})
	\rho(\script{M} | \{\theta\}) d\script{M} \right).
	\label{eq:lnL_integral}
	\end{split}\end{equation}

	\item As the integral within the summation would suggest, the next task is
	to determine the appropriate manner in which to assess the likelihood of
	observing the singular datum~$\script{D}_i$, given the predicted
	track~$\script{M}$.
	Because of observational uncertainties, there is no way to know~\textit{a
	priori} which point on the track~$\script{M}$ is truly associated with.
	It is thus essential to keep the integral within the logarithm in the above
	equation, effectively marginalizing over the entire track to
	quantify~$\L(\script{D}_i | \script{M})$.
	In practice, the track may be complicated in shape, and is generally not
	known as a smooth and continuous function, instead in some piece-wise
	linear approximation computed by a numerical code.

	\item The line segment connecting the knots on the track~$\script{M}_j$ and
	$\script{M}_{j + 1}$ can be expressed as:
	\begin{equation}
	\Delta\script{M}_{j,j + 1} = \script{M}_j + q(\script{M}_{j + 1} -
	\script{M}_j) \quad (0 \leq q \leq 1).
	\end{equation}
	If the errors on the observed datum~$\script{D}_i$ are accurately described
	by a multivariate Gaussian, then the likelihood of observing~$\script{D}_i$
	at a point along this line segment can be expressed as:
	\begin{subequations}\begin{align}
	L(\script{D}_i | j, q) &=
	\frac{1}{\sqrt{2\pi \det(C_i)}}
	\exp\left(\frac{-1}{2}\Delta_{ij} C_{i}^{-1} \Delta_{ij}^T\right)
	\\
	\Delta_{ij} &= \script{D}_i - \Delta\script{M}_{j,j + 1}(q)
	\\
	&= \script{D}_i - \script{M}_j - q(\script{M}_{j + 1} - \script{M}_j)
	\label{eq:delta_segment}
	\\
	&= \delta_{ij} - q m_j
	\end{align}\end{subequations}
	where~$C_i$ is the covariance matrix of the~$i$th datum~$\script{D}_i$
	and~$\Delta_{ij}$ is the difference between the location of~$\script{D}_i$
	and the point along the track~$\Delta\script{M}_{j,j + 1}(q)$ in the
	observed space.
	For notational convenience, we have made the substitution
	$\delta_{ij} = \script{D}_i - \script{M}_j$ denoting the vector difference
	between the datum~$\script{D}_i$ and the model point~$\script{M}_j$ and
	$m_j$ denoting the vector difference between the model points
	$\script{M}_{j + 1}$ and~$\script{M}_j$.

	\item To marginalize over the length of the line segment, we integrate
	this likelihood from~$q = 0$ to 1, but first, we compute the square and
	isolate the terms that depend on~$q$, which results in:
	\begin{equation}
	\label{eq:chi_squared_ij}
	\Delta_{ij}C_i^{-1}\Delta_{ij}^T = 
	\delta_{ij}C_i^{-1}\delta_{ij}^T - 2q \delta_{ij} C_i^{-1} m_j^T +
	q^2 m_j C_i^{-1} m_j^T
	\end{equation}
	Visual inspection of equation~\refp{eq:delta_segment} indicates that
	whenever the spacing between track points (i.e. $\script{M}_{j}$ and
	$\script{M}_{j + 1}$) is small compared to the observational uncertainties,
	then the second and third terms of equation~\refp{eq:chi_squared_ij} are
	negligible.
	In this case, the value of~$\Delta_{ij}C_i^{-1}\Delta_{ij}^T$ approaches
	the value obtained by simply taking the vector difference between the
	datum~$\script{D}_i$ and the track point~$\script{M}_j$.

	\item If we consider the full line segment, integrating from~$q = 0$ to 1:
	\begin{subequations}\begin{align}
	L(\script{D}_i | j) &= \int_0^1 L(\script{D}_i | j, q) dq
	\\
	\begin{split}
	&= \frac{1}{\sqrt{2\pi \det(C_i)}}
	\exp\left(\frac{-1}{2}\delta_{ij}C_i^{-1}\delta_{ij}^T\right)
	\\
	&\qquad \int_0^1 \exp\left(\frac{-1}{2}(aq^2 - 2bq)\right)dq
	\end{split}
	\\
	\begin{split}
	&= \frac{1}{\sqrt{2\pi \det(C_i)}}
	\exp\left(\frac{-1}{2}\delta_{ij}C_i^{-1}\delta_{ij}^T\right)
	\sqrt{\frac{\pi}{2a}} \exp\left(\frac{b^2}{2a}\right)
	\\
	&\qquad \left[\erf\left(\frac{a - b}{\sqrt{2a}}\right) -
	\erf\left(\frac{b}{\sqrt{2a}}\right)\right]
	\end{split}
	\\
	a &= m_j C_i^{-1} m_j^T
	\\
	b &= \delta_{ij} C_i^{-1} m_j^T
	\end{align}\end{subequations}
	For simplicity, we introduce the corrective term~$\beta_{ij}$ given by
	\begin{equation}
	\beta_{ij} = \sqrt{\frac{\pi}{2a}}
	\exp\left(\frac{b^2}{2a}\right)
	\left[
	\erf\left(\frac{a - b}{\sqrt{2a}}\right) -
	\erf\left(\frac{b}{\sqrt{2a}}\right)
	\right],
	\end{equation}
	such that~$L(\script{D}_i | j)$ is given by:
	\begin{equation}
	L(\script{D}_i | j) = \frac{\beta_{ij}}{\sqrt{2\pi \det(C_i)}}
	\exp\left(\delta_{ij}C_i^{-1}\delta_{ij}^T\right),
	\end{equation}
	and as long as the track is densely sampled relative to the observational
	uncertainties, then~$\beta_{ij} \approx 1$ and this term can be safely
	neglected.
	In some cases, computing the evolutionary track~$\script{M}$ may be
	computationally expensive, making it potentially advantageous to reduce the
	number of points computed in exchange for a slightly more complicated
	likelihood calculation.

	\item The remaining term in the integrand is the unnormalized density of
	points predicted by the model~$\rho$.
	As discussed above, this parameter quantifies the model predicted density
	of observed points, incorporating the intrinsic density as well as any
	selection effects present in the data.
	Based on equations~\refp{eq:rho_def} and~\refp{eq:lambda_def}, it can
	be written in terms of these two functions as
	\begin{equation}
	\rho(\script{M} | \{\theta\}) = \frac{1}{N_\lambda}
	\script{S}(\script{M}, \{\theta\})
	\Lambda(\script{M} | \{\theta\}).
	\end{equation}
	In a one-zone GCE model, the predicted intrinsic density~$\Lambda$ is given
	by the SFH, modulo the small effect of mass loss from recycled stellar
	envelopes~\citep[see discussion in, e.g., ][]{Weinberg2017}.
	In our use case of this likelihood function,~$\rho$ should therefore be
	proportional to the SFH, normalized according to
	equation~\refp{eq:rho_integrated} by:
	\begin{equation}
	\rho(\script{M} | \{\theta\}) = \ddfrac{
		\dot{M}_\star(t | \{\theta\})
	}{
		\int_0^T \dot{M}_\star(t | \{\theta\}) dt
	},
	\end{equation}
	where~$t$ denotes time in the GCE model, and~$T$ is the time interval over
	which the GCE model is integrated.

	\item This multiplicative factor on the likelihood~$L$ can be incorporated
	by simply letting the pair-wise component of the datum~$\script{D}_i$ and
	point on the model track~$\script{M}_j$ take on a weight~$w_j$ which is
	proportional to the product of the SFH and the selection function of the
	survey at the point~$\script{M}_j$ in the observed space.
	This gives rise to the following expression for the likelihood of observing
	the datum~$\script{D}_i$ given the point~$\script{M}_j$:
	\begin{equation}
	L(\script{D}_i | \script{M}_j) = \frac{w_j}{\sqrt{2\pi \det(C_i)}}
	\exp\left(\delta_{ij} C_i^{-1} \delta_{ij}^T\right).
	\end{equation}
	Plugging this into equation~\refp{eq:lnL_integral} yields the following
	expression:
	\begin{equation}
	\begin{split}
	\ln L(\script{D} | \{\theta\}, N_\lambda) &=
	-N_\lambda + N \ln N_\lambda +
	\\ &\qquad \sum_i^N \ln
	\left(\int_\script{M} \beta_{ij} w_j\exp(\delta_{ij} C_i^{-1} \delta_{ij}^T)
	d\script{M}\right)
	\\ &\qquad - \sum_i^N \ln \left(\sqrt{2\pi \det(C_i)}\right).
	\end{split}
	\end{equation}
	In the interest of optimizing the likelihood function, we compute the
	gradient of~$\ln L$ with respect to~$N_\lambda$ and find that it is equal
	to zero when~$N_\lambda = N$.
	Because the density~$\rho$ is by definition un-normalized, we can
	simply choose this normalization, after which the first two terms become
	$-N + N \ln N$, a constant for a given data set~\script{D} which can safely
	be neglected for the purposes of optimizing~$\ln L$.
	The final term is also a constant, allowing it to safely be neglected as
	well.
	Moreover, when the track is densely sampled relative to the observational
	uncertainties, then the corrective term~$\beta_{ij} \approx 1$ as discussed
	above, and the integral can be approximated as a summation over the points
	at which~\script{M} is sampled.
	This gives rise to the following expression for the likelihood function:
	\begin{subequations}\begin{align}
	\ln L(\script{D} | \{\theta\}) &\propto
	\sum_i \ln \left(\sum_j w_j
	\exp\left(\delta_{ij} C_i^{-1} \delta_{ij}^T\right)\right)
	\\
	\sum_j w_j &= 1.
	\end{align}\end{subequations}
	where the summations are taken over the entirety of the data~\script{D} and
	the track~\script{M}.

\end{itemize}

\end{document}

































