
\documentclass[ms.tex]{subfiles}
\begin{document}

\section{Conclusions}
\label{sec:conclusions}

\begin{itemize}

	\item Starting from first principles with an IPPP and an arbitrary
	astrophysical model which predicts an arbitrary evolutionary track in some
	observed space, we have developed a Bayesian method for fitting one-zone
	GCE models to observed stellar abundances and ages (where available).
	This derivation does not assume that the evolutionary track is predicted by
	a GCE model or that the observational benchmarks are stellar ages and
	abundances.
	It should therefore be easily extensible to other astrophysical models in
	which the chief prediction is a track of some form (e.g. stellar streams
	in kinematic space and isochrones in CMDs).
	
	\item The result of this process is a single exact form for the likelihood
	function given by equation~\refp{eq:likelihood}.
	Although there are a number of examples in the literature already in which
	the authors apply statistical likelihood calculations with GCE models to
	observed samples of stars~\citep[e.g.][]{Spitoni2020, Spitoni2021,
	Hasselquist2021}, to our knowledge this is the first of such instances in
	which the statistical validity of the likelihood function has been robustly
	demonstrated.
	There are two central features of this likelihood function which are of
	particular importance to ensure accuracy in the inferred best-fit
	parameters (see discussion in~\S~\ref{sec:fitting}).
	First, the likelihood of observing some datum~$\script{D}_i$ must be
	marginalized over the entire evolutionary track~\script{M}.
	This arises due to observational uncertainties: there is now way of knowing
	which point on the track~$\script{M}_j$ that~$\script{D}_i$ arose from,
	and to take this into account, one must consider all pair-wise combinations.
	Second, the likelihood of observing a datum~$\script{D}_i$ given a point
	on the track~$\script{M}_j$ must be weighted by the SFR at that time in
	the model, also incorporating any selection effects of the survey.
	This arises because an observed star is proportionally more likely to have
	been sampled from an epoch of a galaxy's history in which the SFR was large.
	Although it may be tempting to simply associate each datum~$\script{D}_i$
	with the closest point on the evolutionary track, the detailed form of the
	SFH means that this is not necessarily the most likely point from which
	$\script{D}_i$ arose.
	Unless both of these conditions discussed above are met, we are unable
	to recover the known parameters of our mock samples with inaccuracies
	at the many~$\sigma$ level.
	We therefore caution against the accuracy of GCE model fits which do not
	satisfy both criteria.

	\item We test the accuracy of this fitting method by means of mock data
	samples.
	To this end, we construct mock data sets by sampling some number of stars
	from the evolutionary history of one-zone models and perturbing the sampled
	ages and abundances with artificial uncertainties.
	Within the range of sample sizes ($N = 20 - 2000$), abundance uncertainties
	($\sigma_\text{[X/Y]} = 0.01 - 0.5$), age uncertainties
	($\sigma_{\log_{10}(\text{age})} = 0.02 - 1$), and the fraction of the
	sample with age information ($f_\text{age} = 0 - 1$) probed here, we find
	that this method recovers the known evolutionary parameters of the one-zone
	model which generated the mock samples in all cases (see Fig.
	\ref{fig:dp_sigma}).
	The mean deviation from the known parameters is at the~$\sim1\sigma$ level,
	exactly as expected for a Gaussian random process (see discussion
	in~\S~\ref{sec:mocks:variations}).
	We conduct further ``stress tests'' of this method by fitting models with
	complicated evolutionary histories which we elect not to illustrate in this
	paper.
	In all cases, our method accurately recovers the known parameters of the
	input model, reinforcing our interpretation that equation
	\refp{eq:likelihood} should be universally applicable to GCE models as it
	takes the model-predicted track as input and is independent of the details
	of the model from which it was computed.

	\item Intriguingly, we are able to derive accurate fits for evolutionary
	timescales even in the~\textit{absence} of age information.
	This arises through their impact on the normalization and shape of the
	observed MDF, and this information can be extracted with a fit to a sample
	of sufficient size and/or measurement precision.
	While age information does improve the precision of the fit, the inferred
	values remain accurate nonetheless even without such measurements.
	The duration of star formation in particular is independent of the
	yield-outflow degeneracy (see discussion in~\S~\ref{sec:mocks:fiducial_fit}
	and Appendix~\ref{sec:yield_outflow_degeneracy}), allowing precise values
	to be inferred regardless of the assumed absolute scale of nucleosynthetic
	yields and galactic outflows.
	This is of notable interest to authors interested in deriving quenching
	times (i.e. the lookback time to when star formation stopped) for dwarf
	galaxies
	At present, the most robust method for empirically measuring a galaxy's
	quenching time is to directly measure its SFH by some means, such as
	analysing its CMD~\citep[e.g.][]{Sohn2013, Weisz2015}.
	However, there are only a handful of well-constrained SFHs for quenched
	galaxies outside of the Milky Way subgroup (Andromeda II and Andromeda XVI:
	\citealp{Weisz2014a}; Cetus:~\citealp{Monelli2010a}; Tucana:
	\citealp{Monelli2010b}).
	By inferring the duration of star formation from stellar abundances and
	ages, these ``chemical quenching times'' could offer new and independent
	constraints on when star formation shuts off in the low stellar mass
	regime (see discussion in~\S~\ref{sec:mocks:variations} for discussion of
	the required sample size and measurement uncertainties for a given fit
	precision).

	% \item Provided that the one-zone approximation is valid for a given galaxy
	% and that a reasonable fit to its abundances can be obtained, our results
	% suggest that the duration of star formation, and consequently the quenching
	% time, can instead be inferred from the abundances of alpha and iron-peak
	% elemental abundances alone.
	% These ``chemical quenching times'' could offer new and independent
	% constraints on when star formation shuts off in the low stellar mass
	% regime.
	% With the precision of the fit in our mock samples scaling with sample size
	% approximately as~$N^{-0.5}$ and our~$f_\text{age} = 0$ variant re-deriving
	% the duration of star formation with~$\sim$30\% precision, this suggests
	% that with abundance uncertainties of~$\sigma_\text{[X/Y]} \approx 0.05$,
	% a sample size of~$N = 180$ is required to derive the quenching time with
	% $\sim$50\% precision, while precision on the order of unity requires only
	% $\sim$45 stars.

	\item We apply our fitting method to the GSE~\citep{Belokurov2018,
	Helmi2018} and Wukong~\citep{Naidu2020, Naidu2022} stellar streams using
	data from the H3 survey of the Milky Way halo~\citep{Conroy2019}.
	We find that the Wukong stream experience a more extended infall history
	than the GSE ($\tau_\text{in} = 3.47^{+3.58}_{-1.36}$ Gyr versus
	$\tau_\text{in} = 1.01 \pm 0.13$ Gyr), but experienced star formation
	quenching approximately 2 Gyr earlier ($\sim9.8$ Gyr ago versus
	$\sim$7.8 Gyr ago).
	The Wukong stream also experienced~$\sim$5 times stronger winds and
	$\sim$3 times slower star formation than the GSE, though the exact values
	of these parameters are subject to the yield-outflow degeneracy (see
	Appendix~\ref{sec:yield_outflow_degeneracy}), so only the relative values
	that we derive carry any physical meaning.
	These results point to the Wukong progenitor being significantly lower
	stellar mass than the GSE progenitor, as expected since the GSE is believed
	to be among the most massive satellites to have been accreted by the Milky
	Way~\citep{Myeong2018, Deason2019, Fattahi2019, Mackereth2019,
	Vincenzo2019}.
	Such an interpretation is also in qualitative agreement with semi-analytic
	models of galaxy formation~\citep[e.g.][]{Baugh2005, Baugh2006, Bower2006,
	Benson2012, Somerville2015a, Somerville2015b, Croton2016, Behroozi2019}
	which suggest that dwarf galaxies in the field experience more extended
	SFHs at lower stellar mass.

	\item These results highlight the wealth of information that can be
	extracted from the chemical abundances of stars and their ages.
	Although the sample sizes required to measure detailed quenching times of
	local group dwarf galaxies are large compared to the data that are
	available at present, future big data spectroscopic surveys such as the
	auxiliary telescope at the Vera C. Rubin Observatory\footnote{
		Formerly known as the Large Synoptic Survey Telescope (LSST).
	}~\citep{Ivezic2019} will produce considerably larger samples than
	previously compiled.
	Statistically robust methods such as this will be necessary to accurately
	derive evolutionary parameters to deepen our understanding of galaxy
	evolution at the low end of the mass distribution.

\end{itemize}

\end{document}
