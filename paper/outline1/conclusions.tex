
\documentclass[ms.tex]{subfiles}
\begin{document}

\section{Conclusions}
\label{sec:conclusions}

\begin{itemize}

	\item Starting from first principles with an IPPP and an arbitrary
	astrophysical model which predicts an arbitrary evolutionary track in some
	observed space, we have developed a Bayesian method for fitting one-zone
	GCE models to observed stellar abundances and ages (where available).
	This derivation does not assume that the evolutionary track is predicted by
	a GCE model or that the observational benchmarks are stellar ages and
	abundances.
	It should therefore be easily extensible to other astrophysical models in
	which the chief prediction is a track of some form (e.g. stellar streams
	in kinematic space and isochrones in CMDs).
	
	\item The result of this process is a single exact form for the likelihood
	function given by equation~\refp{eq:likelihood}.
	Although there are a number of examples in the literature already in which
	the authors apply statistical likelihood calculations with GCE models to
	observed samples of stars~\citep[e.g.][]{Spitoni2020, Spitoni2021,
	Hasselquist2021}, to our knowledge this is the first of such instances in
	which the statistical validity of the likelihood function has been robustly
	demonstrated.

	\item We test the accuracy of this fitting method by means of mock data
	samples.
	To this end, we construct mock data sets by sampling some number of stars
	from the evolutionary history of one-zone models and perturbing the sampled
	ages and abundances with artificial uncertainties.
	Within the range of sample sizes ($N = 20 - 2000$), abundance uncertainties
	($\sigma_\text{[X/Y]} = 0.01 - 0.5$), age uncertainties
	($\sigma_{\log_{10}(\text{age})} = 0.02 - 1$), and the fraction of the
	sample with age information ($f_\text{age} = 0 - 1$) probed here, we find
	that this method recovers the known evolutionary parameters of the one-zone
	model which generated the mock samples in all cases (see Fig.
	\ref{fig:dp_sigma}).
	The mean deviation from the known parameters is at the~$\sim1\sigma$ level,
	exactly as expected for a Gaussian random process.\footnote{
		The most likely value drawn from a normal distribution is exactly
		$1\sigma$ from the mean.
		The recovered values should also be consistent within~$1\sigma$ only
		68.2\% of the time and discrepant at the~$>1\sigma$ level the remaining
		31.8\% of the time.
	}

	\item Interestingly, we are able to derive accurate fits for evolutionary
	timescales even in the~\textit{absence} of age information.
	While age information does improve the precision of the fit, the inferred
	values remain accurate nonetheless even without such measurements.
	This could be of particular interest to authors interested in deriving
	quenching times for dwarf galaxies (i.e. the lookback time to when star
	formation stopped).
	At present, the most robust method for empirically measuring a galaxy's
	quenching time is to directly measure its SFH by some means, such as analysing its CMD~\citep[e.g.][]{Sohn2013, Weisz2015}.
	However, there are only a handful of well-constrained SFHs for quenched
	galaxies outside of the Milky Way subgroup (Andromeda II and Andromeda XVI:
	\citealp{Weisz2014a}; Ceuts:~\citealp{Monelli2010a}; Tucana:
	\citealp{Monelli2010b}).
	There has been some attention in the literature to constraining quenching
	times in N-body simulations (e.g.~\citealp{Phillips2014, Phillips2015};
	\citealp*{Rocha2012};~\citealp{Slater2013, Slater2014, Wheeler2014}), but
	this is a relatively new technique and the simulation outcomes are strongly
	dependent on the details of the adopted subgrid models
	\citep[see discussion in, e.g.,][]{Li2020}.

	\item Provided that the one-zone approximation is valid for a given galaxy
	and that a reasonable fit to its abundances can be obtained, our results
	suggest that the duration of star formation, and consequently the quenching
	time, can instead be inferred from the abundances of alpha and iron-peak
	elemental abundances alone.
	These ``chemical quenching times'' could offer new and independent
	constraints on when star formation shuts off in the low stellar mass
	regime.
	With the precision of the fit in our mock samples scaling with sample size
	approximately as~$N^{-0.5}$ and our~$f_\text{age} = 0$ variant re-deriving
	the duration of star formation with~$\sim$30\% precision, this suggests
	that with abundance uncertainties of~$\sigma_\text{[X/Y]} \approx 0.05$,
	a sample size of~$N = 180$ is required to derive the quenching time with
	$\sim$50\% precision, while precision on the order of unity requires only
	$\sim$45 stars.

	% \item There are two primary qualitative results of this derivation which we
	% find significantly impact the accuracy of the inferred best-fit parameters.
	\begin{itemize}
		\item It is of central importance to the accuracy of the inferred
		best-fit parameters that the likelihood of observing some
		datum~$\script{D}_i$ be marginalized over the~\textit{entire}
		evolutionary track~\script{M}.
		This arises due to observational uncertainties -- there is no way of
		knowing from which point on the track $\script{M}_j$ the
		datum~$\script{D}_i$ truly arose, and the proper manner in which to
		take this into account is to consider all pair-wise combinations of the
		data and sampled points on the track.
		
		\item Although it may be tempting to simply associate each
		datum~$\script{D}_i$ with the closest point~$\script{M}_j$ on the
		evolutionary track, the detailed form of the SFH means that this is not
		necessarily the most likely point from which~$\script{D}_i$ arose.
		The likelihood of observing~$\script{D}_i$ must be weighted by both the
		selection function of the survey and the model-predicted SFR as a
		function of position on the track.
		This arises because an observed star is proportionally more likely to
		have been sampled from an epoch of a galaxy's history in which the star
		formation rate was large.
	\end{itemize}

\end{itemize}

\end{document}
