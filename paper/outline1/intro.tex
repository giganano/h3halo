
\documentclass[ms.tex]{subfiles}
\begin{document}

\section{Introduction}
\label{sec:intro}

\begin{itemize}

	\item Dwarf galaxies provide a unique window into galaxy formation and
	evolution.
	Their intrinsic abundance~\citep{Bell2003, Baldry2012} allows the
	construction of statistical samples where resolved stars are available, but
	in practice, their low luminosities limit our scope to the local universe.
	Furthermore, field dwarfs have more extended star formation histories (SFHs)
	than more massive galaxies like the Milky Way and Andromeda
	\citep[e.g.][]{Behroozi2019}, while surviving satellites and stellar
	streams often have their star formation ``quenched'' by ram pressure
	stripping from the hot halo of their host
	\citep*[see discussion in, e.g.,][]{Steyrleithner2020}.
	As a result, surviving satellites and stellar stream progenitors assembled
	more of their stellar mass at high redshift and are dominated by old
	stellar populations.

	\item Because stellar age measurements are generally imprecise,
	particularly for old stars (refs), the SFHs of surviving satellites and
	stellar streams are difficult to constrain directly with star-by-star age
	measurements.
	An accurate SFH can instead be derived by fitting the color-magnitude
	diagram (CMD) with a composite set of theoretical isochrones for simple
	stellar populations~\citep[e.g.][]{Dolphin2002}.
	\citet{Weisz2014} demonstrate that this method is accurate even when the
	CMD does not extend to the oldest main sequence turnoff stars, but that
	this is critical for precisely constraining the early epochs of the SFH.

	\item Aside from these CMD-derived SFHs, which can be especially difficult
	for distant systems, chemical abundances encode additional information on a
	galaxy's SFH, including the earliest epochs.
	A star is believed to have the same chemical composition as its natal
	cloud, and this hypothesis holds up against spectroscopic abundance
	measurements in open clusters (refs).
	The chemical composition of a star is therefore a snapshot of the
	gas-phase composition of its host galaxy at its time of formation, allowing
	inferences of the galaxy's evolutionary history by comparing a statistical
	sample of stellar abundances to galactic chemical evolution (GCE) models.

	\item In this paper, we aim to systematically assess the information
	content of stellar abundances in dwarf galaxies by means of one-zone GCE
	models (see discussion in~\S~\ref{sec:onezone}).
	The fundamental assumption of these models is that the mixing of newly
	produced metals throughout the star forming reservoir is fast compared to
	the depletion time.
	Dwarf galaxies are typically gas-rich systems (refs), implying long star
	formation timescales and consequently long depletion times.
	It is therefore reasonable to expect that the one-zone approximation would
	be valid for these systems, but we stress that the accuracy of the methods
	we outline in this paper are contingent on this condition for any given
	galaxy.
	In larger galaxies such as the Milky Way or Andromeda, mixing timescales
	are long, and processes such as the radial migration of stars
	(\citealp{Sellwood2002, Schoenrich2009, Minchev2011};~\citealp*{Minchev2013,
	Minchev2014};~\citealp{Minchev2017, Johnson2021, Chen2022}) and radial gas
	flows~\citep{Lacey1985, Bilitewski2012, Vincenzo2018} are expected to
	significantly impact the observed abundance distribution in a given
	galactic region.

	\item One-zone models are computationally cheap, and with reasonable
	approximations, even allow analytic solutions to the evolution of the
	abundances for simple SFHs~\citep*[e.g.][]{Weinberg2017}.
	This in principle allows them to be fit to data assuming some
	parametrization of the galaxy's evolutionary history, and there are both
	simple and complex examples of how one might go about deriving best-fit
	parameters in the literature.
	\citet{Kirby2011} measure and fit the MDFs of eight Milky Way dwarf
	satellite galaxies with the goal of determining which evolved according to
	``leaky-box,'' ``pre-enriched,'' or ``extra gas'' analytic models.
	To derive best-fit parameters for the two-infall model of the Milky
	Way~\citep[e.g.][]{Chiappini1997},~\citet{Spitoni2020} use Markov chain
	Monte Carlo (MCMC) methods for solar neighbourhood stars and base their
	likelihood function off of the minimum distance between each star and the
	evolution track in the~\afe-\feh~plane.

	\item Here we develop an approach which is widely applicable as long as the
	one-zone approximation is valid and requires no binning of the data.
	We demonstrate by means of mock samples that it is accurate with as few
	as~$\sim$20 stars (our smallest test case; see~\S~\ref{sec:mocks}), even
	when no age information is available.
	This method treats the abundances of stars as a sample from the galaxy's
	evolutionary history according to an inhomogeneous poisson point process
	(IPPP) and only assumes that the model predicts some infinitely thin track
	in some observed space.
	It should therefore be extensible to other astrophysical models which also
	predict tracks, such as stellar streams in kinematic space and isochrones
	on CMDs.

	\item We provide an overview of the equations governing one-zone models
	in~\S~\ref{sec:onezone} and discussion of our fitting method
	in~\S~\ref{sec:fitting} below, with a detailed derivation of our
	likelihood function in Appendix~\ref{sec:l_derivation}.
	In~\S~\ref{sec:mocks} we demonstrate the accuracy of our method with mock
	data and quantify how the precision of the derived parameters are affected
	by the sample size, measurement uncertainty in abundances and ages, and
	what fraction of the sample has available age measurements.
	In~\S~\ref{sec:h3} we apply this method to two systems which have received
	considerable attention in the literature over the past few years using data
	from the H3 survey~\citep{Conroy2019}: the Gaia-Sausage Encelaudus (GSE)
	and the Sagittarius (Sgr) dwarf Spheroidal (dSph) satellite of the Milky
	Way.

\end{itemize}

\end{document}
