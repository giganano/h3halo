
\documentclass[ms.tex]{subfiles}
\begin{document}

\section{Introduction}
\label{sec:intro}

\begin{itemize}

	\item The chemical abundances of stars contain information on the
	evolutionary history of their host galaxy, but how precisely can this
	information be derived from abundances?
	How does this depend on factors like the size of the sample, measurement
	precision, and the availability (or non-availability) of age information?

	\item One-zone models of GCE are computationally cheap, allowing them to be
	fit to data assuming some parametrization for the galaxy's evolutionary
	history.
	This are both simple and complex examples of how one might go about
	obtaining these measurements.
	\citet{Kirby2011} measure and fit the MDFs of eight Milky Way dwarf
	satellite galaxies with the goal being to determine which evolve according
	to ``leaky box,'' ``pre-enriched,'' or ``extra gas'' analytic models.
	\citet{Spitoni2020} use MCMC methods for solar neighbourhood stars and
	base their likelihood function off of the minimum distance between each
	star in their sample and the evolutionary track in the~\afe-\feh~plane to
	derive best-fit parameters for the two-infall model of the Milky Way.

	\item Here we develop an approach which is universally applicable, requires
	no binning of the data, and is accurate with as few as~$\sim$20 stars even
	when there is no age information available.
	We treat the abundances of stars as a sample from the galaxy's evolutionary
	history according to an inhomogeneous poisson point process (IPPP).
	We demonstrate by means of mock data that this method accurately recovers
	evolutionary parameters provided that the sample is well-described by a
	one-zone model.

\end{itemize}

\end{document}
