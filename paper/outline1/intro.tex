
\documentclass[ms.tex]{subfiles}
\begin{document}

\section{Introduction}
\label{sec:intro}

\begin{itemize}

	\item Dwarf galaxies provide a unique window into galaxy formation and
	evolution.
	Their intrinsic abundance~\citep{Bell2003, Baldry2012} allows the
	construction of statistical samples where resolved stars are available, but
	in practice, their low luminosities limit our scope to the local universe.
	Furthermore, field dwarfs have more extended star formation histories (SFHs)
	than more massive galaxies like the Milky Way and Andromeda
	\citep[e.g.][]{Behroozi2019}, while surviving satellites and stellar
	streams often have their star formation ``quenched'' by ram pressure
	stripping from the hot halo of their host
	\citep*[see discussion in, e.g.,][]{Steyrleithner2020}.
	As a result, surviving satellites and stellar stream progenitors assembled
	much of their stellar mass at high redshift, making it difficult to
	constrain this process observationally.

	\item Furthermore, stellar age measurements are generally imprecise, and no
	single method is applicable to all ranges of spectral types across the full
	range of ages (see, e.g., the reviews in~\citealp{Soderblom2010} and
	\citealp{Chaplin2013} as well as the discussion in~\citealp{Angus2019}).
	Consequently, the SFHs of surviving satellites and stellar streams are
	difficult to measure even with star-by-star age measurements.
	Accurate SFHs can instead be derived by fitting the observed
	color-magnitude diagram (CMD) with a composite set of theoretical
	isochrones for simple stellar populations~\citep[e.g.][]{Dolphin2002}.
	\citet{Weisz2014b} demonstrate that this method is accurate even when the
	CMD does not extend to oldest main sequence turnoff stars, but that those
	populations are critical for precisely constraining the earliest epochs of
	the SFH.
	Nonetheless, CMD-derived SFHs can be especially difficult for distant
	systems as the observational uncertainties increase and the main sequence
	turnoff of progressively younger populations are found below the magnitude
	limit of the survey.

	\item Alternatively, chemical abundances can offer independent constraints
	on the evolutionary histories of dwarf galaxies, including the earliest
	epochs of star formation.
	Stars are believed to be born with the same chemical composition of their
	natal molecular clouds, and this hypothesis holds up against spectroscopic
	abundance measurements in open clusters which have demonstrated that FGK
	main-sequence and red giant stars exhibit chemical homogeneities within
	$\sim0.02 - 0.03$ dex~\citep{DeSilva2006, Liu2016b, Bovy2016,
	Casamiquela2020} while inhomogeneities at the~$\sim0.1 - 0.2$ dex level can
	be attributed to diffusion~\citep{BertelliMotta2018, Souto2019, Liu2019}
	or planet formation~\citep{Melendez2009, Liu2016a, Spina2018}.
	A star's detailed metal abundance is therefore a snapshot of the gas-phase
	composition of the galactic environment it was born from, encoding
	information on the nuclear reactions -- and by proxy the stars and
	supernovae which facilitated those reactions -- that polluted the gas since
	big bang nucleosynthesis.

	\item This is the basis of galactic chemical evolution (GCE), which bridges
	the gap between nuclear physics and astrophysics by combining galactic
	processes such as star formation with nuclear reaction networks to estimate
	the production rates of various nuclear species by stars and derive their
	abundances in the interstellar medium (ISM).
	GCE has its roots in the early works assessing the astrophysical sites and
	conditions which give rise to nuclear reactions, a review of which can be
	found in~\citet[][the famous ``B2FH'' paper]{Burbidge1957}.
	The simplest and most well-studied GCE models are called ``one-zone''
	models, reviews of which can be found in works such as~\citet{Tinsley1980},
	\citet{Pagel2009}, and~\citet{Matteucci2012, Matteucci2021}.

	% \item Because stellar age measurements are generally imprecise,
	% particularly for old stars (refs), the SFHs of surviving satellites and
	% stellar streams are difficult to measure directly with star-by-star age
	% measurements.
	% An accurate SFH can instead be derived by fitting the color-magnitude
	% diagram (CMD) with a composite set of theoretical isochrones for simple
	% stellar populations~\citep[e.g.][]{Dolphin2002}.
	% \citet{Weisz2014} demonstrate that this method is accurate even when the
	% CMD does not extend to the oldest main sequence turnoff stars, but that
	% this is critical for precisely constraining the early epochs of the SFH.

	% \item Aside from these CMD-derived SFHs, which can be especially difficult
	% for distant systems, chemical abundances encode additional information on a
	% galaxy's SFH, including the earliest epochs.
	% A star is believed to have the same chemical composition as its natal
	% cloud, and this hypothesis holds up against spectroscopic abundance
	% measurements in open clusters (refs).
	% The chemical composition of a star is therefore a snapshot of the
	% gas-phase composition of its host galaxy at its time of formation, allowing
	% inferences of the galaxy's evolutionary history by comparing a statistical
	% sample of stellar abundances to galactic chemical evolution (GCE) models.

	% \item GCE bridges the gap between nuclear physics and astrophysics,
	% combining galactic processes such as star formation with nuclear reaction
	% networks, one can estimate the production rates of various nuclear species
	% by stars and derive their abundance in the interstellar medium (ISM).
	% This field has its roots in the early works assessing the astrophysical
	% sites and conditions which give rise to nuclear reactions, a review of
	% which can be found in the famous ``B2FH'' paper~\citep{Burbidge1957}.
	% The simplest and most well-studied GCE models are called ``one-zone''
	% models, reviews of which can be found in works such as~\citet{Tinsley1980},
	% \citet{Pagel2009}, and~\citet{Matteucci2012, Matteucci2021}.

	\item In this paper, we systematically assess the information that can be
	extracted from the abundances and ages of stars in dwarf galaxies when
	modeling the data within this framework.
	One-zone models are computationally cheap, and with reasonable
	approximations, even allow analytic solutions to the evolution of the
	abundances for simple SFHs~\citep*[e.g.][]{Weinberg2017}.
	In principle this allows the application of statistical likelihood
	estimates to derive best-fit parameters for some set of assumptions
	regarding the galaxy's evolutionary history.
	There are both simple and complex examples in the literature of how one
	might go about this.
	To name a few,~\citet{Kirby2011} measure and fit the MDFs of eight Milky
	Way dwarf satellite galaxies with the goal of determining which evolved
	according to ``leaky-box,'' ``pre-enriched,'' or ``extra gas'' analytic
	models.
	To derive best-fit parameters for the two-infall model of the Milky
	Way~\citep[e.g.][]{Chiappini1997},~\citet{Spitoni2020, Spitoni2021} use
	Markov chain Monte Carlo (MCMC) methods and base their likelihood function
	off of the minimum distance between each star and the evolutionary track in
	the~\afe-\feh~plane.
	\citet{Hasselquist2021} used similar methods to derive evolutionary
	parameters for the Milky Way's most massive satellites with the
	\textsc{FlexCE}~\citep{Andrews2017} and the~\citet{Lian2018, Lian2020}
	chemical evolution codes.

	\item While these studies have employed various methods to estimate the
	relative likelihood of different parameter choices, to our knowledge the
	literature is lacking a demonstrationg of the statistical validity of these
	estimates.
	The distribution of stars in abundance space is generally non-uniform, and
	the probability of randomly sampling a star from a given epoch of some
	galaxy's evolution scales with the star formation rate (SFR) at that time,
	modulo the selection function of the survey.
	Therefore, to model the enrichment history of a galaxy in a statistically
	robust manner, GCE models by nature demand an explanation according to an
	inhomogeneous poisson point process (IPPP; see, e.g.,~\citealp{Press2007}).
	To this end, in this paper we apply the principles of an IPPP to an
	arbitrary model-predicted track in some observed space to derive a
	likelihood function.
	This derivation does not assume that the track was predicted by a GCE model,
	and it should therefore be easily extensible to other astrophysical models
	which predict evolutionary tracks such as stellar streams in kinematic
	space of isochrones on CMDs, though here we limit our discussion to our use
	case of one-zone GCE models.
	We then demonstrate this method in application to two stellar streams in
	the Milky Way halo using data from the H3 survey (Hectochelle in the Halo
	at High resolution;~\citealp{Conroy2019}).
	One has received a considerable amount of attention in the literature: the
	Gaia-Sausage Enceladus~\citep[GSE;][]{Belokurov2018, Helmi2018}, and
	the other, discovered more recently, is a less deeply studied system: the
	Wukong stream~\citep{Naidu2020, Naidu2022}.

	% models
	% which predict ``tracks'' in some observed space as in GCE models to derive a likelihood
	% function which requires no binning of the data and should be applicable as
	% long as the one-zone approximation is valid.
	% We present a detailed derivation of this likelihood function in Appendix
	% \ref{sec:l_derivation} and present qualitative discussion
	% in~\S~\ref{sec:fitting}.
	% Because this derivation assumes only that the model predicts an
	% evolutionary track in some observed space, it should therefore be
	% extensible to other astrophysical models which also predict evolutionary
	% tracks, such as stellar streams in kinematic space and isochrones on CMDs,
	% though we limit our discussion in the present paper to our use case of
	% one-zone GCE models.

	% \item This procedure results in a single~\textit{exact} form of the
	% likelihood function.
	% We demonstrate by means of mock samples that this likelihood function is
	% accurate with as few as~$\sim$20 stars (our smallest test case;
	% see~\S~\ref{sec:mocks}) even when no age information is available.
	% We then apply this method to two stellar streams observed in the Milky Way
	% halo using data from the H3 survey~\citep{Conroy2019}.
	% One has received a considerable amount of attention in the literature: the
	% Gaia-Sausage Enceladus~\citep[GSE;][]{Belokurov2018, Helmi2018}, and
	% the other, discovered more recently, is a less well-studied system: the
	% Wukong stream~\citep{Naidu2020, Naidu2022}.

	% \item Here we develop an approach which is widely applicable as long as the
	% one-zone approximation is valid and requires no binning of the data.
	% We demonstrate by means of mock samples that it is accurate with as few
	% as~$\sim$20 stars (our smallest test case; see~\S~\ref{sec:mocks}), even
	% when no age information is available.
	% This method treats the abundances of stars as a sample from the galaxy's
	% evolutionary history according to an inhomogeneous poisson point process
	% (IPPP) and only assumes that the model predicts some infinitely thin track
	% in some observed space.
	% It should therefore be extensible to other astrophysical models which also
	% predict tracks, such as stellar streams in kinematic space and isochrones
	% on CMDs.

	% \item We provide an overview of the equations governing one-zone models
	% in~\S~\ref{sec:onezone} and discussion of our fitting method
	% in~\S~\ref{sec:fitting} below, with a detailed derivation of our
	% likelihood function in Appendix~\ref{sec:l_derivation}.
	% In~\S~\ref{sec:mocks} we demonstrate the accuracy of our method with mock
	% data and quantify how the precision of the derived parameters are affected
	% by the sample size, measurement uncertainty in abundances and ages, and
	% what fraction of the sample has available age measurements.
	% In~\S~\ref{sec:h3} we apply this method to two systems which have received
	% considerable attention in the literature over the past few years using data
	% from the H3 survey~\citep{Conroy2019}: the Gaia-Sausage Encelaudus (GSE)
	% and the Sagittarius (Sgr) dwarf Spheroidal (dSph) satellite of the Milky
	% Way.

\end{itemize}

\end{document}
