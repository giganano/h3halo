
\documentclass[ms.tex]{subfiles}
\begin{document}

\section{One-Zone Models of Galactic Chemical Evolution}
\label{sec:onezone}

\begin{itemize}

	\item The fundamental assumption of one-zone models is that newly produced
	metals mix instantaneously throughout the star forming gas reservoir.
	This approximation is valid as long as the mixing time-scale is negligible
	compared to the depletion time-scale (i.e. the average time an fluid
	element remains in the ISM before getting incorporated into new stars or
	ejected in an outflow).
	Based on the observations of~\citet{Leroy2008},~\citet*{Weinberg2017}
	calculate that characteristic depletion times can range from~$\sim$500 Myr
	up to~$\sim$10 Gyr for conditions in typical star forming disc galaxies.
	With the short length-scales and turbulent velocities of dwarf galaxies,
	instantaneous mixing should be a good approximation

	\begin{itemize}
		\item {\color{red}
		If there's an observational reference of metal-mixing in the dwarf
		galaxy regime - specifically if the scatter in the~\afe-\feh~plane is
		dominated by observational uncertainty - Evan Kirby would probably be
		the one to know about it.
		If not, this would be a good thing to call out as a good observational
		test of the validity of the one-zone approximation.
		}
	\end{itemize}

\end{itemize}

\end{document}

