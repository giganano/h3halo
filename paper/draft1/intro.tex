
\documentclass[ms.tex]{subfiles}
\begin{document}

\section{Introduction}
\label{sec:intro}

Dwarf galaxies provide a unique window into galaxy formation and evolution.
In the local universe, dwarfs can be studied in detail using resolved stellar
populations across a wide range of mass, morphology and star formation history
(SFH).
Field dwarfs have more extended SFHs than more massive galaxies like the Milky
Way and Andromeda~\citep[e.g.,][]{Behroozi2019, GarrisonKimmel2019}, while
surviving satellites and stellar streams often have their star formation
``quenched'' by ram pressure stripping from the hot halo of their host
\citep*[see discussion in, e.g.,][]{Steyrleithner2020}.
As a result, disrupted dwarf galaxies assembled much of their stellar mass at
high redshift, but their resolved stellar populations encode a wealth of
information on their progenitor's evolutionary history.
\par
Photometrically, one can constrain the SFH by fitting the observed
color-magnitude diagram (CMD) with a composite set of theoretical isochrones
\citep[e.g.,][]{Dolphin2002, Weisz2014b}.
The CMD also offers constraints on the metallicity distribution function
(MDF;~\citealp*[e.g.,][]{Lianou2011}).
In some cases, the MDF can also be constrained with narrow-band imaging
\citep{Fu2022}, especially when combined with machine learning algorithms
trained on spectroscopic measurements as in~\citet{Whitten2011}.
Depending on the limiting magnitude of the survey and the evolutionary stages
of the accessible stars, it may or may not be feasible to estimate ages on a
star-by-star basis.
When these measurements are made spectroscopically, however, multi-element
abundance information becomes avaialable, and age estimates become more precise
by pinning down various stellar parameters such as effective temperatures and
surface gravities.
\par
Chemcial abundances in a dwarf galaxy can also offer independent constraints
on the evolutionary histories of dwarf galaxies, including the earliest epochs
of star formation.
Stars are born with the same composition as their natal molecular clouds --
spectroscopic abundance measurements in open clusters have demonstrated that
FGK main-sequence and red giant stars exhibit chemical homogeneities
within~$\sim$$0.02 - 0.03$ dex~\citep{DeSilva2006, Bovy2016, Liu2016b,
Casamiquela2020} while inhomogeneities at the~$\sim$$0.1 - 0.2$ dex level can
be attributed to diffusion~\citep{BertelliMotta2018, Liu2019, Souto2019} or
planet formation~\citep{Melendez2009, Liu2016a, Spina2018}.
A star's detailed metal content is therefore a snapshot of the galactic
environment that it formed from.
This connection is the basis of galactic chemical evolution (GCE), which
bridges the gap between nuclear physics and astrophysics by combining galactic
processes such as star formation with nuclear reaction networks to estimate the
production rates of various nuclear species by stars and derive their
abundances in the intertsellar medium (ISM).
A GCE model that accurately describes the observed abundances can offer
conclusions regarding key parameters of the galaxy's evolutionary history, such
as the star formation and accretion histories and their durations, the
efficiency of outflows, and the origin of the observed abundance patterns.
\par
In this paper, we systematically assess the information that can be extracted
from the abundances and ages of stars in dwarf galaxies when modelling the
data in this framework.
The simplest and most well-studied GCE models are called ``one-zone'' models,
reviews of which can be found in works such as~\citet{Tinsley1980},
\citet{Pagel2009} and \citet{Matteucci2012, Matteucci2021}.
One-zone models are computationally cheap, and with reasonable approximations,
even allow analytic solutions to the evolution of the abundances for simple
SFHs~\citep*[e.g.,][]{Weinberg2017}.
This low expense expedites the application of statistical likelihood estimates
to infer best-fit parameters for some set of assumptions regarding a galaxy's
evolutionary history.
There are both simple and complex examples in the literature of how one might
go about these calculations.
For example,~\citet{Kirby2011} measure and fit the MDFs of eight Milky Way
dwarf satellite galaxies with the goal of determining which evolved according
to ``leaky-box,'' ``pre-enriched'' or ``extra-gas'' analytic models.
To derive best-fit parameters for the two-infall model of the Milky Way disc
\citep[e.g.,][]{Chiappini1997},~\citet{Spitoni2020, Spitoni2021} use Markov
chain Monte Carlo (MCMC) methods and base their likelihood function off of the
minimum distance between each star and the evolutionary track in
the~\afe-\feh\footnote{
	We follow the conventional definition in which
	[X/Y]~$\equiv \log_{10}(N_\text{X} / N_\text{Y}) -
	\log_{10}(N_{\text{X},\odot} / N_{\text{Y},\odot})$
	is the logarithmic difference in the abundance ratio of the nuclear species
	X and Y between some star and the sun.
} plane.
\citet{Hasselquist2021} used similar methods to derive evolutionary parameters
for the Milky Way's most massive satellites with the~\textsc{FlexCE}
\citep{Andrews2017} and the~\citet{Lian2018, Lian2020} chemical evolution
codes.
\par
While these studies have employed various methods to estimate the relative
likelihood of different parameter choices, to our knowledge there is no
demonstration of the statistical validity of these methods in the literature.
The distribution of stars in abundance space is generally non-uniform, and the
probability of randomly selecting a star from a given epoch of some galaxy's
evolution scales with the star formation rate (SFR) at that time (modulo the
selection function of the survey).
Describing the enrichment history of a galaxy as a one-zone model casts the
observed stellar abundances as a stochastic sample from the predicted
evolutionary track, a process which proceeds mathematically according to an
\textit{inhomogeneous poisson point process} (IPPP; see, e.g.,
\citealt{Press2007}).
To this end, we apply the principles of an IPPP to an arbitrary model-predicted
track in some observed space.
We demonstrate that this combination results in the derivation of a single
likelihood function which is required to ensure the accuracy of best-fit
parameters.
Our derivation does not assume that the track was predicted by a GCE model,
and it should therefore be easily extensible to other astrophysical models
which predict evolutionary tracks in some observed space, such as stellar
streams in kinematic space or isochrones on CMDs.
We however limit our discussion in this paper to our use case of one-zone GCE
models.
\par
After discussing the one-zone model framework in~\S~\ref{sec:onezone} and
our fitting method in~\S~\ref{sec:fitting}, we establish the accuracy of this
likelihood function by means of tests against mock data in~\S~\ref{sec:mocks},
simultaneously exploring how the precision of inferred parameters is affected
by sample size, measurement uncertainties and the portion of the sample that
has age information.
These methods are able to reconstruct the SFHs of dwarf galaxies because the
GCE framework allows one to convert the number of stars versus metallicity into
the number of stars versus time.
Abundance ratios such as~\afe~quantify the relative importance of type Ia
supernova (SN Ia) enrichment, and constraints on its associated delay-time
distribution (DTD) setting an overall timescale.
In~\S~\ref{sec:h3}, we then demonstrate our method in action by modelling two
disrupted dwarf galaxies in the Milky Way halo.
One has received a considerable amount of attention in the literature: the
\gaia-Sausage Enceladus (GSE;~\citealp{Belokurov2018, Helmi2018}), and the
other, discovered more recently, is a less deeply studied system: Wukong
\citep{Naidu2020, Naidu2022}, independently discovered and given the
alternative name of LMS-1 by~\citet{Yuan2020}.


\end{document}
