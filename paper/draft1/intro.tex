
\documentclass[ms.tex]{subfiles}
\begin{document}

\section{Introduction}
\label{sec:intro}

Dwarf galaxies provide a unique window into galaxy formation and evolution.
Their intrinsic abundance~\citep{Bell2003, Baldry2012} allows the construction
of statistical samples where resolved stars are available, but in practice,
their low luminosities limit our scope to the local universe.
Furthermore, field dwarfs have more extended star formation histories (SFHs)
than more massive galaxies like the Milky Way and Andromeda
\citep[e.g.][]{Behroozi2019}, while surviving satellites and stellar streams
often have their star formation ``quenched'' by ram pressure stripping from the
hot halo of their host~\citep*[see discussion in, e.g.,][]{Steyrleithner2020}.
As a result, surviving satellites and stellar stream progenitors assembled much
of their stellar mass at high redshift, making it difficult to constrain this
process observationally.
\par
Furthermore, stellar age measurements are generally imprecise, and no single
method is applicable to all ranges of spectral types across the full range of
ages (\citealp{Angus2019}; see also the reviews in~\citealp{Soderblom2010}
and~\citealp{Chaplin2013}).
Consequently, the SFHs of surviving satellites and stellar streams are
difficult to measure even with star-by-star age measurements.
Accurate SFHs can instead by derived by fitting the observed color-magnitude
diagram (CMD) with a composite set of theoretical isochrones for simple
stellar populations~\citep[e.g.][]{Dolphin2002}.
\citet{Weisz2014b} demonstrate that this method is accurate even when the CMD
does not extend to the oldest main sequence turnoff stars, but that those
populations are critical for precisely constraining the earliest epochs of the
SFH.
Nonetheless, CMD-derived SFHs can be especially difficult for distant systems
as the observational uncertainties increase and the main sequence turnoff of
progressively younger populations are found below the magnitude limit of the
survey.
\par
Alternatively, chemical abundances can offer independent constraints on the
evolutionary histories of dwarf galaxies, including the earliest epochs of
star formation.
Stars are believed to be born with the same chemical composition as their natal
molecular clouds, and this hypothesis holds up against spectroscopic abundance
measurements in open clusters which have demonstrated the FGK main-sequence and
red giant stars exhibit chemical homogeneities within~$\sim0.02 - 0.03$ dex
\citep{DeSilva2006, Bovy2016, Liu2016b, Casamiquela2020} while inhomogeneities
at the~$\sim0.1 - 0.2$ dex level can be attributed to diffusion
\citep{BertelliMotta2018, Liu2019, Souto2019} or planet formation
\citep{Melendez2009, Liu2016a, Spina2018}.
A star's detailed metal abundance is therefore a snapshot of the gas-phase
composition of the galactic environment it was born from, encoding information
on the nuclear reactions -- and by proxy the stars and supernovae which
facilitated those reaction -- that polluted the gas since big bang
nucleosynthesis.
\par
This connection is the basis of galactic chemical evolution (GCE), which
bridges the gap between nuclear physics and astrophysics by combining galactic
processes such as star formation with nuclear reaction networks to estimate the
production rates of various nuclear species by stars and derive their
abundances in the intertsellar medium (ISM).
GCE has its roots in the early works assessing the astrophysical sites and
conditions which give rise to nuclear reactions, a review of which can be
found in~\citep[][the famous ``B2FH'' paper]{Burbidge1957}.
The simplest and most well-studied GCE models are called ``one-zone'' models,
reviews of which can be found in works such as~\citet{Tinsley1980},
\citet{Pagel2009} and~\citet{Matteucci2012, Matteucci2021}.
\par
In this paper, we systematically assess the information that can be extracted
from the abundances and ages of stars in dwarf galaxies when modelling the
data in this framework.
One-zone models are computationally cheap, and with reasonable approximations,
even allow analytic solutions to the evolution of the abundances for simple
SFHs~\citep*[e.g.][]{Weinberg2017}.
This low expense expedites the application of statistical likelihood estimates
to infer best-fit parameters for some set of assumptions regarding a galaxy's
evolutionary history.
There are both simple and complex examples in the literature of how one might
go about these calculations.
To name a few,~\citet{Kirby2011} measure and fit the MDFs of eight Milky Way
dwarf satellite galaxies with the goal of determining which evolved according
to ``leaky-box,'' ``pre-enriched'' or ``extra-gas'' analytic models.
To derive best-fit parameters for the two-infall model of the Milky Way disc
\citep[e.g.][]{Chiappini1997},~\citet{Spitoni2020, Spitoni2021} use Markov
chain Monte Carlo (MCMC) methods and base their likelihood function off of the
minimum distance between each star and the evolutionary track in
the~\afe-\feh~plane.
\citet{Hasselquist2021} used similar methods to derive evolutionary parameters
for the Milky Way's most massive satellites with the~\textsc{FlexCE}
\citep{Andrews2017} and the~\citet{Lian2018, Lian2020} chemical evolution
codes.
\par
While these studies have employed various methods to estimate the relative
likelihood of different parameter choices, to our knowledge there is no
demonstration of the statistical validity of these methods in the literature.
The distribution of stars in abundance space is generally non-uniform, and the
probability of randomly selecting a star from a given epoch of some galaxy's
evolution scales with the star formation rate (SFR) at that time (modulo the
selection function of the survey).
Describing the enrichment history of a galaxy as a one-zone model casts the
observed stellar abundances as a stochastic sample from the predicted
evolutionary track, a process which proceeds mathematically according to an
\textit{inhomogeneous poisson point process} (IPPP; see, e.g.,
\citealt{Press2007}).
To this end, we apply the principles of an IPPP to an arbitrary model-predicted
track in some observed space.
We demonstrate that this combination results in the derivation of a single
likelihood function which is required to ensure the accuracy of best-fit
parameters.
Our derivation does not assume that the track was predicted by a GCE model,
and it should therefore be easily extensible to other astrophysical models
which predict evolutionary tracks in some observed space, such as stellar
streams in kinematic space or isochrones on CMDs.
We however limit our discussion in this paper to our use case of one-zone GCE
models.
We establish the accuracy of this likelihood function by means of tests
against mock data and explore how the precision of inferred parameters is
affected by sample size, measurement uncertainties and what portion of the
sample has age information.
We then demonstrate this method in application to two stellar streams in the
Milky Way halo using data from the H3 survey (Hectochelle in the Halo at High
resolution;~\citealp{Conroy2019}).
One has received a considerable amount of attention in the literature: the
Gaia-Sausage Enceladus (GSE;~\citealp{Belokurov2018, Helmi2018}), and the
other, discovered more recently, is a less deeply studied system: the Wukong
stream~\citep{Naidu2020, Naidu2022}.

\end{document}
