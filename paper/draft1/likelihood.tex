
\documentclass[ms.tex]{subfiles}
\begin{document}
\renewcommand\theequation{\thesection\arabic{equation}}
\renewcommand\thefigure{\thesection\arabic{figure}}
\setcounter{equation}{0}
\setcounter{figure}{0}

\section{Derivation of the Likelihood Function}
\label{sec:likelihood}

Here we provide a detailed derivation of our likelihood function (equation
\ref{eq:likelihood}) incorporating the principles of an IPPP.
We make no assumptions about the underlying model other than that it predicts
a track in some observed space.
In our use case, this corresponds to the~\afe-\feh-age space, though it could
be extended to higher dimensional data.
This approach should also be easily extensible to other astrophysical models
which predict tracks in some observed space, such as stellar streams and
isochrones.
\par
Given some expression, whether analytic or numerical, for the model predicted
track in the observable space~\script{M}, the likelihood of observing the data
given some set of model parameters~$\{\theta\}$ can be expressed as the
integrated differential likelihood along the track:
\begin{equation}
L(\script{D} | \{\theta\}) = \int_\script{M} dL =
\int_\script{M} L(\script{D} | \script{M}) P(\script{M} | \{\theta\})
d\script{M},
\end{equation}
where~$P(\script{M} | \{\theta\})$ describes the probability that a singular
datum will be drawn from the model at a given point along the track.
The defining characteristic of the IPPP is that~$P(\script{M} | \{\theta\})$
follows a Poisson distribution and can be related to the~\textit{intensity
function}~$\lambda$ quantifying the density of points according to
\begin{equation}
P(\script{M}_j | \{\theta\}) =
e^{-N_\lambda} \prod_i^N \lambda(\script{M}_j | \{\theta\}),
\end{equation}
where~$\script{M}_j$ denotes a specific position on the track and the product
is taken over the~$N$ points in the sample~\script{D}.
$N_\lambda$ quantifies the expected number of instances in the data, which can
be expressed as the line integral of the intensity function along the track
as
\begin{equation}
N_\lambda = \int_\script{M} \lambda(\script{M} | \{\theta\}) d\script{M}.
\end{equation}
The intensity function~$\lambda(\script{M} | \{\theta\})$ describes the
predicted~\textit{observed} density and should therefore incorporate any
selection effects present in the data.
It can be expressed as the product of the selection function~\script{S}~and
the~\textit{intrinsic} density~$\Lambda$ according to
\begin{equation}
\lambda(\script{M}_j | \{\theta\}) = \script{S}(\script{M}_j | \{\theta\})
\Lambda(\script{M}_j | \{\theta\}).
\end{equation}
Plugging this into our expression for the likelihood function, we obtain
\begin{subequations}\begin{align}
L(\script{D} | \{\theta\}) &= \int_\script{M}
\left(\prod_i^N L(\script{D}_i | \script{M})\right)
\left(e^{-N_\lambda} \prod_i^N \lambda(\script{M} | \{\theta\})\right)
d\script{M}
\\
&= e^{-N_\lambda} \prod_i^N \int_\script{M} L(\script{D}_i | \script{M})
\lambda(\script{M} | \{\theta\}) d\script{M},
\end{align}\end{subequations}
where we have exploited the conditional independence of each datum, allowing us
to substitute~$L(\script{D} | \script{M}) = \prod L(\script{D}_i |
\script{M})$.
We have also dropped the subscript~$j$ in~$\lambda(\script{M}_j | \{\theta\})$
because we are computing the line integral along the track~\script{M}, so a
specific location~$\script{M}_j$ is implicit.
\par
Now taking the logarithm of the likelihood function produces the following
expression for~$\ln L$:
\begin{equation}
\ln L(\script{D} | \{\theta\}) = -N_\lambda + \sum_i^N \ln \left(
\int_\script{M} L(\script{D}_i | \script{M})\lambda(\script{M} | \{\theta\})
d\script{M}
\right).
\label{eq:lnL_with_integral}
\end{equation}
The next step is to assess the likelihood~$L(\script{D}_i | \script{M})$ of
observing each datum given the predicted track.
The integral in this equation as written indicates that the most general
solution is to marginalize the likelihood over the entire evolutionary track.
Due to observational uncertainties, there is no way of knowing~\textit{a priori}
from which point on the track any individual datum is truly associated with.
If this were the case, it would reduce~$L(\script{D}_i | \script{M})$ to a
delta function at the known position on the track.
As discussed in~\S\S~\ref{sec:fitting},~\ref{sec:mocks} and
\ref{sec:conclusions}, our tests against mock samples indicates that this
marginalization is essential to ensuring the accuracy of the best-fit
parameters.
\par
In practice, the track may be complicated in shape and is generally not known
as a smooth and continuous function, instead in some piece-wise linear
approximation computed by a numerical code.
The line segment connecting the points~$\script{M}_j$ and~$\script{M}_{j + 1}$
on the track can be expressed as
\begin{equation}
\Delta\script{M}_{j,j + 1} = \script{M}_j + q(\script{M}_{j + 1} -
\script{M}_j) \qquad (0 \leq q \leq 1).
\end{equation}
If the errors associated with the observed datum~$\script{D}_i$ are accurately
described by a multivariate Gaussian, then the likelihood of observing
$\script{D}_i$ given a point along this line segment can be expressed in terms
of its covariance matrix~$C_i$ according to
\begin{subequations}\begin{align}
L(\script{D}_i | \script{M}_j, q) &=
\frac{1}{\sqrt{2\pi \det{(C_i)}}}
\exp\left(\frac{-1}{2} \delta_{ij} C_i^{-1} \delta_{ij}^T\right)
\label{eq:l_di_mj_q}
\\
\delta_{ij} &= \script{D}_i - \Delta \script{M}_{j,j + 1}(q)
\\
&= \script{D}_i - \script{M}_j - q(\script{M}_{j + 1} - \script{M}_j)
\\
&= \Delta_{ij} - qm_j,
\label{eq:delta_minus_qmj}
\end{align}\end{subequations}
where~$\delta_{ij}$ is the vector difference between~$\script{D}_i$ and the
point along the track~$\script{M}_{j,j + 1}(q)$ in the observed space and the
superscript~$T$ denotes the transposed vector.
For notational convenience, we have made the substitutions
$\Delta_{ij} = \script{D}_i - \script{M}_j$ and~$m_j = \script{M}_{j + 1} -
\script{M}_j$.
If the observational uncertainties on any one datum~$\script{D}_i$ are not
sufficiently described by a multivariate Gaussian, then
equation~\refp{eq:l_di_mj_q} must be replaced with some alternative
characterization of the uncertainty in the observed space, such as a kernel
density estimate evaluated at the point~$\Delta\script{M}_{j,j + 1}(q)$.
\par
To marginalize over the full length of the line segment, we integrate this
likelihood from~$q = 0$ to~$1$.
Retaining the assumption that a multivariate Gaussian is an accurate
description of the measurement errors, we first compute the square in the
exponent and isolate the terms which depend on~$q$:
\begin{equation}
\delta_{ij} C_i^{-1} \delta_{ij}^T = \Delta_{ij} C_i^{-1} \Delta_{ij}^T -
2q\Delta_{ij}C_i^{-1}m_j^T + q^2m_jC_i^{-1}m_j^T.
\label{eq:chisquared}
\end{equation}
% Equation~\refp{eq:delta_minus_qmj} suggests that whenever the spacing between
% track points~$m_j$ is small compared to the observational uncertainties, then
% $qm_j \ll \Delta_{ij}$ and~$\delta_{ij} \approx \Delta_{ij}$, indicating that
% the second and third terms in equation \refp{eq:chisquared} are negligible.
% If we consider the full line segment, integrating from~$q = 0$ to~$1$:
To marginalize over the full line segment, we integaate from~$q = 0$ to~$1$:
\begin{subequations}\begin{align}
L(\script{D}_i | \script{M}_j) &= \int_0^1 L(\script{D}_i | \script{M}_j, q) dq
\\
&= \frac{1}{\sqrt{2\pi\det{(C_i)}}} \int_0^1 \exp \left(
\frac{-1}{2}\delta_{ij}C_i^{-1}\delta_{ij}^T
\right) dq.
\end{align}\end{subequations}
Substituting equation~\refp{eq:chisquared} for the argument inside the
exponential and making the substitutions~$a = m_j C_i^{-1} m_j^T$ and
$b = \Delta_{ij} C_i^{-1} m_j^T$ yields
\begin{subequations}\begin{align}
\begin{split}
L(\script{D}_i | \script{M}_j) &= \frac{1}{\sqrt{2\pi \det{(C_i)}}}
\exp\left(\frac{-1}{2}\Delta_{ij} C_i^{-1} \Delta_{ij}^T\right)
\\
&\qquad \int_0^1 \exp\left(\frac{-1}{2}(aq^2 - 2bq)\right)dq
\end{split}
\\
\begin{split}
&= \frac{1}{\sqrt{2\pi\det{(C_i)}}}
\exp\left(\frac{-1}{2}\Delta_{ij}C_i^{-1}\Delta_{ij}^T\right)
\sqrt{\frac{\pi}{2a}}
\\
&\qquad \exp\left(\frac{b^2}{2a}\right)
\left[\erf\left(\frac{a - b}{\sqrt{2a}}\right) - \erf\left(\frac{b}{\sqrt{2a}}
\right)\right].
\end{split}
\end{align}\end{subequations}
For notational simplicity, we introduce the corrective term~$\beta_{ij}$ given
by
\begin{equation}
\beta_{ij} = \sqrt{\frac{\pi}{2a}} \exp\left(\frac{b^2}{2a}\right)
\left[\erf\left(\frac{a - b}{\sqrt{2a}}\right) - \erf\left(\frac{b}{\sqrt{2a}}
\right)\right]
\label{eq:corrective_beta}
\end{equation}
such that~$L(\script{D}_i | \script{M}_j)$ is given by
\begin{equation}
L(\script{D}_i | \script{M}_j) = \frac{\beta_{ij}}{\sqrt{2\pi \det{(C_i)}}}
\exp\left(\frac{-1}{2}\Delta_{ij} C_i^{-1} \Delta_{ij}^T\right).
\end{equation}
We can now express the the likelihood function as written in equation
\refp{eq:lnL_with_integral} as a summation over the points
$\script{M} = \{\script{M}_1, \script{M}_2, \script{M}_3, ..., \script{M}_K\}$
at which the track is sampled:
\begin{equation}\begin{split}
&\ln L(\script{D} | \{\theta\}) = -N_\lambda
- \sum_i^N \ln \left(\sqrt{2\pi \det{(C_i)}}\right) +
\\
&\qquad \sum_i^N \ln \left(
\sum_j^K \beta_{ij}
\exp\left(\frac{-1}{2}\Delta_{ij}C_i^{-1}\Delta_{ij}^T\right)
\lambda(\script{M}_j | \{\theta\})
\right).
\end{split}\end{equation}
Equation~\refp{eq:delta_minus_qmj} indicates that whenever the spacing between
track points~$m_j$ is small compared to the observational uncertainties, then
$qm_j \ll \Delta_{ij}$ and~$\delta_{ij} \approx \Delta_{ij}$.
As a consequence,~$\beta_{ij} \approx 1$ and this corrective term can be safely
neglected.
In some cases, however, computing the evolutionary track~\script{M}~may be
computationally expensive, making it potentially advantageous to reduce the
the number of computed points~$K$ in exchange for a slightly more complicated
likelihood calculation.
\par
As discussed above, the intensity function~$\lambda$ quantifies the observed
density of points, incorporating any selection effects present in the data into
the predicted intrinsic density~$\Lambda$.
In a one-zone GCE model,~$\Lambda$ is given by the SFH, modulo the small effect
of mass loss from recycled stellar envelopes (see discussion in, e.g.,
\citealt{Weinberg2017}).
This multiplicative factor on the likelihood~$L$ can be incorporated by simply
letting the pair-wise component of the datum~$\script{D}_i$ and the point along
the track~$\script{M}_j$ take on a weight
$w_j \equiv \script{S}(\script{M}_j | \{\theta\}) \dot{M}_\star(\script{M}_j |
\{\theta\})$ determined by the survey selection function~\script{S}~and the
SFR~$\dot{M}_\star$ at the point~$\script{M}_j$.
The predicted number of instances~$N_\lambda$, originally expressed as the
line integral of the intensity function~$\lambda$, can now be expressed as the
sum of the weights~$w_j$.
This gives rise to the following likelihood function:
\begin{equation}
\ln L(\script{D} | \{\theta\}) \propto
\sum_i^N \ln \left(\sum_j^K
\beta_{ij} w_j \exp\left(\frac{-1}{2}\Delta_{ij} C_i^{-1} \Delta_{ij}^T\right)
\right) - \sum_j^K w_j,
\label{eq:lnL_minus_weights}
\end{equation}
where we have omitted the term~$\sum \ln \left(\sqrt{2\pi \det{(C_i)}}\right)$
because it is a constant which can safely be neglected in the interest of
optimization.
This likelihood function considers each pair-wise combination of the data and
model, weighting the likelihood according to the predicted density of
observations and penalizing models by the sum of their weights.
This penalty encourages parsimony, lowering the inferred likelihood when data
are predicted to exist in regions of the observed space where there are none
and rewarding models which explain the observations in as few predicted
instances as possible.
\par
In many one-zone GCE models, however, the normalization of the SFH is
irrelevant to the evolution of the abundances.
Because the metallicity is given by the metal mass~\textit{relative} to the ISM
mass, the normalization often cancels.
Because the SFH determines the weights, it is essential in these cases to
ensure that the sum of the weights has no impact on the inferred likelihood.
To this end, we consider a density~$\rho$ with some unknown overall
normalization defined relative to the intensity function according to
\begin{subequations}\begin{align}
\lambda(\script{M} | \{\theta\}) &= N_\lambda \rho(\script{M} | \{\theta\})
\\
\int_\script{M} \rho(\script{M} | \{\theta\}) d\script{M} &= 1.
\end{align}\end{subequations}
Plugging this into equation~\refp{eq:lnL_with_integral} and pulling~$N_\lambda$
out of the natural logarithm yields the following expression for the likelihood
function:
\begin{equation}\begin{split}
\ln L(\script{D} | \{\theta\}) &= -N_\lambda + N \ln N_\lambda +
\sum_i^N \ln \left(\sqrt{2\pi \det{(C_i)}}\right) +
\\
&\qquad \sum_i^N \ln \left(
\int_\script{M} L(\script{D}_i | \script{M}) \rho(\script{M} | \{\theta\})
d\script{M} \right).
\end{split}\end{equation}
Reducing this equation proceeds in the exact same manner as above except with
$\rho$ in place of~$\lambda$ and the extra term~$N \ln N_\lambda$, which
yields
\begin{equation}\begin{split}
\ln L(\script{D} | \{\theta\}) &= -N_\lambda + N \ln N_\lambda +
\sum_i^N \ln \left(\sqrt{2\pi \det{(C_i)}}\right) +
\\
&\qquad \sum_i^N \ln \left(
\sum_j^K \beta_{ij} w_j
\exp\left(\frac{-1}{2}\Delta_{ij} C_i^{-1} \Delta_{ij}^T\right)\right).
\end{split}\end{equation}
For notational convenience, we have left the normalization of the weights
written as~$N_\lambda$.
In the interest of optimizing the likelihood function, we take the partial
derivative of~$\ln L$ with respect to~$N_\lambda$ and find that it is equal to
zero when~$N_\lambda = N$.
Because~$\rho$ is by definition un-normalized, we can simply choose this
overall scale.
The first two terms in the above expression for~$\ln L$ then become
$-N + N \ln N$, a constant for a given sample which can safely be neglected
for optimzation along with the term involving the determinants of the
covariance matrices.
We arrive at the following expression for the likelihood function in cases
where the normalization of the SFH does not impact the evolution of the
abundances:
\begin{subequations}\begin{align}
\ln L(\script{D} | \{\theta\}) &= \sum_i^N \ln \left( \sum_j^K \beta_{ij} w_j
\exp\left(\frac{-1}{2}\Delta_{ij} C_i^{-1} \Delta_{ij}^T\right) \right)
\label{eq:lnL_fracweight}
\\
\sum_j^K w_j &= 1,
\label{eq:lnL_fracweightsum}
\end{align}\end{subequations}
where the second expression arises from the requirement that the line integral
of the un-normalized density~$\rho$ along the track equal 1.
\par
In summary, when inferring best-fit parameters for one-zone GCE models in which
the normalization of the SFH is irrelevant to the evolution of the abundances,
authors should adopt equations~\refp{eq:lnL_fracweight} and
\refp{eq:lnL_fracweightsum}.
If the model is instead parametrized in such a manner that the normalization
does indeed impact the abundance evolution, then authors should adopt
equation~\refp{eq:lnL_minus_weights}.
Such models can arise, e.g., when evolutionary parameters depend to the surface
density of gas or star formation, such as a non-linear scaling between the
surface densities of gas and star formation (i.e. the so-called
Kennicutt-Schmidt relation,~$\dot{\Sigma}_\star \propto \Sigma_\text{g}^N$;
\citealp{Kennicutt1998, delosReyes2019, Kennicutt2021}), or a mass-loading
factor~$\eta$ which grows with the stellar mass to mimic the deepending of
the potential well~\citep[e.g.][]{Conroy2022}.
In either case, the corrective term~$\beta_{ij}$ given by equation
\refp{eq:corrective_beta} is approximately 1 and can be safely neglected when
the track is densely sampled relative to the observational uncertainties.
In this paper, we assume a linear relation between the SFR and the ISM mass and
a constant mass-loading factor, which makes the normalization of the SFH
irrelevant; we therefore use equations~\refp{eq:lnL_fracweight} and
\refp{eq:lnL_fracweightsum} as our likelihood function.

\end{document}
