
\documentclass[ms.tex]{subfiles}
\begin{document}

\section{Galactic Chemical Evolution}
\label{sec:onezone}

The fundamental assumption of one-zone models is that newly produced metals mix
instantaneously throughout the star-forming gas reservoir.
In detail, this assumption is valid as long as the mixing timescale is short
compared to the depletion timescale (i.e. the average time a fluid element
remains in the ISM before getting incorporated into new stars or ejected in an
outflow).
Based on the observations of~\citet{Leroy2008},~\citet{Weinberg2017} calculate
that characteristic depletion times can range from~$\sim500$ Myr up to~$\sim10$
Gyr for conditions in typical star forming disc galaxies.
In the dwarf galaxy regime, the length scales are short, star formation is slow
\citep[e.g.][]{Hudson2015}, and the ISM velocities are turbulent
\citep{Dutta2009, Stilp2013, Schleicher2016}.
With this combination, instantaneous mixing should be a good approximation,
though we are unaware of any studies which address this observationally.
As long as the approximation is valid, then there should exist an evolutionary
track in chemical space (e.g. the~\afe-\feh~plane) about which the intrinsic
scatter is negligible compared to the measurement uncertainty.
This empirical test should be feasible on a galaxy-by-galaxy basis.
\par
With the goal of assessing the information content of one-zone GCE models
applied to dwarf galaxies, we emphasize that the accuracy of the methods we
outline in this paper are contingent on the validity of the instantaneous
mixing approximation.
This assumption reduces GCE to a system of coupled integro-differential
equations which we solve using the publicly available~\textsc{Versatile
Integrator for Chemical Evolution} (\vice\footnote{
	\url{https://pypi.org/project/vice}
};~\citealp{Johnson2020}).
We provide an overview of the model framework below and refer to
\citet{Johnson2020} and the~\vice~science documentation\footnote{
	\url{https://vice-astro.readthedocs.io/en/latest/science_documentation}
} for further details.
\par
At a given moment in time, gas is added to the ISM via inflows and recycled
stellar envelopes and is removed from the ISM by star formation and outflows,
if present.
This gives rise to the following differential equation describing the evolution
of the gas supply:
\begin{equation}
\dot{M}_\text{g} = \dot{M}_\text{in} - \dot{M}_\star - \dot{M}_\text{out}
+ \dot{M}_\text{r},
\label{eq:mdotgas}
\end{equation}
where~$\dot{M}_\text{in}$ is the infall rate,~$\dot{M}_\star$ is the SFR,
$\dot{M}_\text{out}$ is the outflow rate, and~$\dot{M}_\text{r}$ describes
the return of stellar envelopes from previous generations of stars.
\par
\vice~implements the same characterization of outflows as the~\textsc{FlexCE}
\citep{Andrews2017} and~\textsc{OMEGA}~\citep{Cote2017} chemical evolution
codes in which a ``mass-loading factor''~$\eta$ describes a linear relationship
between the outflow rate itself and the SFR:
\begin{equation}
\eta \equiv \frac{\dot{M}_\text{out}}{\dot{M}_\star}.
\label{eq:massloading}
\end{equation}
This parametrization is appropriate for models in which massive stars are the
dominant source of energy for outflow-driving winds.
Empirically, the strength of outflows (i.e. the value of~$\eta$) is strongly
degenerate with the absolute scale of nucleosynthetic yields.
% This ``yield-outflow degeneracy'' arises because the yields themselves are the
% primary source term in describing enrichment rates in galaxies while outflows,
% if present, are the primary sink term.
We discuss this further below and quantify the strength of the degeneracy in
more detail in Appendix~\ref{sec:degeneracy}.
\par
The SFR and the mass of the ISM are related by the star formation efficiency
(SFE) timescale~$\tau_\star$, defined as the ratio of the two:
\begin{equation}
\tau_\star \equiv \frac{M_\text{g}}{\dot{M}_\star}.
\label{eq:taustar}
\end{equation}
The inverse~$\tau_\star^{-1}$ is the SFE itself, quantifying the
\textit{fractional} rate at which some ISM fluid element is forming stars.
Some authors refer to~$\tau_\star$ as the ``depletion time''
\citep[e.g.][]{Tacconi2018} because it describes the e-folding decay timescale
of the ISM mass due to star formation if no additional gas is added.
Our nomenclature is based off of~\citet{Weinberg2017}, who demonstrate that
depletion times in GCE models can shorten significantly in the presence of
outflows.
\par
The recycling rate~$\dot{M}_\text{r}$ is a complicated function which depends
on the stellar initial mass function~\citep[IMF; e.g.][]{Salpeter1955,
Miller1979, Kroupa2001, Chabrier2003}, the initial-final remnant mass relation
\citep[e.g.][]{Kalirai2008}, and the mass-lifetime relation\footnote{
	We assume a~\citet{Kroupa2001} IMF and the~\citet{Larson1974} mass-lifetime
	relation throughout this paper.
	These choices, however, do not significantly impact our conclusions as
	$\eta$ and~$\tau_\star$ play a much more significant role in establish the
	evolutionary histories of our GCE models.
	Our fitting method is nonetheless easily extensible to models which relax
	these assumptions.
}~\citep*[e.g.][]{Larson1974, Maeder1989, Hurley2000}, all of which must then
be convolved with the SFH.
However, the detailed rate of return of stellar envelopes has only a
second-order effect on the gas-phase evolutionary track in the~\afe-\feh~plane.
The first-order details are instead determined by the SFE timescale~$\tau_\star$
and the mass-loading factor~$\eta$~\citep{Weinberg2017}.
With low~$\tau_\star$, nucleosynthesis is fast due to fast star formation, and
a higher metallicity can be attained before the onset of SNe Ia than in lower
SFE models.
For this reason,~$\tau_\star$ plays the dominant role in establishing the
position of the ``knee'' in the~\afe-\feh~plane.
As the galaxy evolves, it approaches a chemical equilibrium in which newly
produced metals are balanced by losses to outflows and new stars.
Controlling the strength of an important sink term,~$\eta$ plays the dominant
role in shaping the late-time equilibrium abundance of the model, with high
outflow models (i.e. high~$\eta$) predicting lower equilibrium abundances
than their weak outflow counterparts.
For observed data, the shape of the track itself directly constrains these
parameters (see discussion in~\S~\ref{sec:mocks:recovered}) below).
The detailed form of the SFH has minimum impact on the shape of the tracks,
provided that there are no sudden events such as a burst of star formation
\citep{Weinberg2017, Johnson2020}.
Instead, that information is encoded in the stellar metallicity distribution
functions (MDFs; i.e. the density of stars along the evolutionary track).
\par
In the present paper, we focus on the enrichment of the so-called ``alpha''
(e.g. O, Ne, Mg) and ``iron-peak'' elements (e.g. Cr, Fe, Ni, Zn), with the
distribution of stars in the~\afe-\feh~plane being our primary observational
diagnostic to distinguish between GCE models.
Massive stars and their SNe are the dominant enrichment source of alpha
elements in the universe, while iron-peak elements are produced in significant
amounts by both massive stars and SNe Ia~\citep[e.g.][]{Johnson2019}.
In detail, alpha and iron-peak elements are also produced by slow neutron
capture nucleosynthesis, an enrichment patyway responsible for much of the
abundances of yet heavier nuclei (specifically Sr and up).
Because the neutron capture yields of alpha and iron-peak elements are
negligible compared to their SN yields, we do not discuss this process further.
Our fitting method is nonetheless easily extensible to GCE models which do,
provided that the data contain such measurements.
\par
Due to the steep nature of the stellar mass-lifetime relation
\citep[e.g.][]{Larson1974, Maeder1989, Hurley2000}, massive stars, their winds,
and their SNe enrich the ISM on~$\sim$few Myr timescales.
As long as this is shorter than the relevant timescales for a galaxy's
evolution and the present-day stellar mass is sufficiently high such that
stochastic sampling of the IMF does not significantly impact the yields, then
it is adequate to approximate this nucleosynthetic material as some
population-averaged yield ejected instantaneously following a single stellar
population's formation.
This implies a linear relationship between the CCSN enrichment rate and the
SFR:
\begin{equation}
\dot{M}_\text{x}^\text{CC} = y_\text{x}^\text{CC} \dot{M}_\star,
\end{equation}
where~$y_\text{x}^\text{CC}$ is the IMF-averaged fractional net yield from
massive stars of some element x.
That is, for a fiducial value of~$y_\text{x}^\text{CC} = 0.01$, 100~\msun~of
star formation would produce 1~\msun~of~\textit{newly produced} element x (the
return of previously produced metals is implemented as a separate term
in~\vice; see~\citealt{Johnson2020} or the~\vice~science documentation for
details).
\par
Unlike CCSNe, SNe Ia occur on a significantly extended delay time distribution
(DTD).
The details of the DTD are a topic of active inquiry~\citep[e.g.][]{Greggio2005,
Strolger2020, Freundlich2021}, and at least a portion of the uncertainty can be
traced to uncertainties in both galactic and cosmic SFHs.
Comparisons of the cosmic SFH~\citep[e.g.][]{Hopkins2006, Madau2014, Davies2016,
Madau2017, Driver2018} with volumetric SN Ia rates as a function of redshift
indicate that the cosmic DTD is broadly consistent with a uniform~$\tau^{-1}$
power-law (\citealp{Maoz2012a};~\citealp*{Maoz2012b};~\citealp{Graur2013,
Graur2014}).
Following~\citet{Weinberg2017}, we take a~$\tau^{-1.1}$ power-law DTD with a
minimum delay-time of~$t_\text{D} = 150$ Myr, though in principle this
delay-time can be as short as~$t_\text{D} \approx 40$ Myr due to the lifetimes
of the most massive white dwarf progenitors.
For any selected DTD~$R_\text{Ia}(\tau)$, the SN Ia enrichment rate can be
expressed as an integral over the SFH weighted by the DTD:
\begin{equation}
\dot{M}_\text{x}^\text{Ia} = y_\text{x}^\text{Ia}\ddfrac{
	\int_0^{T - t_\text{D}} \dot{M}_\star(t) R_\text{Ia}(T - t) dt
}{
	\int_0^\infty R_\text{Ia}(t) dt
}.
\end{equation}
In general, the mass of some element x in the ISM is also affected by outflows,
recycling and star formation.
The total enrichment rate can be computed by simply adding up all of the source
terms and subtracting the sink terms:
\begin{equation}
\dot{M}_\text{x} = \dot{M}_\text{x}^\text{CC} + \dot{M}_\text{x}^\text{Ia} -
Z_\text{x}\dot{M}_\star - Z_\text{x}\dot{M}_\text{out} + \dot{M}_{\text{x,r}}.
\label{eq:enrichment}
\end{equation}
This equation as written assumes that the outflowing material is of the same
composition as the ISM, but in principle, the various nuclear species of
interest may be some factor above or below the ISM abundance.
In the present paper we assume all accreting material to be zero metallicity
gas; when this assumption is relaxed, this equation simply picks up an
additional term~$Z_\text{x,in}\dot{M}_\text{in}$.
\par
As mentioned above, the strength of outflows are degenerate with the absolute
scale of nucleosynthetic yields.
This ``yield-outflow degeneracy'' is remarkably strong, and it arises because
yields and outflows are the dominant source and sink terms in equation
\refp{eq:enrichment} above.
As a consequence, high-yield and high-outflow models generally have a
low-yield and low-outflow counterpart that predicts a similar enrichment
history.
In order to break this degeneracy, only a single parameter setting the absolute
scale is required.
Throughout this paper, with the exception of Appendix~\ref{sec:degeneracy},
we therefore set the alpha element yield from massive stars to be exactly
$\yacc = 0.01$ and let our Fe yields be free parameters.
This value is somewhat informed by nucleosynthesis theory in that massive star
evolutionary models (e.g.~\citealp*{Nomoto2013};~\citealp{Sukhbold2016,
Limongi2018}) typically predict~$y_\text{O}^\text{CC} = 0.005 - 0.015$ (see
discussion in, e.g.,~\citealp{Weinberg2017} and~\citealp{Johnson2020}).
The primary motivation behind this choice, however, is to selected a round
number from which our best-fit values affected by this degeneracy can simply be
scaled up or down to accommodate alternate parameter choices.
This degeneracy is sufficiently strong such that some authors
(e.g.~\citealp*{Minchev2013, Minchev2014};~\citealp{Minchev2017,
Spitoni2020, Spitoni2021}) assume that outflows do not sweep up ambient ISM at
all (i.e.~$\eta = 0$), while others~\citep[e.g.][]{Andrews2017, Weinberg2017,
Cote2017, Trueman2022} instead argue that this is an important ingredient in
GCE models.
In these cases,~\yacc~can instead be included as a free parameter and the
overall scale of yields and outflows is instead set by the assumption
that~$\eta = 0$.
However, the arguments in favor of neglecting outflows have generally been made
in the context of the Milky Way, whereas ejecting ambient ISM in an outflow
should in principle be easier in a dwarf galaxy where the gravity well is
intrinsically shallower, potentially implying~$\eta > 0$.
We reserve further discussion of this topic to Appendix~\ref{sec:degeneracy}
where we also quantify the considerable strength of the yield-outflow
degeneracy in more detail.

\end{document}

