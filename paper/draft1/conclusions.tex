
\documentclass[ms.tex]{subfiles}
\begin{document}

\section{Discussion and Conclusions}
\label{sec:conclusions}

We use statistically robust methods to derive best-fit parameters of
one-zone GCE models for two disrupted dwarf galaxies in the Mily Way stellar
halo: GSE~\citep{Belokurov2018, Helmi2018}, and Wukong (\citealp{Naidu2020,
Naidu2022}; also known as LMS-1,~\citealp{Yuan2020}).
We fit both galaxies with an exponential accretion history
(see~\S~\ref{sec:mocks}), deriving e-folding timescales and durations of star
formation of~$(\tau_\text{in}, \tau_\text{tot}) \approx (1~\text{Gyr},
5.4~\text{Gry})$ for GSE and~$(\tau_\text{in}, \tau_\text{tot}) \approx
(3.1~\text{Gyr}, 3.4~\text{Gyr})$ for Wukong (we refer to table
\ref{tab:results} for exact values).
These differences in evolutionary parameters are qualitatively consistent with
predictions from hydrodynamical simulations~\citep[e.g.,][]{GarrisonKimmel2019}
and semi-analytic models of galaxy formation~\citep[e.g.,][]{Baugh2006,
Somerville2015a, Behroozi2019}.
\par
Quantitatively, we arrive at a longer duration of star formation than
\citet{Gallart2019}, who derived an age distribution for GSE by analysing its
CMD according to the method described in~\citet{Dolphin2002} and found a
median age of 12.37 Gyr.
In agreement with this,~\citet{Vincenzo2019} infer a sharply declining infall
history with a timescale of~$\tau_\text{in} = 0.24$ Gyr.
The star-by-star age measurements provided by H3~\citep{Conroy2019} suggest
that GSE's SFH was more extended (see Fig.~\ref{fig:gse}), and we consequently
deduce a higher value of~$\tau_\text{in} = 1.01 \pm 0.13$ Gyr.
If its first infall into the Milky Way halo was~$\sim$10 Gyr ago
\citep[e.g.,][]{Helmi2018, Bonaca2020}, then the duration of star formation we
derive ($\tau_\text{tot} = 5.4$ Gyr) implies that GSE formed stars for
$1.6 - 2.2$ Gyr, depending on exactly how long ago it began.
\par
To our knowledge, this is the first detailed modelling of multi-element stellar
abundances in Wukong.
Star formation lasted~$\sim$2 Gyr longer in GSE than Wukong -- if they started
forming stars around the same time, then Wukong was quenched around the same
time as the GSE-Milky Way merger.
However, our age distribution is purely a prediction as we do not have any age
measurements of individual stars and we do not model Wukong's CMD here.
We find no statistically significant evidence of IMF variability or
metallicity-dependent Fe yields comparing GSE and Wukong.
\par
Although these models are statistically good descriptions of our GSE and Wukong
data, they are simplified in nature.
In particular, we have assumed a linear relation between the gas supply and the
SFR while empirical results would suggest a non-linear relation
\citep[e.g.][]{Kennicutt1998, Kennicutt2012, delosReyes2019, Kennicutt2021}.
We have also taken a constant outflow mass-loading factor~$\eta$ when in
principle,~$\eta$ could vary with time as the potential well of the galaxy
deepens.
The primary motivation of these choices, however, is to provide proof of
concept for this fitting method with an example application to observations.
We reserve more detailed modelling of galaxies with both simple and complex
evolutionary histories for future work.
\par
Our method is built around a likelihood function which requires no binning of
the data (equation~\ref{eq:likelihood}) and has two central features.
First, the likelihood of observing some datum~$\script{D}_i$ must be
marginalized over the entire evolutionary track~\script{M}.
This requirement arises due to measurement uncertainties: for any given datum,
it is impossible to know where on the track the observation truly arose from,
and mathematically accounting for this requires considering all pair-wise
combinations between~\script{M} and~\script{D}.
Second, the likelihood of observing a datum~$\script{D}_i$ given a point on
the evolutionary track~$\script{M}_j$ must be weighted by the SFR at that time
in the model, simultaneously folding in any selection effects introduced by the
survey.
This requirement arises because an observed star is proportionally more likely
to have been sampled from an epoch of a galaxy's history in which the SFR was
large and/or if the survey designed is biased toward certain epochs.
\par
We establish the accuracy of our method by means of tests against mock data,
demonstrating that the known evolutionary parameters of subsampled input models
are accurately re-derived across a broad range of sample sizes
($N = 20 - 2000$), abundance uncertainties ($\sigma_\text{[X/Y]} = 0.01 - 0.5$),
age uncertainties ($\sigma_{\log_{10}(\text{age})} = 0.02 - 1$) and the
fraction of the sample with age information ($f_\text{age} = 0 - 1$; see
discussion in~\S~\ref{sec:mocks}).
The fit precision of the inferred parameters generally scales with sample size
as~$\sim$$N^{-0.5}$.
We demonstrate that evolutionary timescales can theoretically be derived with
abundances alone, but in practice age information helps reduce the effect of
systematic differences between the data and model, improving both the
accuracy and the precision.
Our likelihood function requires no binning of the data, and we derive it
in Appendix~\ref{sec:likelihood} assuming only that the model predicts an
evolutionary track of some unknown shape in the observed space.
It should therefore be applicable to one-zone models of any parametrization as
well as easily extensible to other astrophysical models in which the chief
prediction is a track of some form (e.g., stellar streams and isochrones).
\par
{\color{red}
Blurbs about quenching times and how doing things with CMD-derived SFHs included
as a prior could be highly impactful -- can also get photometric MDFs from the
CMD, modelling of which would proceed with a similar framwork.
To pin down the yield outflow degeneracy, one can imagine some form of
hierarchical modelling where they simultaneously fit the abundances and ages
across a broad sample of galaxies and enforce that the yields be the same
between all of them -- this could help break the degeneracy.
Blurb about how the next step is to take a ``chemical census'' of the local
group dwarf galaxies, and that Roman is going to revolutionize stellar pops
and statistically robust models such as ours are going to be essential to
deduce what lessons they have to teach us about galaxy evolution at the
low-mass end.}

% We have started from first principles with poisson sampling from an arbitrary
% evolutionary track in abundance space.
% From this, we derive a likelihood function for fitting one-zone GCE models to
% observed stellar abundances and ages, where available (see equation
% \ref{eq:likelihood} in~\S~\ref{sec:fitting} and derivation in
% Appendix~\ref{sec:likelihood}).
% This derivation does not assume that the observational benchmarks are
% stellar ages and abundances, and it should therefore be easily extensible to
% other astrophysical models in which the chief prediction is a track of some
% form (e.g. stellar streams in kinematic space and isochrones in CMDs).
% Although there are a handful of examples in the literature already in which
% the authors derive best-fit parameters for GCE models~\citep[e.g.][]{Kirby2011,
% Spitoni2020, Spitoni2021, Hasselquist2021}, to our knowledge this is the first
% of such instances in which the statistical validity of the method has been
% robustly demonstrated.
% \par
% There are two central features to our likelihood function which appear in all
% of its forms.
% First, the likelihood of observing some datum~$\script{D}_i$ must be
% marginalized over the entire evolutionary track~\script{M}.
% This requirement arises due to observational uncertainties: for any given
% datum, there is no way of knowing where on the track the observation truly
% arose from, and to properly take this into account, one must consider all
% pair-wise combinations of~\script{D}~and~\script{M}.
% Second, the likelihood of observing a datum~$\script{D}_i$ given a point on
% the evolutionary track~$\script{M}_j$ must be weighted by the SFR at that time
% in the model, simultaneously folding in any selection effects introduced by the
% survey.
% This requirement arises because an observed star is proportionally more likely
% to have been sampled from an epoch of a galaxy's history in which the SFR was
% large and/or if the survey designed is biased toward certain epochs.
% \par
% We establish the accuracy of this fitting method by means of mock data samples.
% To this end, we sample from the SFH of an input one-zone model and perturb the
% ages and abundances by some artificial uncertainty.
% Within the range of sample sizes ($N = 20 - 2000$), abundance uncertainties
% ($\sigma_\text{[X/Y]} = 0.01 - 0.5$), age uncertainties
% ($\sigma_{\log_{10}(\text{age})} = 0.02 - 1$) and the fraction of the sample
% with age information ($f_\text{age} = 0 - 1$) probed here, we find that this
% method recovers the known evolutionary parameters of the input model in all
% cases (see Fig.~\ref{fig:accuracy}).
% The mean offset between the known and best-fit values is at the~$\sim 1 \sigma$
% level, exactly as expected for a Gaussian random process (see discussion
% in~\S~\ref{sec:mocks:variations}).
% We have conducted further ``stress tests'' of this method by fitting models
% with complicated evolutionary histories whichwe elect not to illustrate in this
% paper.
% In all cases, our method accurately recovers the input model parameters,
% reinforicng our interpretation that equation~\refp{eq:likelihood} should be
% universally applicable to GCE models as it takes the model-predicted track as
% input and is independent of the details of the model from which it was
% computed.
% We emphasize that it is not that the marginalization over the track and the
% weights each individually improve the precision of the fit; it is that they
% are both~\textit{essential} to the accuracy of the inferred parameters.
% When either of the two are neglected from our fitting procedure, we fail to
% recover the known parameters with discrepancies at the many-$\sigma$ level.
% \par
% Interestingly, our likelihood function accurately recovers input timescales
% even in the~\textit{absence} of age information, including the total duration
% of star formation.
% This arises through their impact on the centroid and shape of the MDF,
% information which can be extracted with a fit to a sample of sufficient size
% and measurement precision.
% While age information does improve the precision of the fit, the inferred
% values remain accurate nonetheless even without such measurements.
% However, we find significant differences in the inferred evolutionary
% timescales when fitting our GSE sample with and without ages.
% This discrepancy suggests that in application to observed data, this requires
% removing all systematic differences between the data and model and that age
% information provides more information than our theoretical justification would
% suggest.
% Nonetheless, this method could be of notable use to authors interested in
% deriving quenching times for dwarf galaxies (i.e. the lookback time to when
% star formation stopped).
% At present, the most robust method for empirically measuring a galaxy's
% quenching time is to directly measure its SFH by some means, such as analysing
% its CMD~\citep[e.g.][]{Sohn2013, Weisz2015}.
% At present, this leaves only a handful of galaxies with well-constrained SFHs
% outside of the Milky Way subgroup~\citep{Monelli2010a, Monelli2010b,
% Weisz2014a}.
% By constraining the quenching times and SFHs from stellar abundances and ages
% (or, alternatively, some empirically-motivated prior on the SFH), these
% chemically-derived evolutionary histories could offer new and independent
% insight into galaxy formation at the low stellar mass regime.
% \par
% We apply our fitting method to the GSE~\citep{Belokurov2018, Helmi2018} and
% Wukong~\citep{Naidu2020, Naidu2022} stellar streams in the Milky Way halo using
% data from the H3 survey~\citep{Conroy2019}.
% We find that the Wukong progenitor experienced a more extended infall history
% than GSE ($\tau_\text{in} = 3.46^{+3.58}_{-1.36}$ Gyr versus
% $\tau_\text{in} = 1.01 \pm 0.13$ Gyr) but that its duration of star formation
% was~$\sim$2 Gyr shorter.
% The Wukong stream also experienced~$\sim$5 times stronger winds and~$\sim$3
% times less efficient star formation than the GSE.
% These results point to the Wukong progenitor being significantly lower stellar
% mass than the GSE progenitor, an unsurprising result since the GSE is believed
% to be among the most massive satellites to have been accreted by the Milky Way
% \citep{Myeong2018, Deason2019, Fattahi2019, Mackereth2019, Vincenzo2019}.
% Such an interpretation is also in qualitative agreement with semi-analytic
% models of galaxy formation, which suggest that dwarf galaxies in the field
% experience more extended SFHs~\citep{Baugh2006, Somerville2015a, Behroozi2019}.
% The parametrization of our GCE models is however independent of stellar mass,
% so they offer no constraints the normalization of the SFH.
% \par
% Although these models are statistically good descriptions of our GSE and Wukong
% data, they are simplified in nature.
% In particular, we have assumed a linear relation between the gas supply and the
% SFR while empirical results would suggest a non-linear relation
% \citep[e.g.][]{Kennicutt1998, Kennicutt2012, delosReyes2019, Kennicutt2021}.
% We have also taken a constant outflow mass-loading factor~$\eta$ when in
% principle, the strength of outflows could vary with time as the potential well
% of the galaxy deepens.
% The primary motivation of these choices, however, is to provide proof of
% concept for this fitting method with an example application to observations.
% We reserve more detailed modelling of galaxies with both simple and complex
% evolutionary histories for future work.
% % \par
% % The methods outlined in this paper are only as accurate as the instantaneous
% % mixing approximation on a galaxy-by-galaxy basis, which likely limits the
% % most reliable application to the dwarf galaxy regime (see discussion
% % in~\S~\ref{sec:onezone}).
% % In larger galaxies such as the Milky Way, processes such as radial gas flows
% % \citep[e.g.][]{Lacey1985, Bilitewski2012} and the radial migration of stars
% % \citep[e.g.][]{Sellwood2002, Loebman2011, Okalidis2022} are expected to
% % significantly impact the observed abundance distributions in a given Galactic
% % region.
% % These results have prompted the development of a number of so-called
% % ``multi-zone'' GCE models~\citep[e.g.][]{Schoenrich2009, Johnson2021, Chen2022}
% % which stitch together multiple one-zone models via the exchange of gas and
% % stars.
% % Though we do not consider these models here, it is an interesting potential
% % direction for future work to properly compute statistical likelihood in the
% % presence of multiple evolutionary tracks.
% % Such models could provide valuable insight into the evolution of higher mass
% % galaxies with data from next-generation spectroscopic surveys like Milky
% % Way Mapper~\citep{Kollmeier2017}.
% % As increasingly sensitive instruments come online, statistically robust methods
% % such as this will be essential to extract the lessons that can be learned
% % about galaxies of all stellar masses.

\end{document}
