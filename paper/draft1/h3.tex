
\documentclass[ms.tex]{subfiles}
\begin{document}

\section{Application to Observations}
\label{sec:h3}

% fig 5
\begin{figure*}
\centering
\includegraphics[scale = 0.65]{gsefit_afe_feh.pdf}
\includegraphics[scale = 0.54]{gsefit_agedist.pdf}
\includegraphics[scale = 0.53]{gsefit_amr.pdf}
\caption{
Our GSE sample.
Red lines in all panels denote the best-fit one-zone model, while the
blue lines in the left and middle panels denote the best-fit model when the
age measurements are excluded from the fit.
Distributions in~\feh,~\afe~and age are convolved with the median uncertainty
of the sample (see discussion in~\S~\ref{sec:h3:gse}).
We additionally subsample 200 sets of parameter choices from our Markov chain
and plot their predictions as highly transparent lines to offer a sense of the
fit uncertainty.
\textbf{Left}: The~\afe-\feh~plane and the associated marginalized
distributions.
Stars are colour-coded according to their ages where available and are
otherwise plotted in black.
The median~\feh~and~\afe~uncertainty in the sample is shown by the error bar
to the right of the data.
Error bars indicate a~$\sqrt{N}$ uncertainty associated with random
sampling both here and in the middle panel.
\textbf{Middle}: The age distribution measured for our GSE sample (black,
binned).
\textbf{Right}: The age-\feh~(top) and age-\afe~(bottom) relations for our
sample (black) and our best-fit chemical evolution model (red).
The median~\feh,~\afe~and age uncertainties are shown by the error bars at the
top and bottom of each panel.
We plot the two stars that we exclude from our fit as black X's (likely
blue stragglers; see discussion in~\S~\ref{sec:h3:gse}).
Red points denote a mock sample drawn from our best-fit model with~$N = 95$
stars (the same size as the stars with ages in our GSE sample) and perturbed by
the median age uncertainty of the sample.
}
\label{fig:gse}
\end{figure*}

% fig 6
\begin{figure*}
\centering
\includegraphics[scale = 0.5]{gsechem_512k.pdf}
\caption{
The ``corner-plot'' showing the results of our fitting method applied to GSE
stars observed by the H3 survey.
Panels below the diagonal show 2-dimensional cross-sections of the likelihood
function while panels along the diagonal show the marginalized distributions
along with the best-fit values and confidence intervals.
Red ``cross-hairs'' mark the element of the Markov chain with the maximum
statistical likelihood.
{\color{red} blurb about the tail in the distribution seen in the
$\yfecc - \tau_\text{in}$ plane.}
}
\label{fig:gse_corner}
\end{figure*}

% fig 7
\begin{figure}
\centering
\includegraphics[scale = 0.65]{wukong_bestfit.pdf}
\caption{
Our Wukong sample in the~\afe-\feh~plane and the associated marginalized
distributions.
Error bars in the central denote the measurement uncertainty on individual
stars' abundances, and error bars in the top and right panels indicate the
uncertainty in the abundance distribution assuming~$\sigma = \sqrt{N}$ from
sampling noise.
The red lines denote our best-fit chemical evolution model in red (see
discussion in~\S~\ref{sec:h3:wukong}), with 200 additional sets of parameter
choices subsampled from our Markov chain to give a sense of the fit precision.
The blue lines denote an alternate fit in which we allow the Fe yields to vary
as free parameters.
}
\label{fig:wukong}
\end{figure}

% fig 8
\begin{figure*}
\centering
\includegraphics[scale = 0.52]{wukong_512k.pdf}
\caption{
The same as Fig.~\ref{fig:gse_corner} but for our Wukong sample.
In this fit we fix the Fe yields at the values inferred from the GSE sample.
}
\label{fig:wukong_corner}
\end{figure*}

We now apply our likelihood function (equation~\ref{eq:likelihood}) to two
disrupted dwarf galaxies in the Milky Way stellar halo.
The first is a relatively well-studied system: GSE~\citep{Belokurov2018,
Helmi2018}, believed to be responsible for a major merger event early in the
Milky Way's history~\citep{Chaplin2020} which contributed~$10^9 - 10^{10}~\msun$
of total stellar mass to the Galaxy~\citep{Deason2019, Fattahi2019,
Mackereth2019, Vincenzo2019}, including eight globular clusters in the stellar
halo~\citep{Myeong2018}.
The second is a less well-studied system: Wukong, a structure chemically
distinct from GSE which sits between it and the Helmi stream~\citep{Helmi1999}
in energy-angular momentum space~\citep{Naidu2020, Naidu2022}.
Wukong was independently discovered and dubbed LMS-1 by~\citep{Yuan2020},
and its nearly polar orbit has been characterized by~\citet{Malhan2021,
Malhan2022} and~\citet{Shank2022}.
We make use of data from the H3 survey (see discussion
in~\S~\ref{sec:h3:survey} below) and discuss our GCE model fits to GSE and
Wukong in~\S\S~\ref{sec:h3:gse} and~\ref{sec:h3:wukong} below with comparison
in~\S~\ref{sec:h3:comparison}.

\subsection{The H3 Survey}
\label{sec:h3:survey}

The H3 survey~\citep{Conroy2019} is collecting medium-resolution spectra
of~$\sim$300,000 stars in high-latitude fields ($\left|b\right| > 20^\circ$).
Spectra are collected from the Hectochelle instrument on the MMT
\citep{Szentgyorgyi2011} which delivers~$R \approx$~32,000 spectra over the
wavelength range of~$5150 - 5300$~\AA.
Spectral lines in this wavelength range are dominated by iron-peak elements and
the MgI triplet (see Fig. 6 of~\citealt{Conroy2019}).
Throughout this section, the alpha element abundances we refer to are therefore
Mg abundances specifically, whereas in previous sections an alpha element
refers to any species where the only statistically significant enrichment
source is a metallicity-dependent yield from massive stars.
\par
The survey selection function is deliberately simple: the primary sample
consists of stars with~$r$ band magnitudes of~$15 < r < 18$
and~\gaia~\citep{Gaia2016} parallaxes~$<$ 0.3 mas (this threshold has evolved
over the course of the survey as the~\gaia~astrometry has become more precise).
Stellar parameters are estimated by the~\textsc{MINESweeper} program
\citep{Cargile2020}, which fits grids of isochrones, synthetic spectra and
photometry to the Hectochelle spectrum and broadband photometry from~\gaia,
Pan-STARRS~\citep{Chambers2016}, SDSS~\citep{York2000}, 2MASS
\citep{Skrutskie2006} and WISE~\citep{Wright2010} with the~\gaia~parallax
used as a prior.
The fitted parameters include radial velocity, spectrophotometric distance,
reddening,~\feh,~\afe~and age.
The default analysis includes a complicated prior on age and distance
(see~\citealt{Cargile2020} for details).
We have also re-fit high signal-to-noise data with a flat age prior for cases
where ages play an important role.
In this paper we use the catalog which uses this flat age prior.


\subsection{\gaia-Sausage Enceladus}
\label{sec:h3:gse}

\subfile{results.tablebody.tex}

Our GSE sample consists of 189 stars, 95 of which have age measurements.
Abundance uncertainties range from~$\sim$0.02 to 0.12 dex in
both~\feh~and~\afe~with median values near~$\sim$0.05.
Every age measurement has a statistical uncertainty
$\sigma_{\log_{10}(\text{age})} \leq 0.05$, corresponding to a measurement
precision of~$\lesssim$12\%.
However, due to the difficulty associated with measuring stellar ages both
accurately and precisely~\citep[e.g.,][]{Soderblom2010, Chaplin2013, Angus2019},
we adopt~$0.05$ as the age uncertainty for the entire sample to account for any
systematic errors that may be present.
\par
We illustrate our sample in Fig.~\ref{fig:gse} along with our best-fit GCE
models (see discussion below).
We note the presence of two outliers at ages of~$\sim$5 and~$\sim$6 Gyr, marked
by X's in the right panel of Fig.~\ref{fig:gse}.
With abundances typical of the rest of the GSE population but anomalously young
ages, these stars are likely blue stragglers, which are thought to be made
hotter and more luminous by accretion from a binary companion and biasing their
age measurements to low values~\citep[e.g.][]{Bond1971, Stryker1993}.
The smooth decline of~\afe~with~\feh~and the unimodal nature of the
distributions in~\feh,~\afe~and age indicate that the GSE did not experience
any significant starburst events.
If this were the case, we would expect to see a multi-peaked age distribution
as well as an increase in~\afe~at a distinct~\feh~due to the perturbed ratio of
CCSN to SN Ia rates~\citep{Johnson2020}.
We therefore fit the GSE with an exponential infall history (the same as our
mock samples explored in~\S~\ref{sec:mocks}), omitting the two~$\sim$5
and~$\sim$6 Gyr old stars from the procedure and retaining the assumption that
star formation commenced 13.2 Gyr ago.
Because H3 selects targets based only on a magnitude range and a maximum
parallax, the selection function in chemical space should be nearly uniform
(i.e.,~$\script{S}(\script{M}_j | \{\theta\}) \approx 1$ for all points
$\script{M}_j$ along the evolutionary track.
We therefore take weights that are proportional to the SFR alone (see
equations~\ref{eq:likelihood} and~\ref{eq:weights} and discussion
in~\S~\ref{sec:fitting}).
\par
We report our best-fit evolutionary parameters in Table~\ref{tab:results}
with Fig.~\ref{fig:gse_corner} illustrating the posterior distributions.
These values suggest strong outflows ($\eta \approx 9$) and inefficient star
formation ($\tau_\star \approx 16$ Gyr).
Invoking the equilibrium arguments of~\citet{Weinberg2017}, strong outflows and
slow star formation are consistent with the metal-poor mode of the MDF and the
``knee'' in the evolutionary track occurring at low~\feh, respectively.
These results are expected for a dwarf galaxy where the gravity well is
intrinsically shallow and the stellar-to-halo mass ratios are known empirically
to be smaller than their higher mass counterparts~\citep{Hudson2015}.
The alpha-enhanced mode of the MDF reflects the short duration of star
formation, stopping before SN Ia enrichment could produce enough Fe to reach
solar~\afe.
The associated truncation of the age distribution (shown in the bottom left
panel of Fig.~\ref{fig:gse}) likely reflects the quenching of star
formation in the GSE progenitor as a consequence of RAM pressure stripping by
the hot halo of the Milky Way after its first infal~$\sim$10 Gyr ago
\citep{Bonaca2020}.
The inferred Fe yields suggest that massive stars account for
$\yfecc / (\yfecc + \yfeia) \approx 40$\% of the Fe in the universe.
These values may however be influenced by the H3 pipeline~\textsc{MINESweeper}
\citep{Cargile2020}, which includes a prior enforcing~$\afe \leq +0.6$ -- if
the~\afe~plateau occurs near this value in nature, this prior could bias the
most alpha-rich stars in our sample to slightly lower~\afe~ratios.
\par
Red lines in Fig.~\ref{fig:gse} illustrate our best-fit model compared to the
data
Visually, this model is a reasonable description of the data, though in detail
it predicts a slightly broader~\feh~distribution and a slightly more peaked age
distribution.
We assess the quality of the fit with equation~\refp{eq:chisquared_dof} and
find~$\chi_\text{dof}^2 = 1.34$, suggesting that this is indeed a good fit but
that there may be some marginal room for improvement.
The substantial scatter in the age-metallicity relation (lower right panel)
arises due to the age uncertainties -- to demonstrate this, we subsample 95
stars (the same number in our sample with age measurements) from our best-fit
SFH and perturb their implied ages and abundances by the median observational
uncertainties.
These random draws (red points) occupy a very similar region of the age-\feh~and
age-\afe~planes.
We do however note an additional~$\sim$6 or 7 potential blue stragglers with
ages of~$\sim$$8 - 9$ Gyr,~$\feh \approx -1.2$ and~$\afe \approx +0.4$.
These stars are less obviously blue stragglers than the~$\sim$5 and~$\sim$6 Gyr
old ones and would not have stood out without this comparison.
These stars likely play a role in increasing the~$\chi_\text{dof}^2$ of our
fit, and removing them from our sample would also bring the observed age
distribution into better agreement with our best-fit model.
We however do not explore more detailed investigations of individual stars for
fits to carefully tailored populations here, and the fit we obtain is
statistically reasonble anyway.
\par
In~\S~\ref{sec:mocks:variations}, we found that our model accurately recovered
the evolutionary timescales of the input model even in the absence of age
information due to their impact on the shape of the MDF.
To assess the feasibility of deducing these parameters from abundances alone,
we conduct an additional fit to our GSE sample omitting the age measurements.
We report the best-fit parameters in Table~\ref{tab:results}.
This procedure results in accurate fits to the~\feh~and~\afe~distributions, and
the SN yields and mass-loading factor~$\eta$ are generally consistent with
and without ages.
The inferred timescales are discrepant by~$\sim$$2\sigma$, and the duration of
star formation shows the largest difference.
These results indicate that such an approach is theoretically possible, but in
practice age information in some form is essential to pinning down these
timescales.
In~\S~\ref{sec:mocks}, we fit our mock samples with the exact underlying GCE
model and same numerical code which integrated the input model, placing the
same systematic effects in the data as the model.
It is also never guaranteed that the evolutionary history built into the model
is an accurate description of the galaxy.
% While the most powerful way to provide age information for a fit under this
% procedure is with star-by-star measurements, that is difficult for external
% galaxies with current instrumentation and could instead be achieved with, e.g.,
% a CMD-derived SFH included as a prior.

\subsection{Wukong}
\label{sec:h3:wukong}

% fig 9
\begin{figure}
\centering
\includegraphics[scale = 0.6]{gse_wukong_timescales.pdf}
\caption{
A comparison of the best-fit timescales for GSE and Wukong.
}
\label{fig:gse_wukong_timescales}
\end{figure}

% fig 11
\begin{figure}
\centering
\includegraphics[scale = 0.6]{gse_wukong_eta.pdf}
\caption{
A comparison of our best-fit mass-loading factors~$\eta$ between the GSE and
Wukong.
}
\label{fig:gse_wukong_eta}
\end{figure}

% fig 10
\begin{figure*}
\centering
\includegraphics[scale = 0.45]{gse_wukong_comparison.pdf}
\caption{
A comparison of the predictions of our best-fit GCE models describing our
GSE (red) and Wukong (blue) data: the age distributions (left), the
age-\feh~relations (middle) and age-\afe~relations (right).
The inset in the right hand panel shows the tracks in the~\afe-\feh~plane.
In all panels, we subsample 200 additional parameter choices from our Markov
chains and plot the predictions as high transparency lines to provide a sense
of the fit uncertainty.
Due to the lack of age information for Wukong, the results in each panel with
the exception of the inset can shift left or right without impacting the
quality of the fit.
}
\label{fig:comparison}
\end{figure*}

Our Wukong sample consists of 57 stars, none of which have age information as
they are all distant halo stars.
Abundance uncertainties range from~$\sim$0.02 to~$\sim$0.10 dex in
both~\afe~and~\feh~with median values near~$\sim$0.045.
Fig.~\ref{fig:wukong} illustrates this sample in chemical space along with our
best-fit GCE model (see discussion below).
Similar to the GSE, the lack of discontinuities in the age and abundance trends
indicates a smooth SFH devoid of any starburst events.
We therefore fit this sample with the same exponential infall history as the
input model to our mock samples, which we also applied to our GSE data.
We retain the assumption that star formation began 13.2 Gyr ago and that the H3
selection function is uniform in chemical space (see discussion
in~\S~\ref{sec:h3:gse}).
However, due to the smaller sample size and the lack of age information, we
initially hold our Fe yields fixed at~$\yfecc = \scinote{7.78}{-3}$ and
$\yfeia = \scinote{1.23}{-3}$ as suggested by the fit to our GSE sample.
It is reasonable to expect SN yields to be the same from galaxy-to-galaxy since
they are set by stellar as opposed to galactic physics, though we explore the
impact of relaxing this assumption below.
\par
Table~\ref{tab:results} reports the inferred best-fit parameters and
Fig.~\ref{fig:wukong_corner} illustrates the posterior distributions.
The degeneracies between parameters are noticeably more asymmetric than in our
GSE sample, a result of the lack of age information (we found similar effects
in our tests against mock data in~\S~\ref{sec:mocks}, though we did not discuss
it there).
The e-folding timescale of the accretion rate in particular has a highly skewed
likelihood distribution ($\tau_\text{in} = 3.08^{+3.19}_{-1.16}$ Gyr).
We have also had reasonable success describing Wukong with a constant star
formation history.
Consequently, the likelihood function has a tail that extends
to~$\tau_\text{in} \rightarrow \infty$.
The exponential infall history is indeed a statistically better fit, so
throughout this section we include a prior that
enforces~$\tau_\text{in} \leq 50$ Gyr to focus on this portion of parameter
space.
This tail is significantly more extended if the Fe yields are allowed to vary
as a free parameter (see Table~\ref{tab:results} and discussion below).
\par
An exponential infall history yields a statistically excellent fit
($\chi_\text{dof}^2 = 0.98$; equation \ref{eq:chisquared_dof}) for Wukong,
though visually it appears that the SN yields implied by our GSE data
underestimate the height of the [$\alpha$/Fe] plateau, which we indirectly
held fixed via the Fe yields.
Although we asserted above that it is reasonable expect SN yields to be the
same between Wukong and GSE, variations in the plateau height could indicate
either metallicity-dependent yields or variations in the IMF.
To investigate this, we conduct an additional fit where we allow the Fe yields
to vary as free parameters, reporting the results in Table~\ref{tab:results}
and illustrating the deduced model for comparison in Fig.~\ref{fig:wukong}.
A higher plateau indeed provides an even better fit
($\chi_\text{dof}^2 = 0.84$), but with~$\chi_\text{dof}^2$ less than 1, this
could be an overparametrization of the data.
This possibility is not necessarily to a worrisome extent though; we cannot
rule out either model.
The best-fit SFE timescales between the two fits are in excellent agreement,
indicating that~$\tau_\star$ does not significantly impact the height of the
plateau (to first-order, it determines the position of the knee in the
track;~\citealp{Weinberg2017}).

\subsection{Comparison}
\label{sec:h3:comparison}

Fig.~\ref{fig:gse_wukong_timescales} compares the best-fit evolutionary
timescales between GSE and Wukong as a function of their stellar mass (we adopt
the stellar masses inferred by~\citealt{Naidu2021, Naidu2022}; our GCE models
as we have parametrized them do not offer any constraints on this quantity).
Due to the yield-outflow degeneracy (see Appendix~\ref{sec:degeneracy}), only
relative values of~$\tau_\star$ carry meaning, while the absolute values of
$\tau_\text{in}$ and~$\tau_\text{tot}$ do.
Qualitatively consistent with semi-analytic models of galaxy formation
\citep[e.g.,][]{Baugh2006, Somerville2015a, Behroozi2019} and results from
hydrodynamical simulations~\citep[e.g.,][]{GarrisonKimmel2019}, the less
massive of the two galaxies experienced the more extended accretion history.
Star formation in Wukong, however, was less efficient and did not last as long
as in GSE -- sensible results given the empirical correlation between
stellar-to-halo mass ratioes and stellar mass~\citep{Hudson2015}.
To the extent that our one-zone model framework is accurate, we have
constrained the duration of star formation in Wukong and GSE to 15.2\% and
5.8\%, respectively.
However, our Wukong sample has no age measurements, and we have not derived
an SFH from its CMD here.
The failure of our fit to GSE omitting all ages (see Table~\ref{tab:results})
suggests that these best-fit parameters may need revised if some form of age
information is added to the fit.
\par
As expected given Wukong's shallower gravity well, it experienced stronger
mass-loading than GSE.
Fig.~\ref{fig:gse_wukong_eta} demonstrates this in comparison to Muratov et
al.'s~\citeyearpar{Muratov2015} scaling of~$\eta$ with stellar mass computed
from the FIRE simulation suite~\citep{Hopkins2014}.
There is excellent agreement between their calculations and our one-zone model
fits, rather remarkably so given that quantifying outflows in simulations
requires choices to be made regarding what constitutes an outflow.
\par
In Fig.~\ref{fig:comparison}, we compare our best-fit models for GSE and Wukong.
The intrinsic age distribution of GSE is predicted with considerably higher
precision than for Wukong, a consequence of the lack of age information in our
Wukong sample.
The uncertainties in the Wukong age distribution are noticably asymmetric due
to the skewed posterior distribution of the infall timescale
($\tau_\text{in} = 3.08^{+3.19}_{-1.16}$ Gyr).
If our assumption that star formation began~$T \approx 13.2$ Gyr ago (see
discussion in~\S~\ref{sec:mocks:fiducial}) is accurate for Wukong, then it
experienced quenching~$\sim$2 Gyr earlier than the GSE ($\sim$9.8 versus
$\sim$7.8 Gyr ago).
However, because we do not have age information for Wukong, this distribution
could shift uniformly to lower values with affecting the quality of the fit.
Constraints on the centroid of the distribution could be derived by
analysing the CMD as in, e.g.,~\citet{Dolphin2002} and~\citet{Weisz2014b}, but
we do not pursue this in the present paper as it involves an entirely separate
mathematical framework.
\par
Also as a consequence of the lack of age information, our fits constrain the
intrinsic age-\feh~and age-\afe~relations to somewhat higher precision for GSE
than Wukong.
While the age-\feh~relations are significantly offset from one another, the
predicted age-\afe~relations are remarkably consistent with one another.
A portion of this agreement can likely be traced back to our fixing the Fe
yields in our fit to Wukong to the values inferred in our fit to GSE.
Nonetheless, it is reasonable to assume that the SN yields are the same between
the two galaxies because this should be set by stellar physics, sufficiently
decoupled from the galactic environment.
The evolution of~\afe~with time is in principle impacted by the various
evolutionary timescales at play, so their consistency with one another is still
noteworthy.

\end{document}
