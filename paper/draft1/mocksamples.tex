
\documentclass[ms.tex]{subfiles}
\begin{document}

\section{Mock Samples}
\label{sec:mocks}

Using our parametrization of one-zone GCE models described
in~\S~\ref{sec:onezone}, here we define a set of parameter choices from which
mock samples of stars can be drawn.
We then demonstrate the validity of our likelihood function (equation
\ref{eq:likelihood}) in~\S~\ref{sec:mocks:recovered} by applying it to a
fiducial mock sample and comparing the best-fit values to the known parameters
of the input model.
In~\S~\ref{sec:mocks:variations}, we then explore variations in sample size,
measurement precision, and the availability of age information.

\subsection{A Fiducial Mock Sample}
\label{sec:mocks:fiducial}

% fig 1
\begin{figure*}
\centering
\includegraphics[scale = 0.5]{fiducial_mock_afe_feh.pdf}
\includegraphics[scale = 0.42]{fiducial_mock_agedist.pdf}
\includegraphics[scale = 0.41]{fiducial_mock_amr.pdf}
\caption{
\textbf{Left}: Our fiducial mock sample in the~\afe-\feh~plane.
There are~$N = 500$ stars with abundance uncertainties of~$\sigma_\feh =
\sigma_\afe = 0.05$ as indicated by the errorbar.
$N = 100$ of the stars have age information as indicated by the colorbar with
an artificial uncertainty of~$\sigma_{\log_{10}(\text{age})} = 0.1$.
On the top and right, we show the marginalized distributions in~\afe~and~\feh,
with red lines denoting the known distribution.
\textbf{Center}: The age distribution of the mock sample (black, binned).
The dashed red line indicates the age distribution that is obtained by sampling
$N = 10^4$ rather than~$N = 500$ stars from the input model and assuming the
same age uncertainty.
\textbf{Right}: The age-\feh~(top) and age-\afe~(bottom) relations for the
mock sample with artificial uncertainties denoted by the error bars on each
panel.
Solid red and blue lines in all panels denote the known relations from the
input model and the recovered best-fit model, respectively.
}
\label{fig:fiducial_mock}
\end{figure*}

% fig 2
\begin{figure*}
\centering
\includegraphics[scale = 0.5]{fiducial_512k.pdf}
\caption{
The ``corner-plot'' showing the results of applying our fitting method to our
fiducial mock sample (see discussion in~\S\S~\ref{sec:fitting} and
\ref{sec:mocks:fiducial}).
Panels below the diagonal show 2-dimensional cross-sections of the likelihood
function while panels along the diagonal show the marginalized distributions
along with the best-fit values and confidence intervals.
Blue stars mark the element of the Markov chain with the maximum likelihood.
Red ``cross-hairs'' denote the true, nown values of the parameters from the
input model (see the top row of Table X).
}
\label{fig:fiducial_mock_corner}
\end{figure*}

We take an exponential infall history~$\dot{M}_\text{in} \propto e^{-t /
\tau_\text{in}}$ with an e-folding timescale of~$\tau_\text{in} = 2$ Gyr and an
initial ISM mass of~$M_\text{g} = 0$.
We select an SFE timescale of~$\tau_\star = 15$ Gyr, motivated by the
observational result that dwarf galaxies have generally inefficient star
formation~\citep[e.g.][]{Hudson2015}.
We additionally select a mass-loading factor of~$\eta = 10$ because the
strength of outflows should, in principle, contain information on the depth of
the gravity well of a given galaxy, with lower mass systems being more
efficient at ejecting material from the ISM.
If the SFH in this model were constant, the analytic formulae of
\citet{Weinberg2017} suggest that the equilibrium alpha element abundance
should be~$\sim16$\% of the solar oxygen abundance, in qualitative agreement
with the empirical mass-metallicity relation for galaxies
(\citealp{Tremonti2004, Gallazzi2005};~\citealp*{Zahid2011};
\citealp{Andrews2013, Kirby2013, Zahid2014}).
\par
With these choices regarding~$\tau_\star$ and~$\eta$, our parameters are in
the regime where the normalization of the infall history, and consequently the
SFH, is inconsequential to the predicted evolution of the abundances.
The appropriate likelihood function is therefore equation~\refp{eq:likelihood}
with normalized weights, whereas equation~\refp{eq:lnL_minus_weights} with
un-normalized weights would be the proper form if we had selected a
parametrization in which the absolute scale of the SFH impacts the enrichment
history.
Inspection of the average SFHs of predicted by the~\textsc{UniverseMachine}
semi-analytic model for galaxy formation~\citep{Behroozi2019} suggests that the
onset of star formation tends to occur a little over~$\sim$13 Gyr ago across
many orders of magnitude in stellar mass extending as low as
$\text{M}_\star \approx 10^{7.2}~\msun$.
We therefore assume that the onset of star formation occurred~$\sim$13.2 Gyr
ago, allowing~$\sim$500 Myr between the Big Bang and the first stars.
We evolve this model for 10 Gyr exactly (i.e. the youngest stars in the mock
sample have an exact age of 3.2 Gyr), stopping short of 13.2 Gyr because
surviving dwarf galaxies and stellar streams often have their star formation
quenched (e.g.~\citealp{Monelli2010a, Monelli2010b, Sohn2013, Weisz2014a,
Weisz2014b, Weisz2015}).
These choices are not intended to resemble any one galaxy, but instead to
qualitatively resemble some dwarf galaxy whose evolutionary parameters can
be re-derived using our likelihood function as a sanity check that it produces
accurate best-fit parameters.
\par
As discussed in~\S~\ref{sec:onezone}, thoughout this paper we assume that the
IMF-averaged alpha element yield is exactly~$\yacc = 0.01$ and
$y_\alpha^\text{Ia} = 0$.
While loosely motivated by nucleosynthesis models in massive stars
\citep[e.g.][]{Nomoto2013, Sukhbold2016, Limongi2018}, this choice is intended
to set some normalization of the effective yields which can be scaled up or
down to accommodate alternative choices.
If no scale is assumed, then extremely strong degeneracies arise in the
inferred yields, the strength of outflows~$\eta$, and the SFE timescale
$\tau_\star$ due to the yield-outflow degeneracy (see discussion in
Appendix~\ref{sec:degeneracy}).
We do not distinguish between alpha elements in this validation of our
likelihood function because, from a modelling standpoint, they can all be
treated the same with a metallicity-independent yield from CCSNe and negligible
yields from all other sources (at least for the lighter alpha elements such as
O and Mg;~\citealp{Johnson2019}).
In practice, however, we take O as the canonical alpha element when integrating
these models with~\vice, adopting~$Z_{\text{O},\odot} = 0.00572$ as the
abundance of O in the sun according to~\citet{Asplund2009} and consistent with
the recent revisions of~\citet*{Asplund2021}, though similar~\afe~ratios would
arise anyway if we instead took, e.g., Mg and asserted that [O/Mg]~$\approx 0$.
\par
\citet{Weinberg2017} adopt~$\yacc = 0.014$,~$\yfecc = 0.0012$ and
$\yfeia = 0.0017$ (see discussion in their~\S~2.2).
This massive star yield of Fe is appropriate for nucleosynthesis models in
which most~$M > 8~\msun$ stars explode as a CCSN~\citep[e.g.][]{Woosley1995,
Chieffi2004, Chieffi2013, Nomoto2013} assuming a~\citet{Kroupa2001} IMF.
This SN Ia yield of Fe is based on the W70 explosion model of
\citet{Iwamoto1999} which produces~$\sim$0.77~\msun~of Fe per SN Ia event and
assuming that~$\scinote{2.2}{-3}~\msun^{-1}$ SNe Ia arise per solar mass of
star formation based on~\citet{Maoz2012a}.
Following this, we scale these yields down by factors of~$\sim$2/3 such that
$\yacc = 0.01$, adopting~$\yfecc = \scinote{8}{-4}$ and
$\yfeia = \scinote{1.1}{-3}$ in our mock samples.
We retain the assumption that~$\yacc = 0.01$ in our fits to our mock samples
but otherwise let the Fe yields~\yfecc~and~\yfeia~be free parameters to be
recovered by our likelihood function.
We use this procedure in our application to the H3 survey in~\S~\ref{sec:h3}
below as well.
We then sample~$N = 500$ stars from the underlying SFH each of which have -- in
the interest of mimicking the typical precision achieved by a spectroscopic
survey of a local group dwarf galaxy --~$\sigma_\afe = \sigma_\feh = 0.05$.
100 of these stars have age measurements with an uncertainty of
$\sigma_{\log_{10}(\text{age})} = 0.1$ (i.e.~$\sim$23\% precision).

\subsection{Recovered Parameters of the Fiducial Mock}
\label{sec:mocks:recovered}

We now apply the method outline in~\S~\ref{sec:fitting} to the mock sample
detailed in~\S~\ref{sec:mocks:fiducial}.
Fig.~\ref{fig:fiducial_mock_corner} shows the ``corner-plot'' derived from this
procedure with the marginalized likelihood distributions along the diagonal and
2-dimensional cross sections below the diagonal comparing the best-fit values
with their known values.
Our likelihood function accurately recovers each of the model parameters.
We include the predictions of the best-fit model in Fig.~\ref{fig:fiducial_mock},
finding excellent agreement with the input model.
To quantify the quality of the fit, for each datum~$\script{D}_i$ we find the
point along the track~$\script{M}_j$ with the maximum likelihood of observation
(i.e.~$\{\script{D}_i,\script{M}_j~|~\ln L(\script{D}_i | \script{M}_j) =
\max(\ln L(\script{D}_i | \script{M}))\}$).
We then compute the chi-squared per degree of freedom diagnostic according to
\begin{equation}
\chi_\text{dof}^2 = \frac{1}{N_\text{obs} - N_\theta}
\sum_{i,j} \Delta_{ij} C_i^{-1} \Delta_{ij}^T,
\label{eq:chisquared_dof}
\end{equation}
where~$N_\text{obs}$ is the number of quantities in the observed sample,
$N_\theta$ is the number of model parameteres, and the summation is taken over
the pair-wise combinations of the data and model with the maximum likelihood of
observation.
Although marginalizing over the track~\script{M}~is necessary to derive
accurate best-fit parameters (see discussion below and in~\S~\ref{sec:fitting}),
it should be safe to estimate the quality of a fit by simply pairing each datum
with the most appropriate point on the track.
As noted in the middle panel of Fig.~\ref{fig:fiducial_mock}, our method
achieves~$\chi_\text{dof}^2 = 0.55$, indicating that we have perhaps
over-parametrized the data.
This result is unsurprising, however, because we have fit the mock data with
the exact, known parametrization of the evolutionary history and
nucleosynthetic yields of the input model in the interest of demonstrating
proof of concept that equation~\refp{eq:likelihood} provides accurate best-fit
values.
\par
Although it may appear that there are a worrying number of~$\gtrsim 1\sigma$
discrepancies in Fig.~\ref{fig:fiducial_mock_corner}, we demonstrate
in~\S~\ref{sec:mocks:variations} below that the differences between the known
and best-fit values here are consistent with randomly sampling from a Gaussian
distribution due to measurement uncertainty.
There are a handful of degeneracies in the likelihood distribution of the
recovered parameters which arise as a consequence of having an impact on the
same observable.
We discuss them individually below.
\par
\textit{The centroid of the MDF.}
With a fixed alpha element yield of~$\yacc = 0.01$ as we have adopted here,
the strength of outflows~$\eta$ is set by ensuring the centroid of
the~\ah~distribution is in agreement with the data.
This in turn requires a total Fe yields of~$\yfecc + \yfeia$ which corresponds
to the centroid of the~\feh~distribution.
The total Fe yield, however, can be achieved with different breakdowns between
CCSN and SN Ia enrichment, giving rise to the inverse relationship
between the two Fe yields.
\par
\textit{The height of the ``plateau'' and position of the ``knee'' in the
evolutionary track.}
The plateau in the~\afe-\feh~plane occurs in our input model
at~$\afe_\text{CC} \approx +0.45$ and arises due to the IMF-averaged
massive star yields of alpha and iron-peak elements.
The knee occurs thereafter with the onset of SN Ia enrichment, a
nucleosynthetic source of Fe but negligible amounts of alpha elements like O
and Mg~\citep{Johnson2019}.
For a fixed choice of yields,~\citet{Weinberg2017} demonstrate that the SFE
timescale~$\tau_\star$ plays the dominant role in establishing the value
of~\feh~at which the knee occurs (see discussion in~\S~\ref{sec:onezone}).
With the knee in an observed position,~\yfecc~can be lowered in exchange for
faster star formation to reach the same abundance before the onset of the
first SNe Ia.
This relationship gives rise to the direct degeneracy between~$\tau_\star$
and~\yfecc~in Fig.~\ref{fig:degeneracy} in Appendix~\ref{sec:degeneracy} where
we have relaxed our assumption that~$\yacc = 0.01$.
In Fig.~\ref{fig:fiducial_mock_corner}, however, we see the opposite
degeneracy.
At fixed~\yacc, variations in~\yfecc~adjust the height of the plateau
$\afe_\text{CC}$.
With a lowered plateau (i.e. higher~\yfecc), the track moves uniformly downward
in the~\afe-\feh~plane.
If this adjustment is accompanied by faster star formation (i.e.
lower~$\tau_\star$), the knee will occur at a slightly higher~\feh, placing the
portion of the evolutionary track in which~\afe~is decreasing in a similar
region of abundance space.
This gives rise to the inverse degeneracy between~\yfecc~and~$\tau_\star$ in
Fig.~\ref{fig:fiducial_mock_corner} when an overall scale of nucleosynthetic
yields is chosen.
\par
\textit{The location of the track in abundance space.}
This observable which plays an important role in producing the inverse
degeneracy between~\yfecc~and~$\tau_\star$ in Fig.
\ref{fig:fiducial_mock_corner} is also related to the SN Ia yield~\yfeia.
Similar to~\yfecc~impacting the height of the plateau, adjustments to the
value of~\yfeia~move the track vertically in the~\afe-\feh~plane when other
parameters are held fixed (there is horizontal movement as well, though the
vertical movement is stronger).
In combination with the yields themselves, the outflow mass-loading factor
$\eta$ establishes the late-time equilibrium abundance (\citealp{Weinberg2017};
see discussion in~\S~\ref{sec:onezone}).
Adjusting~$\eta$ shifts the track horizontally in the~\afe-\feh~plane by
changing the metal mass retained by the ISM.
Therein lies the origin of the degeneracy between~\yfeia~and~$\eta$: a
downward shift in the evolutionary track can be paired with a rightward shift
in order to place the track in a similar location in abundance space to achieve
similar agreement with the data.
\par
\textit{The shape of the MDF.}
The~\ah~and~\feh~distributions are affected in a handful of ways by the
parameters of this input model.
The duration of star formation has the simplest effect of cutting off the MDF
at some abundance.
Inefficient star formation (i.e. high~$\tau_\star$) increases the frequency of
low metallicity stars because it takes significantly longer for the ISM to
reach the equilibrium abundance.
Sharp infall histories (i.e. low~$\tau_\text{in}$) predict wide MDFs because
the ISM mass declines with time through losses to star formation and the lack
of replenishment by accretion.
Metals are then deposited into a ``gas-starved'' reservoir which then reaches
higher abundances due to a deficit of hydrogen and helium.
This effect is particularly strong for Fe because of the delayed nature of SN
Ia enrichment~\citep{Weinberg2017}.
These models achieve higher metallicities in the ISM, but their declining SFHs
produce a larger fraction of their stars early in their evolutionary history
when the abundances are lower than the late-time equilibrium abundance.
Consequently, the MDF which arises is wider for sharp infall histories but has
a peak in a similar position regardless of~$\tau_\text{in}$.
Folding these effects together, degeneracies arise in the inferred parameters
as a consequence of their effects on the MDF.
Between~$\tau_\text{in}$ and~$\tau_\text{tot}$, a sharp infall history can
broaden the MDF, but cutting off star formation earlier can allow the
distribution to remain peaked if the data suggest it.
Similarly, efficient star formation (i.e. low~$\tau_\star$) allows the ISM to
spend more time near its equilibrium abundance, enhancing the peak of the MDF,
but this change in shape can be reverted by cutting off star formation.
Between~$\tau_\text{in}$ and~$\eta$, a sharp infall history gives rise to a
high metallicity tail of the MDF, but increasing the strength of outflows
can lower the overall metallicity if this tail is too metal-rich compared to
the data.
\par
We clarify that our fits achieve this level of precision by selecting an
overall scale for nucleosynthetic yields and outflows ($\yacc = 0.01$; see
discussion in~\S~\ref{sec:onezone} and Appendix~\ref{sec:degeneracy}).
Any GCE parameter which influences the centroid of the MDF or the position or
shape of the evolutionary track in abundance space is subject to the
yield-outflow degeneracy.
This includes all evolutionary parameters in our input model with the exception
of~$\tau_\text{in}$ and~$\tau_\text{tot}$ because they impact only the shape of
the MDF, an observational diagnostic which is constrained by a sufficiently
large sample.
We discuss the implications of this result applied to observed data
in~\S~\ref{sec:mocks:variations} below.
\par
These tests against mock samples demonstrate the importance of the two central
features of this method -- weighting the likelihood of observation by the SFH
in the model and marginalizing over the evolutionary track for each datum.
When either the weights or the marginalization are excluded from equation
\refp{eq:likelihood}, the fit fails to recover the parameters of the input
model with discrepancies at the many-$\sigma$ level between the best-fit and
known values.
For this reason, we caution against the reliability of GCE parameters inferred
from simplified likelihood estimates, such as matching each datum with the
nearest point on the track.

\subsection{Variations in Sample Size, Measurement Precision and the
Availability of Age Information}
\label{sec:mocks:variations}

\subfile{mocksamples.tablebody.tex}

% fig 3
\begin{figure*}
\centering
\includegraphics[scale = 0.42]{dp_sigma_samplesize.pdf}
\includegraphics[scale = 0.42]{dp_sigma_precision.pdf}
\includegraphics[scale = 0.42]{dp_sigma_agefrac.pdf}
\caption{
The mean deviation between the re-derived parameters
$\{\theta\} = \{\tau_\text{in}, \eta, \tau_\star, \tau_\text{tot}, \yfecc,
\yfeia\}$ and their known values from the input model in units of the
uncertainty on the best-fit values.
We show the mean offset as a function of the sample size (left), measurement
precision in~\feh~and~\afe~abundances (middle, black), measurement precision
in~$\log_{10}(\text{age})$ (middle, red), and the fraction of the sample with
available age information (right).
Error bars denote the error in the mean deviation of the six parameters.
Blue dotted lines mark~$\langle \Delta \theta / \sigma \rangle = 1$, the
expected mean offset due to randomly sampling from a Gaussian distribution.
}
\label{fig:accuracy}
\end{figure*}

% fig 4
\begin{figure*}
\centering
\includegraphics[scale = 0.55]{precision_samplesize.pdf}
\includegraphics[scale = 0.55]{precision_agefrac.pdf}
\includegraphics[scale = 0.55]{precision_abundanceuncertainty.pdf}
\includegraphics[scale = 0.55]{precision_ageuncertainty.pdf}
\caption{
The precision (i.e. $\left|\Delta \theta\right| / \sigma$) of our re-derived
mock sample parameters~$\{\theta\} = \{\tau_\text{in}, \eta, \tau_\star,
\tau_\text{tot}, \yfecc, \yfeia\}$ as a function of sample size (top left),
the fraction of the sample with age measurements (top right), measurement
uncertainty in~\feh~and~\afe~(bottom left), and measurement uncertainty
in~$\log_{10}(\text{age})$ (bottom right).
We plot timescales in red, Fe yields in blue, and the mass-loading
factor~$\eta$ in black in all panels according to the legend.
}
\label{fig:precision}
\end{figure*}

We now explore variations of our fiducial mock sample.
We retain the same evolutionary parameters of the input model (see discussion
in~\S~\ref{sec:mocks:fiducial}), but each variant differs in one of the
following:
\begin{itemize}

	\item Sample size.

	\item Measurement precision in~\feh~and~\afe.

	\item Measurement precision in~$\log_{10}(\text{age})$.

	\item The fraction of the sample that has age measurements.

\end{itemize}
The left-hand column of Table~\ref{tab:recovered_values} provides a summary of
the values we take as exploratory cases with the fiducial mock marked in bold.
In the remaining columns, we provide the associated values derived for each
GCE parameter~$\theta$ along with their~$1\sigma$ confidence intervals.
The sample sizes we consider are intended to reflect the range that is
typically achieved in dwarf galaxies.
Much more distant than nearby Galactic stars and intrinsically smaller, dwarf
galaxies are considerably less conducive to the large sample sizes achieved by
Milky Way surveys like APOGEE~\citep{Majewski2017} and GALAH
% \footnote{
% 	APOGEE: Apache Point Observatory Galaxy Evolution Experiment. \\
% 	GALAH: GALactic Archaeology with Hermes.
% }~\citep{DeSilva2015, Martell2017}.
\citep{DeSilva2015, Martell2017}.
Our choices in measurement precision are intended to reflect typical values
achieved by modern spectroscopic surveys.
Although deriving elemental abundances through spectroscopy is a nontrivial
problem known to be affected by systematics~\citep[e.g.][]{Anguino2018},
stellar age measurements
are generally the more difficult of the two~\citep{Soderblom2010, Chaplin2013}.
The age measurements may therefore be available for only a small portion of the
sample and are often less precise than the abundances ($f_\text{age} = 30$\%
and~$\sigma_\feh = \sigma_\afe = 0.05$ versus
$\sigma_{\log_{10}(\text{age})} = 0.1$ in our fiducial mock).
\par
Fig.~\ref{fig:accuracy} demonstrate the accuracy of our fitting method with
respect to variations in these details surrounding the data.
We compute the deviation between each re-derived parameter~$\theta$
(i.e.~$\tau_\text{in}$,~$\eta$,~$\tau_\star$, etc.) and its known value from
the input model, then divide by the fit uncertainty~$\sigma_\theta$ and plot
the mean on the y-axis.
Under all variants that we explore, our likelihood function accurately recovers
the input parameters to~$\sim1\sigma$ or slightly better.
This is exactly as expected when the uncertainties are described by a
Gaussian random process, wherein the most likely deviation from the true value
is exactly~$1\sigma$.
This is true even with infinite data, though in that limit the~$1\sigma$
uncertainty interval becomes arbitrarily small.
This demonstrates that equation~\refp{eq:likelihood} provides accurate best-fit
parameters even when the sample size is as low as~$N \approx 20$, when the
measurement uncertainties are as imprecise as~$\sigma_\text{[X/Y]} \approx 0.5$
and~$\sigma_{\log_{10}(\text{age})} \approx 1$, or even when there is no age
information available at all.
The precision of the fit will indeed suffer in such cases (see Fig.
\ref{fig:precision} and associated discussion below), but the inferred
parameters will remain accurate nonetheless.
\par
We have explored alternate parametrizations of our mock sample's evolutionary
history and indeed found that our method accurately recovers the parameters
in all cases.
To name a couple of these ``stress tests,'' one is a case in which we build in
a significant starburst, finding that we accurately recover both the timing and
the strength of the burst.
We have also explored an infall rate which varies sinusoidally about some mean
value, mimicking natural fluctuations in the accretion history or a series of
minor starbursts.
Although idealized and potentially unrealistic, our likelihood function
accurately recovers the amplitude, phase and frequency in this case as well.
\par
Fig.~\ref{fig:precision} demonstrates how the uncertainty of each best-fit
parameter is affected by these details of the sample.
With differences in the normalization, the precision of each inferred parameter
scales with sample size approximately as~$N^{-0.5}$.
In general, the mass-loading factor~$\eta$ and the Fe yields are constrained
more precisely than the timescales.
The primary exception to this rule is when the abundance uncertainties are
large compared to the age uncertainties, in which case the Fe yields are
constrained to a similar precision as~$\tau_\text{in}$ and~$\tau_\star$
but~$\tau_\text{tot}$ is determined more precisely.
The Fe yields are, unsurprisingly, the most sensitive parameters to the
abundance uncertainties, while~$\eta$ can be determined by~$\sim$10\% precision
even with highly imprecise measurements ($\sigma_\text{[X/Y])} \approx 0.5$).
Even with imprecise abundances, the centroid of the MDF can still be robustly
determined with a sufficiently large sample, which allows a precise inference
of the strength of winds due to its impact on the equilibrium metallicity (for
an assumed scale of nucleosynthetic yields such as~$\yacc = 0.01$ in this
paper).
\par
Only the inferred timescales are impacted by the availability of age
information and the uncertainties thereof.
Even with order of magnitude uncertainties in stellar ages, however, the
evolutionary timescales of our mock samples are recovered to~$\sim$20\%
precision.
Interestingly the introduction of age information to the sample impacts the
fit uncertainty only for~$f_\text{age} \lesssim 30$\%.
Above this value, there is only marginal gain in the precision of best-fit
timescales.
These results suggest that authors seeking to determine best-fit evolutionary
parameters for one-zone models applied to any sample should focus their efforts
on sample size and precise abundance measurements with age information being
a secondary consideration.
Thankfully, abundances are generally easier than ages to measure on a
star-by-star basis~\citep{Soderblom2010, Chaplin2013}.
\par
These results are of particular interest to authors seeking to derive quenching
times for dwarf galaxies (i.e. the lookback time to when their star formation
stopped).
At present, the most reliable method to empirically determine a dwarf galaxy's
quenching time is via a direct reconstruction of its SFH through some method,
such as analysing its CMD~\citep[e.g.][]{Sohn2013, Weisz2015}.
Consequently, the most precise SFH measurements are for nearby systems with
resolved stars, a considerable limitation even with modern instrumentation.
To our knowledge, there are only four quenched galaxies outside of the Milky
Way subgroup with well-constrained SFHs: Andromeda II, Andromeda XIV
\citep{Weisz2014a}, Cetus~\citep{Monelli2010a} and Tucana~\citep{Monelli2010b}.
Some authors have connected quenching timescales to observed galaxy properties
in N-body simulations (e.g.~\citealp*{Rocha2012};~\citealp{Slater2013,
Slater2014, Phillips2014, Phillips2015, Wheeler2014}), but unfortunately
simulation outcomes are strongly dependent on the details of the adopted
sub-grid models~\citep[e.g.][]{Li2020} as well as how feedbak and the grid
itself are implemented~\citep{Hu2022}.
Our results suggest that the duration of star formation can instead be inferred
from elemental abundance alone, though having~\textit{some} age information can
significantly improve the precision of the fit.
While this is theoretically possible, we demonstrate in~\S~\ref{sec:h3:gse}
below that the exclusion of age information from the fit to our GSE sample
produces significantly different results for evolutionary timescales.
These discrepancies indicate that age information in some form is more
important in practice than this proof of concept would suggest.
% While this is theoretically possible, this proof of concept assumes the same
% evolutionary history of the mock data and even fits them with the same
% numerical code which which we integrated the input model.
% The model and data are therefore affected by the exact same systematics, while
% in practice the systematics affecting observed data and a one-zone GCE model
% are different.
% In~\S~\ref{sec:h3:gse} below, we demonstrate that the exclusion of age
% information from the fit to our GSE sample produces different results for
% evolutionary timescales.

\end{document}
