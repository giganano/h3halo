
\documentclass[ms.tex]{subfiles}
\begin{document}

\section{The Fitting Method}
\label{sec:fitting}

Our fitting method uses the abundances and ages (where available) of an
ensemble of stars and, with no binning of the data, accurately constructs the
\textit{likelihood function}~$L(\script{D} | \{\theta\})$ describing the
probabiliy of observing the data~$\script{D}$ gien a set of model parameters
$\{\theta\}$.
This is related to the~\textit{posterior probability}
$\L(\{\theta\} | \script{D})$ according to Bayes' Theorem:
\begin{equation}
L(\{\theta\} | \script{D}) = \frac{
	L(\script{D} | \{\theta\}) L(\{\theta\})
}{
	L(\script{D})
},
\label{eq:bayes}
\end{equation}
where~$L(\{\theta\})$ is the likelihood of the parameters themselves (known as
the~\textit{prior}) and~$L(\script{D})$ is the likelihood of the data (known as
the~\textit{evidence}).
Although it is more desirable to measure the posterior probability, in practice
only the likelihood function can be robustly determined because the prior is
not directly quantifiable.
The prior requires quantitative information independent of the data on the
accuracy of a chosen set of parameters~$\{\theta\}$.
With no additional information on what the parameters should be, the best
practice is to assume a ``flat'' or ``uniform'' prior in which~$L(\{\theta\})$
is a constant, and therefore~$L(\{\theta\} | \script{D}) \approx
L(\script{D} | \{\theta\})$; we retain this convention here.
\par
As mentioned in~\S~\ref{sec:intro}, the sampling of stars from an underlying
evolutionary track in abundance space proceeds according to an IPPP
\citep[e.g.][]{Press2007}.
Due to its detailed nature, we reserve a full derivation of our likelihood
function for Appendix~\ref{sec:likelihood} and provide qualitative discussion
of its form here.
Though our use case in the present paper is in the context of one-zone GCE
models, our derivation assumes only that the chief prediction of the model is
a track of some arbitrary form in the observed space.
It is therefore highly generic and should be easily extensible to other
astrophysical models which predict tracks of some form (e.g. stellar streams
in kinematic space and stellar isochrones on CMDs).
\par
In practice, the evolutionary track predicted by a one-zone GCE model is
generally not known in some analytic functional form (unless specific
approximations are made as in, e.g.,~\citealp{Weinberg2017}).
Instead, it is most often quantified in a piece-wise linear form predicted by
some numerical code (in our case,~\vice).
For a sample~$\script{D} = \{\script{D}_1, \script{D}_2, \script{D}_3, ...,
\script{D}_N\}$ containing~$N$ abundance and age (where available) measurements
of individual stars and a track~$\script{M} = \{\script{M}_1, \script{M}_2,
\script{M}_3, ..., \script{M}_K\}$ sampled at~$K$ points in abundance space,
the likelihood function is given by
\begin{equation}
\ln L(\script{D} | \{\theta\}) = \sum_i^N \ln \left(
\sum_j^K w_j \exp\left(
\frac{-1}{2}\Delta_{ij}C_i^{-1}\Delta_{ij}^T
\right)
\right),
\label{eq:likelihood}
\end{equation}
where~$\Delta_{ij} = \script{D}_i - \script{M}_j$ is the vector difference
between the~$i$th datum and the~$j$th point on the predicted track,~$C_i^{-1}$
is the inverse covariance matrix of the~$i$th datum, and~$w_j$ is a weight to
be attached to~$\script{M}_j$.
This functional form is appropriate for GCE models in which the normalization
of the SFH is inconsequential to the evolution of the abundances; in the
opposingcase where the normalization does impact the predicted abundances,
one additional term subtracting the sum of the weights is required (see
discussion below).
\par
Equation~\refp{eq:likelihood} arises from marginalizing the likelihood of
observing each datum over the entire evolutionary track and has the more
general form of
\begin{subequations}\begin{align}
\ln L(\script{D} | \{\theta\}) &= \sum_i^N \left(
\int_\script{M} L(\script{D}_i | \script{M}) d\script{M}
\right)
\label{eq:likelihood_general_int}
\\
&\approx \sum_i^N \ln \left(
\sum_j^K L(\script{D}_i | \script{M}_j)\right). 
\label{eq:likelihood_general}
\end{align}\end{subequations}
Equation~\refp{eq:likelihood_general} follows from equation
\refp{eq:likelihood_general_int} when the track is densely sampled by the
numerical integrator (see discussion below), and equation~\refp{eq:likelihood}
follows thereafter when the likelihood $L(\script{D}_i | \script{M}_j)$ of
observing the~$i$'th datum given the~$j$th point on the evolutionary track is
given by a weighted~$e^{-\chi^2/2}$ expression.
Mathematically, the requirement for this marginalization arises naturally from
the application of statistical likelihood and an IPPP to an evolutionary track
(see Appendix~\ref{sec:likelihood}).
Qualitatively, it arises due to observational uncertainties -- there is no way
of knowing which point on the evolutionary track the datum~$\script{D}_i$ is
truly associated with, and the only way to properly take this into account is
to consider all pair-wise combinations of~\script{D}~and~\script{M}.
\par
The mathematical requirement for a weighted as opposed to unweighted
$e^{-\chi^2/2}$ likelihood expression also arises naturally in our derivation.
Qualitatively, the weights arise because the likelihood of observing the datum
$\script{D}_i$ is proportionally higher for points on the evolutionary track
when the SFR is high or if the survey selection function is deeper.
For a selection function~\script{S}~and SFR~$\dot{M}_\star$, the weights should
scale as their product:
\begin{equation}
w_j \propto \script{S}(\script{M}_j | \{\theta\})
\dot{M}_\star(\script{M}_j | \{\theta\}).
\end{equation}
Whether or not the weights require an overall normalization is related to the
parametrization of the GCE model -- in particular, if the normalization of the
SFH impacts the abundances or not (see discussion below).
\par
The marginalization over the track and the weighted likelihood are of the
utmost importance to ensure accurate best-fit parameters.
In our tests against mock samples (see~\S~\ref{sec:mocks} below), we are unable
to recover the known evolutionary parameters of input models with discrepancies
at the many-$\sigma$ level if either are neglected.
While these details always remain a part of the likelihood function, equation
\refp{eq:likelihood} can change in form slightly if any one of a handful of
conditions are not met.
We discuss these conditions and the necessary modifications below, referring to
Appendix~\ref{sec:likelihood} for mathematical justification.

\end{document}
