
\documentclass[foo.tex]{subfiles}
\begin{document}

\begin{abstract}
We model the stellar abundances and ages of two disrupted dwarf galaxies in the
Milky Way stellar halo:~\gaia-Sausage Enceladus (GSE) and Wukong/LMS-1.
Using a statistically robust likelihood function, we fit one-zone models of
galactic chemical evolution with exponential infall histories to both systems,
deriving e-folding timescales of~$\tau_\text{in} = 1.01 \pm 0.13$ Gyr for GSE
and~$\tau_\text{in} = 3.08^{+3.19}_{-1.16}$ Gyr for Wukong/LMS-1.
GSE formed stars for~$\tau_\text{tot} = 5.40^{+0.32}_{-0.31}$ Gyr, sustaining
star formation for~$\sim$$1.5 - 2$ Gyr after its first infall into the Milky
Way~$\sim$10 Gyr ago.
Our fit suggests that star formation lasted for
$\tau_\text{tot} = 3.36^{+0.55}_{-0.47}$ Gyr in Wukong/LMS-1, though our sample
does not contain any age measurements.
The differences in evolutionary parameters between the two are qualitatively
consistent with trends with stellar mass~$M_\star$ predicted by simulations and
semi-analytic models of galaxy formation.
Our fitting method is based only on poisson sampling from an evolutionary
track and requires no binning of the data.
We demonstrate its accuracy by testing against mock data, showing that
it accurately recovers the input model across a broad range of sample sizes
($20 \leq N \leq 2000$) and measurement uncertainties
($0.01 \leq \sigma_\afe,~\sigma_\feh \leq 0.5$;
$0.02 \leq \sigma_{\log_{10}(\text{age})} \leq 1$).
Our inferred values of the outflow mass-loading factor reasonably match
$\eta \propto M_\star^{-1/3}$ as predicted by galactic wind models.
Due to the generic nature of our derivation, this likelihood function should
be applicable to one-zone models of any parametrization and easily extensible
to other astrophysical models which predict tracks in some observed space.
\end{abstract}

\end{document}
