
\documentclass[ms.tex]{subfiles}
\begin{document}

\begin{abstract}
We develop a Bayesian method which requires no binning of the data for fitting
one-zone models of galactic chemical evolution to observed stellar abundances
and ages (where available).
Assuming only that the data arise as a stochastic sample from an
arbitrary evolutionary track perturbed by measurement uncertainties, we
demonstrate that a single likelihood function appropriate for one-zone models
can be derived.
We establish the accuracy of this likelihood function by means of tests
against mock data, showing that it accurately recovers the known parameters of
input models across a broad range of sample sizes ($20 \leq N \leq 2000$) and
measurement uncertainties in both abundances ($0.01 \leq \sigma_\text{[X/Y]}
\leq 0.5$) and ages ($0.02 \leq \sigma_{\log_{10}(\text{age})} \leq 1$).
The precision of the inferred parameters generally scales with sample size
as~$\sim N^{-1/2}$.
We apply this method to the~\gaia-Sausage Enceladus (GSE) and the Wukong stellar
streams in the Milky Way halo using data from the H3 survey and parametrize
their evolution as an exponential infall history.
We derive~$\tau_\text{in} = 1.01 \pm 0.13$ Gyr with
$\chi_\text{dof}^2 \approx 1.34$ for GSE and
$\tau_\text{in} = 3.47^{+3.58}_{-1.36}$ Gyr with
$\chi_\text{dof}^2 \approx 0.98$ for Wukong.
In principle, this method allows accurate derivations of evolutionary
timescales even in the absence of age measurements, though in practice either
ages or an empirically-motivated prior on the star formation history are
required for the application to observed data.
Due to the first principles nature of its derivation, our likelihood function
should be universally applicable to one-zone models of any parametrization.
\end{abstract}

\end{document}
